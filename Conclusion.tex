\chapter{Conclusions}
\label{sec:Conclusions}
\chaptermark{Conclusions}

A set of analysis tools which can be used to enhance the
sensitivity of diboson signatures as well as study resonance
properties has been developed.  Two specific implementations 
of these tools have been presented in the context of searches
for a Higgs boson. 

A search for a SM Higgs boson using $ZZ^{(*)}\to2\ell2q$ events
was presented.  
Drawing on the ideas presented in chapter~\ref{sec:HiggsPhen},
a novel discriminant was used to reduce the dominant SM
background. 
Techniques for measuring expected background
shapes and event yields using data control regions were used.
No significant deviation from the background 
only hypothesis were found and upper limits were set.
Standard model Higgs boson mass hypotheses between XXX and YYY
were ruled out at 95\% condifence level.

A search for a SM Higgs boson using $ZZ^{(*)}\to4\ell$ events
was presented.  Again, ideas from chapter~\ref{sec:HiggsPhen} 
were used to build discriminants to further enhance sensitivity
to signal events.  A local excess of events near 126~\GeV was
observed.  The MELA technique was found to provide a
significant increase of the observed significance of this 
excess which is found to be $6.8\sigma$ at 125.7~GeV .
At other masses, no significant excesses was observed 
and Higgs mass hypotheses in the range [114.5,119] and [129-800]
were ruled out at 95\% confidence level. 

Other discriminants were designed to test the
compatibility of excess in data with respect to either
a SM Higgs or
a number of alternative hypotheses using event kinematics.  All 
test show that data is prefers the SM Higgs hypotheses over
the alternative hypotheses.  In the specific case of a
pseudoscalar alternative hypothesis, data disfavors this
hypothesis at the of 0.04\%.  These property
measurements have helped to shape our understanding of the
observed resonance in that we have concrete evidence that 
this resonance is likely to be a scalar.  

As demonstrated in chapter~\ref{sec:HiggsPhen}, it is also
possible to constrain the contribution of CP-violating 
interactions by measuring the parameter $f_{3}$.  The best-fit
value of this parameter is found to be $f_{3}=0.00^{+0.16}_{-0.00}$
which is consistent with SM expectation.  The 95\% confidence 
interval of this parameter is found to be [0.00,0.49].

As shown in section~\ref{sec:HZZ4l} and summarized in
figure~\ref{fig:spinParitySummary}, all property measurements
done thus far using ZZ kinematics suggest that this resonance
is consistent with a SM Higgs boson.   Similar measurements
were performed using WW events to access the compatibility 
of data with eith a SM Higgs boson or a minimal coupling graviton.
This result has also been combined with the ZZ result by performing
simultaneous fits in both channels\cite{CMS:yva}.  The results of this is shown
in figure~\ref{fig:2mp_spinCombination}.  The median toys of
the SM Higgs distribution
has a $CL_s$ value of 1.25\%, corresponding to an average
separation of $3.0\sigma$.  The data is found to disfavor the
minimal coupling graviton with a $CL_s$ value of 0.6\%, compared
to the observed $CL_s$ of 1.3\%\footnote{Note that this results
corresponds to an earlier version of the analysis~\cite{CMS:xwa}}
and 6.8\% using the ZZ and WW channels alone.  Other measurements
performed by the ATLAS collaboration~\cite{ATLAS:2013mla,ATLAS:2013mla} using the same tools
developed in chapter~\ref{secHiggsPhen} also support the results
presented in section\ref{sec:HZZ4l}.



\begin{figure}
\begin{center}
\includegraphics[width=.89\linewidth]{ConclusionPlots/JP_SummaryPlot.eps}
\caption{Distribution of test statistics for SM Higgs toys (blue),
 alternative $J^P$ signals toys (orange), and the observed test
statistic (points).}
\label{fig:spinParitySummary}

\end{center}
\end{figure}
\begin{figure}
\begin{center}
\includegraphics[width=.49\linewidth]{ConclusionPlots/hvv_a-posteriori.qvals.root_2pmgg.eps}
\caption{Distributions of the test statistic comparing the 
SM Higgs hypothesis against the $J^P=2_m^+$ hypothesis using a
simultaneous fit of the signal strength in the ZZ and WW channels.
The orange distribution represents the SM Higgs toys, the blue 
distribution represents the $2_m^+$ hypothesis.  The red arrow
show the observed test statistic. }
\label{fig:2mp_spinCombination}
\end{center}
\end{figure}

Cross section measurements in other channels also support the 
SM Higgs hypothesis~\cite{CMS:yva}.  The left plot of
figure~\ref{fig:crossSectionByChannel} shows the best-fit signal 
strength of each decay channel separately.  All are statistically
consistent with the the SM Higgs hypothesis, $\mu=1$.  
The best-fit signal for different production mechanism is shown
in the right plot of figure~\ref{fig:crossSectionsByChannel}.

\begin{figure}
\begin{center}
\includegraphics[width=.49\linewidth]{ConclusionPlots/sqr_mlz_ccc_mH1257_decay.eps}
\includegraphics[width=.49\linewidth]{ConclusionPlots/sqr_mlz_ccc_mH1275_prod.eps}
\caption{Best-fit signal strength modify, $\mu$, for various 
production and decay modes.  Red error bar represent the 68\%
condifence interval of the individual measurements.  Black lines
represent the combined measurement of all channels (production and
decay); the green band represents the the 68\% confidence interval.
All fits are done for a fixed mass hypothesis, $m_H=125.7$~GeV, 
which correspond to the combined best-fit value.}
\label{fig:crossSectionsByChannel}
\end{center}
\end{figure}

The natural evolution of the tools used in 
section~\ref{sec:HZZ4l} were discussed in 
chapter~\ref{sec:FutureMeasurements}.  
The advantages and disadvantages of using either the MELA 
technique or multidimensional fits were compared.  Projected
sensitivities were estimated for high luminosity LHC scenarios
and future colliders.  These projections suggest that
other Higgs processes, such as 
$q\bar{q}\to ZH$ or $q\bar{q}\to Hq\bar{q}$, will play 
an important role in the campaign for precision measurements
of Higgs properties.  
  
The tools developed in this thesis, have provided emmense utility
to Higgs physics.  
Similar measurements using similar tools
are now being done in multiple di-boson final states in both 
the CMS and ATLAS collaborations.  As demonstrated, these 
tools will also help to transition into a the next generation
of measurements and, hopefully, one day help us to discover
physics beyond the SM.  




