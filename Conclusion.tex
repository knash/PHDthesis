\chapter{Conclusions}
\label{sec:Conclusions}
\chaptermark{Conclusions}

A set of analysis tools which can not be used to enhance the
sensitivity of diboson signatures as well as to study resonance
properties have been developed.  Two specific implementations 
of these tools have been presented in the context of searches
for a Higgs boson. 

A search for a SM Higgs boson using $ZZ^{(*)}\to2\ell2q$ events
was presented.  
Drawing on the ideas presented in Chapter~\ref{sec:HiggsPhen},
a novel discriminant was used to reduce the dominant SM
background. 
Techniques for measuring expected background
shapes and event yields using data control regions were used.
No significant deviation from the background 
only hypothesis was found and upper limits were set.
Standard Model Higgs boson masses between 340 and 390~GeV
were ruled out at 95\% confidence level.

A search for a SM Higgs boson using $ZZ^{(*)}\to4\ell$ events
was presented.  Again, ideas from Chapter~\ref{sec:HiggsPhen} 
were used to build discriminants to further enhance sensitivity
to signal events.  These techniques have been an integral part
of the $ZZ\to4\ell$ analysis at CMS since the discovery of the 
Higgs-like resonance in July of 2012.  
Now, an excess of events is observed with a local significance
of $6.8\sigma$ at 125.7~GeV.  
At other masses, no significant excesses were observed 
and Higgs boson masses in the range [114.5,119] and [129-800]
were ruled out at 95\% confidence level. 

Other MELA discriminants were designed to test the
compatibility of the excess in data with respect to either
a SM Higgs boson or a number of signal models.  All 
tests show that data prefers the SM Higgs hypothesis over
the alternative hypotheses.  Most notably data disfavors the 
pseudoscalar model at the level of 0.04\%.  These property
measurements are summarized in Figure~\ref{fig:spinParitySummary}.
%have helped to shape our understanding of the
%observed resonance in that they strongly suggest that 
%this resonance is likely to be a scalar.
The contributions of CP-violating 
interactions were constrained through the measurement of $f_{a3}$.
The best-fit value of this parameter is found to be 
$f_{a3}=0.00^{+0.16}_{-0.00}$ which is consistent with SM expectation.  The 95\% confidence interval of this parameter is found to be 
[0.00,0.49].

\begin{figure}
\begin{center}
\includegraphics[width=.89\linewidth]{ConclusionPlots/JP_SummaryPlot.eps}
\caption{Distribution of test statistics for SM Higgs toys (blue),
 alternative $J^P$ signals toys (orange), and the observed test
statistic (points).}
\label{fig:spinParitySummary}

\end{center}
\end{figure}

Hypothesis separation measurements
were also performed using WW events for testing the minimal coupling
graviton model.
This result has been combined with the ZZ result by performing
simultaneous fits in both channels\cite{CMS:yva}.  The result
is shown in Figure~\ref{fig:2mp_spinCombination}.  The median
toy of the SM Higgs distribution
has a $CL_s$ value of 1.25\%, corresponding to an average
separation of $3.0\sigma$.  The data is found to disfavor the
minimal coupling graviton with a $CL_s$ value of 0.6\%, compared
to the observed $CL_s$ of 1.3\%\footnote{Note that this results
corresponds to an earlier version of the analysis~\cite{CMS:xwa}.
The most up to date ZZ/WW combination does not exist yet.}
and 6.8\% using the ZZ and WW channels alone.  Other measurements
performed by the ATLAS collaboration~\cite{ATLAS:2013mla,ATLAS:2013nma} using the same ideas developed in Chapter~\ref{sec:HiggsPhen}
are consistent with those presented in Section~\ref{sec:HZZ4l}.

\begin{figure}
\begin{center}
\includegraphics[width=.49\linewidth]{ConclusionPlots/hvv_a-posteriori.qvals.root_2pmgg.eps}
\caption{Distributions of the test statistic comparing the 
SM Higgs hypothesis against the $J^P=2_m^+$ hypothesis using a
simultaneous fit of the signal strength in the ZZ and WW channels.
The orange distribution represents the SM Higgs toys, the blue 
distribution represents the $2_m^+$ hypothesis.  The red arrow
shows the observed test statistic. }
\label{fig:2mp_spinCombination}
\end{center}
\end{figure}

Cross section measurements in other channels also support the 
SM Higgs hypothesis~\cite{CMS:yva}.  The left plot of
Figure~\ref{fig:crossSectionsByChannel} shows the best-fit signal 
strength of each decay channel separately.  The best-fit signal
for different production mechanisms is shown
in the right plot of Figure~\ref{fig:crossSectionsByChannel}.
All are consistent with the the SM Higgs hypothesis,
$\mu=1$.  
As described in Chapter~\ref{sec:HiggsPhen}, it is expected that
the fermionic couplings to the Higgs field will scale with the
mass of the fermion while the bosonic couplings to the Higgs field
will scale with the square of the vector boson's mass.  
Figure~\ref{fig:couplingMeasurements} shows the best-fit
fermionic coupling and the square-root of the bosonic couplings
divided by twice the Higgs vacuum expectation value.
All couplings measured thus far are consistent with a linear
correlation between the couplings and the masses.  

\begin{figure}
\begin{center}
\includegraphics[width=.49\linewidth]{ConclusionPlots/sqr_mlz_ccc_mH1257_decay.eps}
\includegraphics[width=.49\linewidth]{ConclusionPlots/sqr_mlz_ccc_mH1257_prod.eps}
\caption{Best-fit signal strength modifier, $\mu$, for various 
production and decay modes.  Red error bars represent the 68\%
confidence interval of the individual measurements.  Black lines
represent the combined measurement of all channels (production and
decay); the green band represents the the 68\% confidence interval.
All fits are done for a fixed mass hypothesis, $m_H=125.7$~GeV, 
which correspond to the combined best-fit value.}
\label{fig:crossSectionsByChannel}
\end{center}
\end{figure}

\begin{figure}
\begin{center}
\includegraphics[width=.49\linewidth]{ConclusionPlots/sqr_m6summary_run.eps}
\caption{Summary of the fits for deviations in the coupling for the generic five-parameter model not including effective loop couplings, expressed as function of the particle mass. For the fermions, the values of the fitted Yukawa couplings hff are shown, while for vector bosons the square-root of the coupling for the hVV vertex divided by twice the vacuum expectation value of the Higgs boson field. Particle masses for leptons and weak boson, and the vacuum expectation value of the Higgs boson are taken from the PDG. For the top quark the same mass used in theoretical calculations is used (172.5 GeV) and for the bottom quark the running mass $m_b$($m_H=125.7$~GeV)=2.763~GeV is used.}
\label{fig:couplingMeasurements}
\end{center}
\end{figure}

The measurements discussed above strongly suggest that the 
resonance observed is a scalar which participates in
electroweak symmetry breaking.  Extensions to the SM which 
fall under the generic class of 2HDM provide an interesting
framework to further study the Higgs sector.  These models
predict two more neutral scalar bosons and could lead to 
CP-violating interactions.  As discussed in Chapter~\ref{sec:intro},
this could help to explain the baryon asymmetry problem
or even dark matter if the specific 2HDM turns out
to be SUSY.

Although CMS measurements have begun to constrain the presence
of CP-violating interactions by setting limits on $f_{a3}~(f_{g4})$, 
these measurements still have large uncertainties.  However, 
the same tools which are currently being used in the $H\to ZZ$ 
process could be applied to other processes at either the LHC or a
future $e^+e^-$ collider.    Projected sensitivities were
estimated for high luminosity LHC scenarios
and future colliders in Chapter~\ref{sec:FutureMeasurements}.
These projections suggest that other Higgs processes, such as 
$q\bar{q}\to ZH$ or $q\bar{q}\to Hq\bar{q}$, will play 
an important role in the campaign for precision measurements
of Higgs properties.   

Other mechanisms for electroweak symmetry breaking include 
models in which the Higgs is composite.  Measuring all of
the HZZ amplitude parameters may one day provide
hints of compositeness.  However, it is necessary to use 
more advanced techniques in order to measure all parameters.
Multidimensional fits provide the necessary flexibility to do
so and are a natural evolution of the MELA technique.
  
The MELA techniques have provided immense utility
to the high energy physics community.  These tools have been used 
to discover and characterize the 126~GeV Higgs-like
resonance both at CMS and ATLAS~\cite{ATLAS:2013nma}.  The property
measurements made have helped to shape our understanding of the
role this resonance plays in nature and whether new physics is
involved in its interactions with the SM fields.  Even in the next 
generation of experiments, the MELA techniques will
continue to provide a framework for performing high precision
measurements and, hopefully, one day help us to better understand
the universe we live in.



