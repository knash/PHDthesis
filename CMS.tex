\chapter{The Compact Muon Solenoid}
\label{sec:CMS}
\chaptermark{The Compact Muon Solenoid}

The Compact Muon Solenoid (CMS), is a general purpose particle detector.  It
was designed to not only have a broad scope of discovery potential but also
to mitigate the extreme conditions created by the LHC.  CMS is made up of
several different types of apparati, designed to improve identification of 
particles and measure their properties.  There is a two-stage trigger system
to filter the extreme rates comming from the LHC.  There is an all silicon tracking 
system at the center to carefully record the positions of charged particles passing 
through the detector.  There is a 4 Tesla magnet to bend charge particles 
providing the tracker and muon system sensitivity to the momentum of charged 
particles.  There are two calorimeters designed to induce particle showers which 
can then be used to measure energy deposits.  Finally, there is a Muon system 
at the edge of the detector to detect semi-stable, charged particles with long 
interaction lengths, e.g. the muon. This chapter describes these detectors in
detail. 

\section{Tracker}
\label{sec:Tracker}

The CMS tracker is an all silicon detector that consists of more than 16,000
individual silicon modules.  These modules are of two basic variaties, pixels
which provide a 2-dimensional measurement of particle positions and strips
which provide 1-dimensional measurements of particles positions.  The tracker
is the closest sub-detectro to the interaction point.  As such, it is exposed to
the highest radiation flux and must be radiation hard to survive the extreme
conditions of the LHC.  The occupancy of these detectors ... UPDATE!!!!
%%%%%%%%% UPDATE!! %%%%%%%%%%%%%
\subsection{Tracker Geometry}

The tracker extends to 1.1 m in the radial diections,  has a total legnth 
of 5.4 m, and extends to 2.5 units of pseudo-rapidity.  The pixel modules
are located closest to the interaction point and are distributed over three
layers in the barrel (BPix) and two disk at either end (PEC).  These modules
provide the most sensitivity to vertexing and are important for migitate 
pileup effects and measuring discplaced vertices.  

\subsection{Pixel Modules}



\subsection{Strip Modules}


\section{Magnet}
\label{sec:Magnet}

The CMS magnet ...

\section{ElectroMagnetic Calorimeter}
\label{sec:ECal}

The ECal ...

\section{Hadronic Calorimeter}
\label{sec:HCal}

The HCal ...

\section{Muon System}
\label{sec:Muon System}

The Muon System ...
