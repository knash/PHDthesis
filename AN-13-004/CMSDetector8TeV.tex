\chapter{CMS Detector}
\label{sec:cms_detector}
CMS ~\cite{cmsexp} is a general purpose detector at the Large Hadron Collider.  The central 
feature of CMS is a solenoid capable of creating a uniform, axial magnetic field of 
3.8T.  Within the solenoid is an advanced charged particle tracking system consisting 
of an inner pixel detector with three layers extending from 4.4 to 10.2cm, and an 
outer silicon strip tracker which extends out to 110cm.  The trackers are also equipped 
with endcap segments that allow the detector to cover an azimuthal range of 
$|\eta| < 2.5$, where $\eta$ refers to  pseudorapidity. 
 
Outside of the tracker, the CMS detector houses two essential calorimeter systems.  
The fine granularity lead tungstate electromagnetic calorimeter (ECAL) allows the 
reconstruction of charged particles and photons.  The brass scintillator Hadronic 
Calorimeter (HCAL) allows for reconstruction of jets.  The HCAL and ECAL both allow an 
acceptance of $|\eta| < 3$.  The energies of the cells within the HCAL and ECAL are summed 
to create towers, which are then clustered and identified as jets.  The information provided 
by the tracking system and calorimeters are combined in an algorithm called particle flow.
