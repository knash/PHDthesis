\chapter{Analysis Strategy}
\label{sec:analysisStrategy}

%The primary focus of this analysis is a high mass resonance ($>1.3$~\TeV) decaying to $\tbbar$. At higher energies, the top decay products merge into a single jet. We 
%consider the case of a top decaying in the fully hadronic decay chain $t \to W+b$ followed by $W \to$ hadrons. These high energy final states lead to an inefficient 
%determination of the limits on production of $M_{\wpr}$ with masses below  $1.3~\TeV$, which are not included in the analysis.  

The focus of this analysis are heavy resonances decaying into
$\tbbar$.  Thus the $\wpr \to \tbbar$ decay results in a mostly
dijet topology, with the b and top jets being  predominantly back-to-back. 
% Events that have this topology are referred to as ``type
% b+1'' referring to the completely merged top candidate as a ``type 1''
% top and the b candidate jet in the opposite hemisphere.  
% The event topology
% is similar to reference \cite{7tevZprime}, however this analysis does
% not include the ``type 1+2'' topology, due to the negligible signal
% efficiency in the kinematic region of interest.  The type b+1 analysis
% is then a dijet search (one jet in each hemisphere).

After requiring a high transverse momentum, 
the primary sources of background are QCD multijet and SM $\ttbar$ 
production due to the high abundance of QCD present by requiring a dijet topology and the high $\ttbar$ production contribution fraction that remains after top jet discrimination. 

Of these two main sources, QCD multijet production is dominant and is estimated by a data driven technique similar 
to \cite{7tevZprime}.  We
invert certain substructure cuts used to identify top jets
to define sideband regions; we keep the cut on the mass of the
top jet to avoid the kinematic bias in forming the invariant mass
distribution of the top-b candidate.  One sideband region is used
to measure the average b-tagging rate, which is then applied to
pre-b-tag data to obtain the QCD estimate.  The other sidebands are
then used to check the performance of the QCD estimation in data.

The SM $\ttbar$ contribution is estimated from Monte Carlo (MC) simulation.
We also subtract the pre-tagged $\ttbar$ from the pre-tagged data sample when
measuring the average b-tagging rate.  The measurement
of the average b-tagging rate is then independent of $\ttbar$, and the 
$\ttbar$ contribution is added at the end as a separate background
component.
The contribution of the single top production to the background was
found to be negligible.

% &&& need to put the study proving this into the appendix

The data and background components are then used as
templates by the Bayesian statistical procedure 
to set the exclusion limits on $\wpr$.  This procedure uses a binned likelihood to calculate limits for the signal cross-section 
in the $\wpr \to \tbbar$ branching fraction.  
We use the limit setting framework ``Theta'' \cite{theta}, which calculates exclusion limits using a shape based approach.
