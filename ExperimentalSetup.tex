\chapter{Experimental Setup}
\label{sec:ExpSetup}
\chaptermark{Experimental Setup}


\section{The Large Hadron Collider}
\label{sec:LHC}
The Large Hadron Collider was designed to accelerates protons to up to 
energies of 7 TeV using a 26 km storage ring and  1232  8.33 T dipole magnets. 
Although it is also capable of accelerating heavier nuclei up to energies of 
2.76 TeV, heavy ion physics is outside the scope of this work.  The proton
energies accessible to the LHC are a factor of seven times higher than its
most advanced predicessor, the Tevatron.  These energies are not only 
important for accessing new particles which might exist at large invariant
mass, on the order of serveral TeV, they are also necessary for efficient 
production of the Higgs boson.  In the more favorable region of the SM 
Higgs mass,  $m_{H}$, parameter space, these energies provides a 
factor of XX in total cross section over the production cross section at
the Tevatron.  

The LHC also has the capability to collide bunches of $1\times10^11$
protons every 25 ns at $\beta*=.55$ and $\sigma*=16.7$.  These 
conditions combine to allow the LHC to produce instantaneous luminosities 
of up to $10^34 cm^{-2}s^{-1}$.  This translates to roughly 1 billion proton
proton interactions per second and 20 collisions per bunch crossing, known
as pileup.  These
parameter are summarized in table \ref{blah} for both proton-proton (pp)
and heavy ion (HI) collisions. 

The LHC is only responsible for accelerating hadrons from 450 GeV to the 
final energy.  Before hadrons are injected into the LHC, they are first 
accelerated in two separated stages.  Once up to 26 GeV in the Proton
Synchrotron (PS) with the correct 25 ns bunch spacing and then again
to 450 GeV using  the Super Proton Synchrotron before being injected 
into the LHC.  This process is done 12 times per beam.  

These conditions provide unprecidented conditions to probe the SM and 
discovering new particles but also extreme conditions for reconstructing
objects with a high degree of efficiency.  The large number of proton-proton
collisions produce a considerable amount of background noise which can 
either produce extra objects from secondary interactions as well an overall
increase in the energy deposited in the calorimeters.  The  high rate of the 
LHC far exceed the capabilities of the Data Aquisition (DAQ) system.  As a 
result it is necessary to use fast hardware logic to filter the majority of events. 
This system is know as the leve-1 trigger system.  

The short time between bunch crossings also puts significant constraints on 
detectors since subdetectors should have fast response times and lot occupancy. 
High granularity tracking will be necessary for high precision vertexing in
order to mitigate the effects of pileup.




\subsection{Injectors}
\label{sec:Injectors}

The LHC uses several smaller pre-accelerators to produce beam of protons with 
a minimal energy ...

\section{The Compact Muon Solenoid}
\label{sec:CMS}

The Compact Muon Solenoid (CMS), is a general purpose particle detector.  It
was designed to not only have a broad scope of discovery potential but also
to mitigate the extreme conditions created by the LHC.  CMS is made up of
several different types of apparati, designed to improve identification of 
particles and measure their properties.  There is a two-stage trigger system
to filter the extreme rates comming from the LHC.  There is an all silicon tracking 
system at the center to carefully record the positions of charged particles passing 
through the detector.  There is a 4 Tesla magnet to bend charge particles 
providing the tracker and muon system sensitivity to the momentum of charged 
particles.  There are two calorimeters designed to induce particle showers which 
can then be used to measure energy deposits.  Finally, there is a Muon system 
at the edge of the detector to detect semi-stable, charged particles with long 
interaction lengths, e.g. the muon. This chapter describes these detectors in
detail. 

\subsection{Tracker}
\label{sec:Tracker}

The CMS tracker is an all silicon detector that consists of more than 16,000
individual silicon modules.  These modules are of two basic variaties, pixels
which provide a 2-dimensional measurement of particle positions and strips
which provide 1-dimensional measurements of particles positions.  The tracker
is the closest sub-detectro to the interaction point.  As such, it is exposed to
the highest radiation flux and must be radiation hard to survive the extreme
conditions of the LHC.  The occupancy of these detectors ... UPDATE!!!!
%%%%%%%%% UPDATE!! %%%%%%%%%%%%%
\subsubsection{Tracker Geometry}

The tracker extends to 1.1 m in the radial diections,  has a total legnth 
of 5.4 m, and extends to 2.5 units of pseudo-rapidity.  The pixel modules
are located closest to the interaction point and are distributed over three
layers in the barrel (BPix) and two disk at either end (PEC).  These modules
provide the most sensitivity to vertexing and are important for migitate 
pileup effects and measuring discplaced vertices.  

\subsubsection{Pixel Modules}



\subsubsection{Strip Modules}


\subsection{Magnet}
\label{sec:Magnet}

The CMS magnet ...

\subsection{ElectroMagnetic Calorimeter}
\label{sec:ECal}

The ECal ...

\subsection{Hadronic Calorimeter}
\label{sec:HCal}

The HCal ...

\subsection{Muon System}
\label{sec:Muon System}

The Muon System ...


\subsection{Electron Reconstruction and Identification}
\label{sec:HZZ4lelectrons}

\subsection{Muon Reconstruction and Identification}
\label{sec:HZZ4lmuons}

\subsection{Photon Reconstruction and Identification}
\label{sec:HZZ4lphotons}

\subsection{Jet Reconstruction and Identification}
\label{sec:HZZ4ljets}

\subsection{Datasets and Triggers}
\label{sec:HZZ4ldatasets}
