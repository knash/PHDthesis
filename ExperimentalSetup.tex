\newcommand{\microns}{$\mu m$}


\chapter{Experimental Setup}
\label{sec:ExpSetup}
\chaptermark{Experimental Setup}


\section{The Large Hadron Collider}
\label{sec:LHC}
The Large Hadron Collider was designed to accelerate two beams of protons up to 
energies of 7 TeV using a 27 km storage ring and  1232 individual 8.33 T dipole magnets. 
Although it is also capable of accelerating heavier nuclei up to energies of 
2.76 TeV, heavy ion physics is outside the scope of this work.  The proton
energies accessible to the LHC are a factor of seven times higher than its
most advanced predecessor, the Tevatron.  These energies are not only 
important for accessing new particles which might exist at large invariant
mass, on the order of several TeV, they are also necessary for efficient 
production of moderately heavy particles, like the Higgs boson or the top 
quark.  For a 125~GeV Higgs boson these energies provide a 
factor of $\sim50$ in total cross section over the production cross section at
the Tevatron.  

The LHC also has the capability to collide bunches of $1\times10^{11}$
protons every 25 ns at $\beta^*=.55$ and $\sigma^*=16.7$.  These
parameters and others, summarized in Table~\ref{table:LHCparameters},
combine to allow the LHC to produce instantaneous luminosities 
of up to $10^{34} cm^{-2}s^{-1}$ according to
\begin{equation}
\mathscr{L} = \frac{\gamma f k_B N_p^2}{4\pi \epsilon_n\beta^*}F,
\end{equation}
where $\gamma$ is the Lorentz factor, $f$ is the revolution frequency,
$k_B$ is the number of protons per bunch, $\epsilon_n$ is the betatron
function at the interaction point, and $F$ is the reduction factor
due to the crossing angle.  This translates to roughly 1 billion 
proton-proton interactions per second and up to 50 collisions per bunch crossing.

\begin{table}
\begin{center}
\begin{tabular}{l|c|c}
\hline 
\hline
Energy per nucleon           & $E$           & 7 TeV    \\
Dipole field at 7 TeV        & $B$           & 8.33 T   \\
Design luminosity            & $\mathscr{L}$ & $10^{34}~cm^{-2}s^{-1}$\\
Bunch separation             &               & 25 ns    \\
No. of bunches               & $k_B$         & 2808      \\
No. of particles/bunch       & $N_p$         & $1.15\times 10^{11}$\\ \hline
\multicolumn{3}{l}{{\bf Collisions}} \\ \hline
$\beta$-value at IP          & $\beta^*$     & 0.55 m   \\
RMS beam radius at IP        & $\sigma^*$    & $16.7~\mu m$  \\
Luminosity lifetime          & $\tau_L$      & 15 hr         \\
Number of collisions/crossing& $n_c$         & $\equiv20$      \\
\hline 
\hline
\end{tabular}
\caption{Relevant operational parameters of the LHC.}
\label{table:LHCparameters}
\end{center}
\end{table}

These conditions provide the necessary environment to probe
the SM and discover new particles, but also an extreme environment
for reconstructing particle paths and energy deposits with a high
degree of accuracy and efficiency. The inclusive proton-proton
cross section at 14~TeV is approximately 100~mb, roughly 10 orders
of magnitude larger than the largest Higgs cross sections.  
At design luminosity, this corresponds to an event rate of
$10^9$~Hz.  
The large number of proton-proton collisions produce a considerable
amount of background noise which can produce extra
particles from secondary interactions, also known as pileup, as
well an overall increase in the energy deposited in the 
calorimeters.  The  high rate of collisions at the LHC far exceeds the
capabilities of the CMS Data Acquisition (DAQ) system.  As a 
result it is necessary to use fast hardware logic to filter
the vast majority of events. 
The short time between bunch crossings also puts significant
constraints on detector design since sub-detectors should have fast
response times and low occupancy. High granularity tracking will
be necessary for high precision vertexing in order to mitigate the
effects of pileup.

\section{The Compact Muon Solenoid}
\label{sec:CMS}

The Compact Muon Solenoid (CMS), is a general purpose particle
detector.  It was designed to not only have a broad scope of
discovery potential but also to mitigate the extreme conditions
created by the LHC.  CMS is made up of several different types of
apparatuses designed to improve identification of particles and
measure their properties.  There is a two-stage trigger system
to filter the extreme rates coming from the LHC.  There is an all
silicon tracking system at the center to carefully record the 
positions of charged particles passing through the detector.  
There is a 4 Tesla magnet to bend charged particles providing the
tracker and muon system sensitivity to the momentum of charged 
particles.  There are two calorimeters designed to induce particle
showers which can then be used to measure energy deposits.  
Finally, there is a Muon system at the edge of the detector to
detect semi-stable, charged particles with long interaction
lengths, e.g. the muon. This chapter provides a brief description
of these sub-detector.  

\subsection{Magnet}
\label{sec:Magnet}

CMS employs a 4~T superconducting aluminum solenoid magnet to bend
tracks for both charge identification and momentum resolution.  
The field strength was chosen to have good momentum resolution, 
$\Delta p/p\equiv10\%$ at $p=1~TeV/c$.  The magnet has an inner 
bore of 5.9~m, large enough to house the tracker and both 
calorimeters, and a length of 12.9~m.  Drawing a current of 
19.5~kA, the magnet's total stored energy is 2.7~GJ, making it
one of the largest magnets in the world.  The outer return yolk 
of the magnet concentrates the magnetic field in the
region near the muon system, which is placed outside of the 
solenoid.  


\subsection{Trigger and data acquisition}

The event rate delivered to CMS is approximately $10^9$~Hz.  
However, only about 100~Hz can be processed by CMS.  
This requires a large, yet efficient, rejection scheme.  
CMS employs a two level system to make fast decisions on 
which events to record.  The level-1 system consists of 
custom electronics which monitor the activity in the calorimeters
and the muon system.   Decisions are based on raw energy
and momentum thresholds.  The level-1 system reduces the
event rate down to roughly 100~kHz while the High-Level 
Trigger (HLT), an on-line
processing farm which executes reconstruction software, 
further reduces the rate to 100~Hz.  Customized HLT selections
are designed to ensure high efficiencies for different 
physics signatures.  

\subsection{Electromagnetic Calorimeter}
\label{sec:ECal}

The Electromagnetic Calorimeter (ECal) is a high granularity
calorimeter intended to induce electromagnetic showers which 
are collected by either avalanche photodiodes (barrel) or vacuum 
phototriodes (endcap).  The material used is scintillating lead 
tungstate crystal which was chosen for its: short radiation 
($X_0$=0.89~cm) and Moliere (2.2~cm) lengths; the time scale in 
which showers occur (80\% of light is emitted in 25~ns); and the
radiation hardness.  The ECal is divided into barrel (EB)
and endcap (EE) regions.

The EB region has an inner radius of 129~cm and is constructed 
from 36 identical {\it supermodules}, each covering half of the 
barrel in the z-direction (1.479 unit of pseudorapidity).  Each 
individual crystal covers 1 degree in both $\Delta\phi$ and 
$\Delta\eta$, corresponding to a cross sectional area of 
$22\times22~mm^2$, and is 230~mm long, corresponding to 25.8~$X_0$.

The EE region is located at a distance of 314~cm along the 
z-direction and covers the pseudorapidity range 
$1.479<|\eta|<3.0$.  The crystals are clustered into $5\times5$ 
{\it supercrystals} which are combined to form semi-circular 
structures.  Each crystal has a cross sectional area of 
$28.6\times28.6~mm^2$ and is 220 mm (24.7~$X_0$) in length.  
The endcap region is also preceded by a preshower which consists 
of a lead absorber whose thickness is 2-3~$X_0$ followed by 2 
planes of silicon strip detectors.

The energy response of the ECal was measured in test beams.  
The energy resolution was parameterized according to
\begin{equation}
\left(\frac{\sigma}{E}\right)^2 = \left(\frac{S}{\sqrt{E}}\right)^2 + \left(\frac{N}{E}\right)^2 + C^2,
\end{equation}
where S, N and C represent the stochastic, noise, and constant 
contributions.  
%The coefficients measurements for two energy 
%clustering schemes are shown if figure~\ref{fig:???}.

\subsection{Hadronic Calorimeter}
\label{sec:HCal}

The hadronic calorimeter (HCal) consists of brass absorbers and 
plastic scintillators in which light is collected from 
wavelength-shifting fibers. Fiber cables transmit light into 
hybrid photodiodes.  The HCal is separated into four regions:
the barrel (HB), the outer (HO), the endcap (HE), and the forward 
(HF) regions.

The HB is made up of 32 towers which cover the pseudorapidity 
region $|\eta|<1.4$, totaling 2304 towers with a segmentation 
of $\Delta\eta\time\Delta\phi=0.087\time0.087$.  There are 15 
brass plates, each 5~cm thick and two steel plates for structural 
stability. Particles entering the HCal barrel region first impinge 
upon a scintillating layer that is 9~mm thick, instead of the 
typical 3.7~mm for other scintillating layers. More details of the 
HB design and test beam performance can be found 
elsewhere~\cite{Bayatian:2006zz,Ball:2007zza}.

The HO region contains 10~mm thick scintillators.  Each 
scintillating tile matches the segmentation pattern of the muon 
system's Drift tubes.  The purpose of the HO is to catch hadronic 
showers leaking through the HB region. This makes the effective 
length of the barrel region 10~$X_0$ and improves missing 
transverse energy $E_T^{miss}$ resolution.

The HE region consists of 14 $\eta$ towers with 5 degree
segmentation in $\phi$ and covers the region between 
$1.3<|\eta|<3.0$.  There are 2304 towers in total.  
The HF region extends between $3.0<|\eta|<5.0$ and is made from steel absorbers
and quartz fibers.  The fibers are intended to measure Cherenkov radiation.  
The HF will mainly be used for detecting very forward jets and real-time
luminosity measurements.  
More details of
the design and test beam performance of the HE and HF can be found 
elsewhere~\cite{Bayatian:2006zz,Ball:2007zza}.


\subsection{Muon System}
\label{sec:Muon System}

The Muon system plays an important role in identifying muons.  However, 
because of the vast distance from the interaction point and the muon 
chambers, resolution of low energy muons is dominated by energy loss 
due to multiple scattering in the inner detector.  In this region, it 
is found that the tracker dominates the momentum resolution.  However, 
for muons above $\sim100$~GeV, the combination of the tracker and muon systems
provides superior energy resolution to either system alone. Thus,
the muon system plays a major role in momentum resolution of high
momentum muons.

The muon system employs three different gaseous detectors, drift
tube (DT) chambers, cathode strip chambers (CSC), and resistive
plate chambers (RPC).  The DT chambers are used in the barrel region, 
$|\eta|<1.2$, where the magnetic field is low.  The CSC detectors
are used in the endcaps, $1.2<|\eta|<2.4$, where the rates of both
rates and the magnetic field is high. The RPC detectors are used
both in barrel and endcaps.  

The RPCs are fast response detectors with good timing resolution,
although not as sensitive to spatial measurements as the DTs and
CSCs.  Thus, RPCs provide the necessary input to distinguish
which bunch crossing a particle should be identified with, which
is critical for triggering.  All three sub-systems provide a key
element to level-1 triggering.

The DTs are arranged in four layers of wheels made up of 12 
segments each covering 30 azimuthal degrees.  The outermost layer
has 1 extra segment in the top and bottom, totaling 14.  Each DT is
paired with either one or two RPCs, two on either side in the first
two layers and one on the inner most edge in the second two layers.
A high-$p_T$ track can cross up to 6 RPCs and 4 DTs, providing 44
measurements for track reconstruction.

The CSCs are trapezoidal chambers containing 6 gas gaps, each with
corresponding cathode strips running radially and anode wires
running azimuthally.  Charge from ionized gas is collected on 
strips and wires.  Signals on the wires are fast and can be used
for level-1 triggering, while cathodes provide a better measurement
of position, on the order of 200~\microns.

%\subsection{Track Reconstruction}
%\label{sec:trackRECO}
%
%\subsection{Muon Reconstruction}
%\label{sec:muonRECO}
%
%
%Muon tracks are reconstructed using the Kalman filter technique
%starting with RecHits at the inner most radius.
%In the barrel region, where the DTs provide the most sensitive 
%information about particle position, primitive track segments
%are reconstructed and fed to the Kalman filter.
%In the endcap, 3D hits are constructed from wire and strip 
%information which is then pass to the Kalman filter. In both
%cases, the RecHits from the RPCs are used.  Track states are 
%propagated through each layer taking into account the effects
%of energy loss in material, multiple scattering and inhomogeneities
%in the magentic field.  Once the outer most measurement is reached,
%the Kalman filter is then applied in reverse to define the final
%set of track parameters.  After the track is extrapolated to the 
%to the nominal interaction point, the beam-spot, a vertex-constrained
%fit to the track parameters is performed.
%Muons can also be reconstructed using the information from the 
%silicon tracker.  In this case, tracker hits which are compatible
%with the muon track are added to the Kalman filter.
%
%Muon reconstruction typically makes use of other subdetectors, such
%as the calorimeters.  Calorimeter cells which are compatible with 
%the extrapolated muon track should have energy deposits which are 
%consistent with the presence of a minimum ionizing particle.  
%Since events from pile-up can contribute the HO can be efficiently
%used to discriminate against pile-up activity.  
%
%Muon are also typically required to be well isolated.  An isolation
%cone is defined around the muon direction and a maximum threshold of 
%$E_T$ is allowed from calorimeter cells whithin this corn, or a maximum
%sum $p_T$ is allowed from tracks within this cone.  These thresholds are
%typically set as a fraction of the candidate muons $p_T$.
%
%The details of muon identification and isolation typically depend on
%the specific use case.  Thus, more detailed explainations will follow
%in conjunction with descriptions of analyses. 
%
%\subsection{Electron Reconstruction}
%\label{sec:electronRECO}
%
%Electron produce tracks in the silicon track and induce showers 
%in the ECal.  In beam test, typically, more than 90\% of the incident 
%energy of a single electron is contained in a $3\times3$ array of 
%crystals and more than 95\% is contained in a $5\times5$ array.  
%Within CMS, showing patterns are broadened by bremsstrahlung 
%induced by the presence material within the ECal and by the
%presence of a strong magnetic field.  In order to better identify 
%the energy deposits from electrons clustering algorithms are 
%used to combine energy measurements from individual crystals which
%take into account effects of bremsstrahlung and the magnetic field.  
%
%Seed crystals are identified which contain energy above some predefined
%threshold.  Nearby crystal are then combined to for a cluster.  In 
%an analogous procedure, cluster are combined into superclusters 
%starting from seed clusters.  Supercluster energies are then 
%corrected for systematic effects based on shower shape and the
%location of the supercluster.  The position of the reconstructed
%electron candidate is then measured from the energy-weighted mean
%position of each of the crystals.
%
%Superclusters are matched to pixel hits in the tracker which serve 
%as seed for electron tracks to be built.  Tracks are the built using
%a modified Kalman filter known as the Gaussian Sum Filter (GSF) which
%is a nonlinear filter which makes use of guassian mixtures to model
%errors.  The GSF algorithm allows for momentum at either end of the 
%tracker to be measured reliably and allows for the amount of brem to 
%be accurately estimated.  
%
%E-p combination...
%
%electron isolation ...
% 
%electron identification ...
%
%\subsection{Photon Reconstruction}
%\label{sec:photonRECO}
%
%Photon energy is clustered according to either the hybrid (EB) or the 
%island (EE) algorithm, similar to electrons.  For unconverted photons,
%most of the energy will be contained in a $3\times3$ array of crystals.
%In contrast, photons which get converted into $e^+e^-$ pairs can 
%cause the supercluster to spread over more crystal.  
%
%The R9 variable provides a powerful discriminator for converted photons.
%R9 is defined as the ratio of the energy in a $3\time3$ array, centered 
%on the highest energy crystal, to the total energy in a supercluster.
%Values of R9 close to 1 are indicitive of unconverted photon
%
%photon isolation...
%
%\subsection{Jet Reconstruction}
%\label{sec:jetRECO}

\subsection{Tracker}
\label{sec:Tracker}

The CMS tracker is an all silicon detector that consists of more
than 16,588 individual silicon modules.  These modules are of two
basic varieties, pixels which provide a 2-dimensional measurement
of particle positions and strips
which provide 1-dimensional measurements of particle positions 
within the plane of the module.  The tracker
is the closest sub-detector to the interaction point.  As such, 
it is exposed to the highest radiation flux and must be radiation 
hard to survive the extreme conditions of the LHC.  As such, the 
design of the tracker barrel has been broken into three distinct 
regions in order to optimize occupancy against
signal-to-noise (S/N): the pixel barrel (PXB), the tracker inner 
barrel (TIB), and the tracker outer barrel (TOB).  The latter two 
regions consist of silicon microstrip detectors. 

\subsubsection{Pixel Modules}

The pixel modules are exposed to the highest particle flux, 
roughly $10^7$~Hz at $r=10~cm$.  As a result, small pixels, 
$100\times150~\mu m^2$, are used giving an occupancy of about 
$10^{-4}$ per pixel per bunch crossing. Three layers make up the 
pixel barrel at radii $r=4.4,~7.3,$ and $10.2~cm$ consisting of 
768 pixel modules in total.  There are also two endcap disks on
either side of the pixel barrel made of 672 pixel modules arranged
in a turbine fashion.  The layout of the pixel modules is shown in 
Figure~\ref{fig:TrackerGeometry}. In total, there are 66 million 
pixels which provide precise hit measurements.

\subsubsection{Strip Modules}

The strip modules are arranged into four regions: inner barrel 
(TIB), outer barrel (TOB), inner disks (TID), and end caps (TEC). 

The TIB is divided into 4 layers which extend out to 
$|z|<65~cm$, consisting of 2724 strip modules.  The microstrip 
sensors on each module have a thickness of 320~\microns~and a 
pitch of 80-120~\microns. The two inner most layers of the TIB 
have stereo modules offset by an angle of 100 mrad, providing 2D
measurements.  The hit position resolution of these modules ranges
from 23-34~\microns~in $r-\phi$ and 230~\microns~in the z-direction.

The TOB is divided into 6 layers extending out to $|z|<65~cm$, 
consisting of 5208 strip modules.  Each microstrip sensor has a 
thickness of 500~\microns~and a pitch ranging from 
120-180~\microns.  Since the radii of the strip layers is large, 
strips can be thicker in order to have better S/N while 
still have low occupancy.  Similar to the TIB, the first two 
layers of the TOB have stereo modules offset by 100~mrad so that 
the single point resolution in $r-\phi$ is 35-52~\microns~ while 
it is 530~\microns~ in the z-direction.

The TID is divided into 3 disks, the first two of which are stereo, 
arranged at various distances between $120<|z|<280~cm$.  Modules
are arranged in wheels around the beam axis.  Each microstrip 
sensor has a  thickness of 320~\microns.  Similarly, the TEC has 9
disks, the first two and the fifth of which are stereo.  The 
thickness of each microstrip sensor is 500~\microns. 

\begin{figure}
\begin{center}
\hspace{-2.5cm}
\includegraphics[width=.8\linewidth]{ExperimentalSetupPlots/las.eps}
\caption{Quarter slice of the CMS tracker.  Single-sided silicon strip modules
are indicated as solid light (purple) lines, double-sided strip modules as 
open (blue) lines, and pixel modules as solid dark (blue) lines.}
\label{fig:TrackerGeometry}
\end{center}
\end{figure}

\subsubsection{Tracking Performance \& Alignment}
\label{sec:alignment}

The tracker provides high precision measurements of track 
parameters for all charged particles; this includes both the
momentum and direction of tracks.  These track parameters can be
used to better understand resonance properties, as will be shown in 
Chapters~\ref{sec:HiggsPhen} and~\ref{sec:HZZsearches}.  Thus, the 
tracker will be one of the most important tools in searching for 
new resonances, such as the Higgs boson, and understanding their 
role in nature.  

The tracker is also the only detector which can reconstruct 
vertices, either displaced or not.  Vertexing provides critical 
information to help mitigate the effects of pile-up as well as 
tagging b-jets.  Since pile-up will be a continuing challenge at 
the LHC, continued performance of the tracker will be critical.  
The use of the tracker in b-tagging will also play a central role 
in physics measurements since b-jets provide a distinct 
signature which is relevant to many models beyond the SM
as well of Higgs physics.  

In order to ensure high quality performance of track 
reconstruction algorithms, uncertainties of
module positions, which refers to both the location and
orientation which are depicted in 
Figure~\ref{fig:ModuleCoordinates}, should be reduced to within 
the precision of each module.  For the pixel modules, this 
precision is around 10~\microns~while for the strips, this 
precision can be as large as 30~\microns.  Because of changing 
environmental conditions of the detector, the tracker geometry can 
be time dependent. In order to efficiently determine module 
positions through run periods, offline track-based alignment 
algorithms must be employed.  

\begin{figure}
\begin{center}
\includegraphics[width=.49\linewidth]{ExperimentalSetupPlots/coordinates.eps}
\caption{Diagram of module position variables, u, v, w, and module orientation
variables, $\alpha$, $\beta$, $\gamma$.}
\label{fig:ModuleCoordinates}
\end{center}
\end{figure}

Track-based alignments are intended to determine the position of each module
in the tracker from a large collection of reconstructed tracks.  Each track
is built from a set of charge deposition sites, or hits, on a given module
which are used to produce a piece-wise helical trajectory using the
Combinatorial Track Finder (CTF) algorithm~\cite{Borrello:2010zz}.  
Alignment of each module position can be performed by minimizing
\begin{equation}
\chi^2(\vec{p},\vec{q}) = \sum_{j}^{tracks}\sum_{i}^{hits}\vec{r}_{ij}^{T}(\vec{p},\vec{q}_j)V^{-1}_{ij}\vec{r}_{ij}(\vec{p},\vec{q}_j),
\end{equation}
where $\vec{p}$ is the position correction, $\vec{q}_j$ is the set of 
track parameters for the j tracks, $\vec{r}_{ij}$ are the track 
residuals, and $\vec{V}_{ij}$ is the covariance matrix.  The residuals are
defined as $\vec{r}_{ij} = \vec{m}_{ij} - \vec{f}_{ij}(\vec{p},\vec{q}_j)$, 
where $\vec{m}_{ij}$ are the measured hit positions and $\vec{f}_{ij}$ are
the track trajectory impact point in the plane of the modules.  The 
$\chi^2$ function is then minimized with respect to the module position 
corrections, $\vec{p}$.

Since there are more than 16,588 modules with 6 parameters to be
determined, tracker alignment is an extremely difficult to solve
exactly.  As a result, approximations 
must be employed.   One such approximation is to minimize the
$\chi^2$ for each module individually, ignoring the correlation
between the change in parameters between different modules.  The 
correlation is then recovered 
by recalculating $f_{ij}$ and iterating the procedure many times.
Solving for each individual module's position corrections is then 
reduced to a six-dimensional  matrix equation, 
\begin{equation}
\chi^2(\vec{p})=\sum_i^{hits}\vec{r}^T_i\vec{V}_i^{-1}\vec{r}_i(\vec{p}).
\end{equation}
This local iterative algorithm, described in detail 
elsewhere~\cite{Karimaki:2006az,Brown:2008ccb}, was employed to produce 
the first geometry using minimum bias collision tracks.  

Validations of tracker geometries are critical to understanding 
that the output of alignment algorithms improves physics
measurements.  Several validations which can demonstrate
improvements in the tracker geometry are the primary vertex 
validation and the cosmic splitting validation.  Both of these 
validations provide a direction connection between the 
tracker geometry and measurements relevant for physics analyses.   

The cosmic splitting validation makes use of cosmic tracks recorded 
during inter-fills.  Cosmic tracks have the unique feature that the 
tracks can pass through silicon layers on both sides of the 
tracker.  As a result, a cosmic track is qualitatively similar to 
two collision tracks produced back to back.  This feature can be 
taken advantage of by dividing
each cosmic track into subsets of hits and reconstructing these hits 
into {\it split} tracks which are reconstructed independently.  The 
track parameters of the 
split tracks should, by construction, have the same track 
parameters.  Thus, by comparing the track parameters, 
resolution and biases can be gauged.  

%% describe the slew of track parameters that can be studied.
The resolution of individual track parameters can be quantified
and compared between different tracker geometries.  This is
represented by the distribution of the difference of a given 
track parameters between the two split tracks.  This difference
can also be compared in slices of other track parameters in 
order to quantify systematic misalignments.

To demonstrate this, the
difference of 5 track parameters: $\Delta d_{xy}$, $\Delta d_z$, 
$\Delta \eta$, $\Delta\phi$, and $\Delta p_T$ are shown in 
Figure~\ref{fig:trackSplitting2012A} 
using cosmic tracks recorded during 2012 Run A.  Three geometries 
are compared, the ideal geometry, the 
prompt geometry (before alignment) and the Re-RECO geometry (after 
alignment).  Improvements are found over the prompt geometry and
in some cases, the aligned geometry is found to be consistent 
with the ideal geometry tested on MC simulations.  

\begin{figure}
\begin{center}
\includegraphics[width=.32\linewidth]{ExperimentalSetupPlots/2012A/histDelta_dxy.eps}
\includegraphics[width=.32\linewidth]{ExperimentalSetupPlots/2012A/histDelta_dz.eps}
\includegraphics[width=.32\linewidth]{ExperimentalSetupPlots/2012A/histDelta_eta.eps}
\includegraphics[width=.32\linewidth]{ExperimentalSetupPlots/2012A/histDelta_phi.eps}
\includegraphics[width=.32\linewidth]{ExperimentalSetupPlots/2012A/histDelta_pt.eps}
\caption{Resolution of 5 track parameters from track splitting 
validation using three geometries, ideal (blue), prompt geometry 
(black), and the aligned geometry (red).  Cosmic track recording 
during the 2012 Run A period were used.}
\label{fig:trackSplitting2012A}
\end{center}
\end{figure}

From Figures~\ref{fig:trackSplitting2012A}, we can see 
that the average errors of the impact parameters are 
25\microns~(42\microns) for the transverse (longitudinal) 
directions with respect to the beam line.  The angular variables 
are found to have extremely good precision, on the level of 
the $3.2\times10^{-4}$~radians for the azimuthal angle, $\phi$, 
and ($4.6\times10^{-4}$) for pseudorapidity, $\eta$.  The 
transverse momentum, $p_T$, has a relative precision of 1\%.  

Since the $p_T$ distribution of cosmic tracks is dominated by low 
$p_T$ tracks, the $p_T$ resolution for high momentum tracks can be
better understood by plotting the width of the $\Delta p_T$ 
distribution in bins of $p_T$.  This is shown in the right plot of 
Figure~\ref{fig:trackSplittingProfiles}.  The relative resolution 
on $p_T$ varies from .1~GeV to .45~GeV for tracks with $p_T$ 
between 10 and 100 GeV.  Cosmic tracks provide a unique source
of very high $p_T$ muons.  Using the track splitting procedure, 
these muons can be used to better understand tracking in this
extreme phase space.  
In general, all track parameter errors can also be measured in 
bins of other variables, known as profile plots.  The left 
plot of Figure~\ref{fig:trackSplittingProfiles} shows $d_{xy}$ in 
bins of $\phi$ for the 2012 Run A cosmic data.  There is a 
significant improvement between the prompt and re-RECO geometries.

\begin{figure}
\begin{center}
\includegraphics[width=.49\linewidth]{ExperimentalSetupPlots/2012A/profiledxy_orgDelta_phi.eps}
\includegraphics[width=.49\linewidth]{ExperimentalSetupPlots/2012A/resolutionpt_orgDelta_pt.eps}
\caption{Profile plots of several reference geometries using cosmic tracks
recorded during the 2012 Run A period.  The left plot shows the difference 
in $d_{xy}$ between the two split tracks, $\Delta d_{xy}$ vs $\phi$.  The 
right plot shows the width of the $\Delta p_T$ distribution, $\sigma(p_T)$, 
vs $p_T$.}
\label{fig:trackSplittingProfiles}
\end{center}
\end{figure}

Profile plots are sensitive to structures like the ones shown in 
Figure~\ref{fig:trackSplittingProfiles} and can be used to gauge
the presence of systematic misalignments of the tracker. In some 
cases, these misalignments are $\chi^2$ invariant, also known
as weak modes.  Some examples include a systematic shift of modules
in the r-$\phi$ direction which is a function of $\phi$ itself.  
This  type of deformation would result in the structure that is 
seen in the left plot of Figure~\ref{fig:trackSplittingProfiles}
in the prompt geometry.  In this case, the deformation is not a
weak mode since the alignment procedure is sensitive to it and
corrects the module positions accordingly.  However, understanding
similar deformations is important for assessing uncertainties in
physics measurements.  

%% discuss primary vertex validations

The primary vertex validation uses the position of primary vertices as
an estimator of the true impact parameters of an individual track.  
Residuals can be constructed from the difference between the primary 
vertex and a track's fitted impact parameter as demonstrated in 
Figure~\ref{fig:PVcartoon}.    If tracks truly originate from 
the vertex, then on average the above assumption
will be true.  However, individual tracks which pass through poorly 
aligned regions of the tracker will give larger residuals, thus providing
a self consistent probe of the tracker geometry.  

\begin{figure}
\begin{center}
\includegraphics[width=.69\linewidth]{ExperimentalSetupPlots/PVcartoon.eps}
\caption{Diagram depicting the calculation of residuals used in 
the primary vertex validation.}
\label{fig:PVcartoon}
\end{center}
\end{figure}

Distributions of impact parameter residuals are sensitive to changes in the
pixel modules.
Figure~\ref{fig:PVresiduals}
shows a number of residual distributions of the longitudinal
impact parameter, $d_z$, in various bins of $\eta$ and
$\phi$.  Each bin represents tracks from a specific region of pixel 
module.  The mean and RMS of these distributions, which are measured
using double Gaussian fits, can provide useful
information about systematic misalignments of the pixel barrel.  In
particular, this validation is sensitive to the presence of separation of the
pixel half barrels, which tend to move when detector conditions 
change.

\begin{figure}
\begin{center}
\includegraphics[width=.99\linewidth]{ExperimentalSetupPlots/VertexDzResidualsEtaBin_20110318.eps}\\
\includegraphics[width=.99\linewidth]{ExperimentalSetupPlots/VertexDzResidualsPhiBin_20110318.eps}
\caption{Residual transverse impact parameter distributions in 
bins of $\eta$ (top) and $\phi$ (bottom).}
\label{fig:PVresiduals}
\end{center}
\end{figure}

To quantify the separation of the pixel half barrels, the mean and
width of the residual distributions are plotted as a function of
$\phi$.  If a separation between the two half barrels is present, it
will cause a discontinuity at zero.  Figure~\ref{fig:dzResidVsPhi}
shows an example plot of this using MC tracks with either the ideal
geometry or a geometry in which the two half barrels have been 
purposefully shifted.  The size of the discontinuity directly 
corresponds to the size of the physical separation.  

\begin{figure}
\begin{center}
\includegraphics[width=.69\linewidth]{ExperimentalSetupPlots/PVValidation_dzResidVsPhi_paper.eps}
\caption{Distribution of mean and width of transverse impact parameter residuals 
in bins of the probe tracks azimuthal angle, $\phi$, for an ideal geometry (black), 
ideal geometry plus 40~\microns~separation between the pixel half barrels (red),
and the 2011 candidate geometry (blue).}
\label{fig:dzResidVsPhi}
\end{center}
\end{figure}

The presence of a shift can have significant impact on vertex 
measurements,
which can affect either efficiency of associating tracks with the 
primary
vertex or efficiency of b-tagging. Thus, monitoring and correcting
these deformations in time is critical.  Figure~\ref{fig:DzVStime}
shows the 
measured separation of the pixel half barrels versus time before 
and after alignment parameters were determined.  This procedure
was critical for determining an effective alignment procedure by 
defining
run ranges to perform independent alignments of large structures in order to 
correct the time dependence seen.  The red points in Figure~\ref{fig:DzVStime}
show that most of the time dependence is reduced to below 5-10~\microns.

\begin{figure}
\begin{center}
\includegraphics[width=.49\linewidth]{ExperimentalSetupPlots/plotForAlignmentPaper.eps}
\caption{Measured separation between pixel half barrels versus time
before and after alignment.}
\label{fig:DzVStime}
\end{center}
\end{figure}

\section{Summary}

The necessary but challenging environment provided by the LHC has produced higher collision energies than have ever previously been attained.  This is critical for 
producing heavy resonances as well as increasing the phase space
for producing intermediate mass resonances such as the Higgs boson.  
The design of CMS has allowed for high quality data collecting even in the midst 
of the high rates and high pileup environments produced by the LHC.  Offline validation,
calibration, and alignment of the various sub-detectors is a critical 
aspect of the success of CMS.  

The continued monitoring and adjustment of the tracker geometry using offline
track-based alignment algorithms is critical for producing high precision
track measurements.  This will be critical to physics measurements,
especially those related to Higgs boson searches.  Since angular and mass distributions of the final state 
particles of resonances can be exploited for property measurements, to be
discussed in Chapters~\ref{sec:HiggsPhen} and~\ref{sec:HZZsearches}, it is 
important to have tools like those mentioned above to monitor tracker 
performance using either collision tracks or cosmic tracks.
