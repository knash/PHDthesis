%% FRONTMATTER
\begin{frontmatter}

% generate title
\maketitle

\begin{abstract}
Understanding the exact mechanism of electroweak symmetry breaking
through the discovery and characterization of the Higgs boson
is one of the primary goals of the Large Hadron Collider (LHC).
Although the Standard
Model (SM) of electroweak interactions has been extremely
successful in describing a number of phenomena, there are still
questions to be addressed pertaining to its naturalness and its
possible connection to beyond the SM physics.  
The foundation of a new set of tools which 
are used to search for new resonances and to
characterize the newly discovered 125~GeV resonance at the LHC
is developed in this thesis.
The use of these tools is demonstrated with two searches 
for a Higgs boson decaying to a pair of Z bosons with subsequent
decay to either $2\ell2q$ or $4\ell$ using data recorded with 
the Compact Muon Solenoid (CMS). 
Prospects for using these new analysis techniques
in other Higgs channels and at future colliders are addressed.
Finally, results are interpreted in the context of possible
extensions to the SM and their effect on our understanding of
the universe.

\vspace{1cm}

\noindent Primary Reader: Andrei Gritsan\\
Secondary Reader:

\end{abstract}

\begin{acknowledgment}

I would like to thank Andrei Gritsan for accepting me as a student.
I am lucky to have been apart of developing the great ideas that
have resulted from his research program and have learned an 
emmense amount physics and how to approach research problems 
in doing so.  I have been fortunate to take on a leading role in
my field and to represent my collaboration on more than one 
occation as an ambassador to the greater scientific community. 
This would not have been possible with his encouragement and
guidance.

I would also like to thank everyone involved with CMS and the
LHC.  It has been a remarkable experience to be 
apart of the callaboration and see such what can be done when
thousands of people put their minds to one big idea.  
Special thanks to all the conveners of the Higgs PAG,
HZZ subgroup, and tracker alignment group.  
I am eternally greatful for those who have supported me
in my continue academic career: Chiara, Joe, Andrey, and Yves. 

Thanks to the National Science Foundation for their generous 
grants, without which many things would not be possible...

\end{acknowledgment}

%\begin{dedication} 

%\end{dedication}

% generate table of contents
\tableofcontents

% generate list of tables
\listoftables

% generate list of figures
\listoffigures

\end{frontmatter}
