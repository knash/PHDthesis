\chapter{Introduction to the $\bs$ Search}
\label{sec:bsintroduction}

The focus of this analysis is a BSM predicted \cite{Tait:2000sh} excited b quark referred to as $\bs$.  
The decay mode considered in this analysis is $\bs \to $tW, as it is dominant in the $\bs$ mass region of interest.

$\bs$ production at the LHC takes place through the strong interaction.  The lagrangian describing this interaction is as follows: 

\begin{eqnarray}
{\cal L}_{production} = \frac{g_s}{2\Lambda}G_{\mu\nu} \overline{b} \sigma^{\mu\nu} (\kappa^{b}_{L}P_L + \kappa^{b}_{R}P_R)\bs + H.c.
\label{eqn:Lag1}
\end{eqnarray}

$\bs \to $tW decay takes place through the weak interaction and is described through the following lagrangian. 

\begin{eqnarray}
{\cal L}_{decay} = \frac{g_2}{\sqrt{2}}W^{+}_{\mu} \overline{t} \gamma^{\mu}(g_{L}P_L + g_{R}P_R)\bs + H.c.
\label{eqn:Lag2}
\end{eqnarray}

We consider three hypotheses for the right- and left-handed couplings.

\begin{eqnarray}
\text{left-handed: }\kappa^{b}_{L}=g_{L}=1 \text{ and } \kappa^{b}_{R}=g_{R}=0 \\
\text{right-handed: }\kappa^{b}_{L}=g_{L}=0 \text{ and } \kappa^{b}_{R}=g_{R}=1 \\
\text{vectorlike : }\kappa^{b}_{L}=g_{L}=1 \text{ and } \kappa^{b}_{R}=g_{R}=1 
\label{eqn:couplings}
\end{eqnarray}

Searches for the $\bs$ quark in the tW decay mode have been performed at the ATLAS detector at the LHC \cite{Aad:2013rna}.  
Using 19.7$\fbinv$ of integrated luminosity at 8$\TeV$, we exclude a left-handed $\bs$ quark couplings between 0.99$\TeV$ and 1.40$\TeV$  

Similar to the $\wpr$ search described in the previous chapters, the $\bs$ quark region of interest is high mass.  
Therefore, similar boosted techniques are used to identify the top quark decay products and reduce the QCD background.  
Additionally, similar methods are used to estimate the background due to the success of these methods in the $\wpr$ search.  

%As the $\bs$ region of interest covers masses in excess of $\sim1\TeV$, the top quark and W boson from the $\bs$ decay will be highly boosted.  
%Due to the Lorentz boost of the top quark and W boson, the decay products 
%merge, resulting in a dijet topology.  
%This analysis uses special techniques to identify the substructure of these merged jets in order to reduce the QCD multijet background contribution to 
%the full background estimate.
