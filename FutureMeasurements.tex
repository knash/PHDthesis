\chapter{Future Measurements}
\label{sec:FutureMeasurements}
\chaptermark{Future Measurements}

The discovery of a Higgs-like resonance provides a new window
for beyond the SM physics searches.  Results presented in 
Section~\ref{sec:HZZ4l} are consistent with this resonance 
being the SM Higgs boson.  As a result, the resonance will be
referred to as a Higgs boson throughout this chapter. 
The development of a campaign to perform high precision
measurements of Higgs properties is now a top priority.
If this resonance ends up 
being exactly the Higgs boson described by the GWS model, 
this campaign will likely extend into the next generation of 
particle accelerators.  

This chapter will discuss the logical progression of the MELA
techniques which have been developed and applied in previous 
chapters.  The use of multidimensional fits for measuring the 
HZZ amplitude parameters (see
equation~\ref{eq:HZZmodelParams}) will be expounded.  
Projections to high luminosity scenarios of the $H\to ZZ^*\to4\ell$
process at the LHC will be studied using both multidimensional
and the MELA techniques will be presented.  The same tools will
be adapted to a future $e^+e^-$ collider using $Z^*\to ZH\to 2\ell2b$
events.   Finally, speculation will be made on adapting the MELA
techniques to other processes at the LHC.

\section{Multidimensional Fits}

The use of multidimensional fits 
and the MELA technique for measuring model parameters are 
complementary methods.  While multidimensional fits provide 
the flexibility to measure all model parameters, its does so at
the cost of simplicity. In contrast, it is not possible to use
the MELA 
technique for simultaneously measuring all of the HZZ model 
parameters, but allows for 
kinematics to be easily described, including all detector effects,
in terms of one or two observables. 

Consider an experiment in which no background events are expected
and an ideal detector is used.  In this case, the analytic formulas
describing differential cross
sections, which have been used as inputs to the MELA discriminants
can be used to directly build the likelihood for fitting 
model parameters,
\begin{equation}
\mathscr{L}=\Pi^N_i \mathscr{P}_{sig}(\vec{x}_i ; \vec{\xi}),
\end{equation}
where $\mathscr{P}$ represents the differential cross section, 
$\vec{x}_i$ are the observables for event i, and $\vec{\xi}$ 
are the model parameters for which the likelihood will be 
maximized with respect to.
This method can be computationally efficient, if the analytical
integral of the likelihood can be provided for all points in the
parameter space.  For example, the 
$H\to ZZ^*\to4\ell$ analysis at the LHC makes use of 8 observables
which distinguish different scalar models and background.  If one
were to attempt to measure each of the four model parameters
simultaneously,
either the 8D integral should be known a piori at each
point in the 4D parameter space or numerical integration 
over the 8 observables must be performed at each point in the 
4D parameter space.  The latter is nearly impossible.  

For the ideal distribuions, it is possible to calculate the
integral of the likelihood analytically as a function of the
4 model parameters and has been done for the $H\to ZZ^*\to 4\ell$
process.  Using the the likelihood presented above, 
toys studies can be performed to compare the sensitivity for 
measuring $f_{g4}$ using either multidimensional fits or the
MELA technique.  Figure~\ref{fig:fa3MultiDimComp} shows the
results of three 
types of fits: multidimensional fits in which $f_{g4}$ is
floated, multidimensional fits in which $f_{g4}$ and $\phi_{g4}$
are floated, and 1D fits using the MELA technique floating $f_{g4}$.
In all three cases, toys generated correspond to
a scalar resonance with $f_{g4}=0.18$.  The results of the
1D fit and the 5D, 2 parameter fit are both compatible.  However,
it is found that the 1 parameter multidimensional fit provides a
4\% improvement.  For generated
values of $f_{g4}=0.06$ and 0.02 this improvement is found to be
13\% and 30\%, respectively.  The interpretation of this is that
the relative importance of interference terms in Equation~\ref{eq:angularDist},
which is not accounted for in the MELA technique, become large for
small values of $f_{g4}$.

\begin{figure}
\begin{center}
\includegraphics[width=.49\linewidth]{FutureMeasurementsPlots/lhc_5Dvs1Dfit.eps}
\caption{Distribution of best-fit $f_{g4}$ values using either 
the MELA method (solid black) or multidimensional fits (dashed
lines) with either 1 (pink) or 2 (blue) parameters floating.}
\label{fig:fa3MultiDimComp}
\end{center}
\end{figure}

\section{LHC Projections}

As a point of reference, it is instructive to estimate the
expected sensitivity that CMS can reach using the current method
for measuring $f_{g4} (=f_{a3})$ in the $H\to ZZ^*4\ell$ analysis.  Detector
simulations are modeled by 
including finite momentum and angular resolution of lepton
four vectors and applying analysis selections both of which are
meant to roughly mimic the CMS public analysis~\cite{CMS:xwa}.
Leptons are
required to have $|\eta|<2.4$, $p_T>5$~GeV, and $m_2>12$~GeV.
The resolution effects result in a $m_{4\ell}$ width of
approximately 2~GeV, similar to that of the $2e2\mu$ channel.  

Two luminosity scenarios are tested, $300~fb^{-1}$ and 
$3000~fb^{-1}$.  Shapes are modeled using ideal MC simulations
with the approximate detector effects described above.
Background shapes are taken purely from \verb+POWHEG+ simulation
of $q\bar{q}\to ZZ^*\to 2e2\mu$ events.  Signal shapes are
taken purely from \verb+JHUGen+ simulation of 
$gg\to H\to ZZ^*\to 2e2\mu$ events.  
The number of events expected for signal and background are 
used are listed in Table~\ref{table:FutureMeasEventYields}.
Events yields for $3000~fb^{-1}$ are simply scaled up by 
a factor of 10. 

\begin{table}
\begin{center}
\begin{tabular}{ccccc}
\hline 
\hline
energy & $\int \mathscr{L}dt~[fb^{-1}]$ & 
$\sigma\times\mathscr{B}~[fb^{-1}]$ & $N_{prod}$ & $N_{reco}$ \\ 
\hline
\hline
\multicolumn{5}{c}{$pp\to H \to ZZ^* \to 4\ell$}  \\
\hline
14~TeV & 300  &  6.23 & 18694 & 5608 \\
\hline
\multicolumn{5}{c}{$pp\to ZZ^* \to 4\ell$}  \\
\hline
14~TeV & 300 &  -- & -- & 2243 \\
\hline
\multicolumn{5}{c}{$e^+e^-\to Z^* \to ZH \to 2\ell 2b$} \\
\hline
250 GeV & 250 & 9.35 & 2337 & 1870 \\
%350 GeV & 350 & 5.03 & 1760 & 1408 \\
%500 GeV & 500 & 2.22 & 1110 & 888 \\
%1 TeV& 1000& .51  & 505  & 404 \\
\hline
\multicolumn{5}{c}{$e^+e^-\to ZZ \to 2\ell 2b$} \\
\hline
250 GeV & 250 & -- & -- & 187 \\
%350 GeV & 350 & --- & --- & --- \\
%500 GeV & 500 & --- & --- & --- \\
%1 TeV   & 1000& --- & --- & --- \\
\hline
\hline
\end{tabular}
\end{center}
\label{table:FutureMeasEventYields}
\caption{List of cross sections and event yields for Higgs
production and decay processes.}
\end{table}

Some approximations are used for modeling the detector
effects and background distributions.  In both cases,
uncorrelated distributions are used.  Although, these
approximations are found to cause 
small biases in toys studies, they are found to provide 
a description which is accurate enough to estimate the
sensitivity of such measurements using toys generated
directly from probability distribution functions.

The sensitivity to $f_{g4}$ using multidimensional fits
is found to be similar as those estimated from 1D fits.  
It is estimated that CMS will have sufficient sensitivity for
at least a $3\sigma$ discovery of CP-violating interactions in 
the $H\to ZZ^*$ channel for values of $f_{g4}\geq0.18~(0.06)$
 with $300~(3000)~fb^{-1}$, respectively.
Using multidimensional fits, it is estimated that CMS can also
achieve sufficient sensitivity 
for a $3\sigma$ or better discovery of anomalous CP-even 
couplings for values of $f_{g2}\geq0.14~(0.088)$ with 
$300~(3000)~fb^{-1}$, respectively.

\section{Future Colliders}

Similar measurements can be made with other processes such as
$e^+e^-\to Z^*\to ZH \to 2\ell2b$.  The diagrams in 
Figure~\ref{fig:HZZprocesses} demonstrate that 
this processes is equivalent to the $pp\to H\to ZZ\to4\ell$
process, except it probes a different region of phase space.
Thus, the differential cross sections
presented in Section~\ref{sec:HiggsPhen} are all still applicable.
The probability distribution is given by equation~\ref{eq:masterEq}
where
\begin{equation}
\begin{split}
\frac{d\Gamma_{J=0}}{\Gamma d\vec{\Omega}} = 4|A_{00}^2|\sin^2\theta_1\sin^2\theta_2 \\
+|A_{++}|^2(1-2R_1\cos\theta_1+\cos^2\theta_1)(1+2A_{f2}\cos\theta_2+\cos^2\theta_2)\\
+|A_{--}|^2(1+2R_1\cos\theta_1+\cos^2\theta_1)(1-2A_{f2}\cos\theta_2+\cos^2\theta_2) \\
-4|A_{00}||A_{++}|(R_1-\cos\theta_1)\sin\theta_1(A_{f2}+\cos\theta_2)\sin\theta_2\cos(\Phi+\phi_{+0}) \\
-4|A_{00}||A_{--}|(R_1+\cos\theta_1)\sin\theta_1(A_{f2}-\cos\theta_2)\sin\theta_2\cos(\Phi-\phi_{-0}) \\
+2|A_{++}||A_{--}|\sin^2\theta_1\sin^2\theta_2\cos(2\Phi-\phi_{-0}-\phi_{+0}),
\end{split}
\end{equation}
$R_1$ and $A_{f2}$ are the amplitude for Z coupling to fermions.  
Note, the in the translation from coupling the
coupling defition in Equation~\ref{eq:scalarAmp} and those 
used in Equation~\ref{eq:masterEq},  s should be negated.
%provided that \textcolor{red}{blah be changed to blah}. 
For the present case process, the Z boson and Higgs boson are
both on-shell 
and their mass can be approximated as constant. 
Thus, three non-trivial angular distributions describe 
the kinematics of this process. Figure~\ref{fig:ILCprojections}
shows these distributions for several scalar models, SM Higgs,
a pseudoscalar, and two mixed parity scalar models
with phases $\phi_3=0$, $\pi/2$.  

\begin{figure}
\begin{center}
\includegraphics[width=.32\linewidth]{FutureMeasurementsPlots/angles-ZZHbb_snowmass_3.eps}
\includegraphics[width=.32\linewidth]{FutureMeasurementsPlots/angles-ZZHbb_snowmass_cms_4.eps}
\includegraphics[width=.32\linewidth]{FutureMeasurementsPlots/angles-HZZ4l_snowmass.eps}
\caption{Diagrams showing the different processes produced
via the HZZ amplitude.  The $e^+e^-\to Z^*\to ZH\to 2\ell2b$ process
in the $Z^*$ and H rest frame are shown in the left and middle
plot, respectively.  The $pp\to H\to ZZ^*\to 4\ell$ process
is shown in the H rest frame is shown in the right plot.}
\label{fig:HZZprocesses}
\end{center}
\end{figure}

\begin{figure}
\begin{center}
\includegraphics[width=.32\linewidth]{FutureMeasurementsPlots/ILCshapes_costheta1.eps}
\includegraphics[width=.32\linewidth]{FutureMeasurementsPlots/ILCshapes_costheta2.eps}
\includegraphics[width=.32\linewidth]{FutureMeasurementsPlots/ILCshapes_phi.eps}
\caption{Angular distributions, $\cos\theta_1$ (left), 
$\cos\theta_2$ (middle), and $\Phi$ (right), of four different 
scalar models of the process $e^+e^-\to Z^*\to ZH$.  Markers
show angular distributions from JHUGen simulations while
lines show projections of the angular distributions presented
in Section~\ref{sec:HiggsPhen}. Red line/circles represent a 
SM Higgs, blue lines/diamonds represent a pseudoscalar, green
lines/squares and purple lines/solid circles represent a 
mixed parity scalar ($f_{g4}$=0.1) with various phases.}
\label{fig:ILCprojections}
\end{center}
\end{figure}

Note, the equivalent $f_{g4}$ parameter, which will be referred
to as $f_{g4}$, for this processes will 
have slightly different meaning.  For example, 
Table~\ref{table:fa3Conversion} summarizes
how the value for $f_{g4}$ of the $H\to ZZ^*$ process can be 
translated.  The numbers in this table reflect the fact that 
the ratio $\sigma_1/\sigma_3$, as define in 
Section~\ref{sec:HiggsPhen}, can vary by orders of magnitude
between different processes.  Larger $f_{g4}$ correspond to having
effectively more events which look like a pseudoscalar.  As a 
result, the sensitivity to CP-violating interactions, is
expected to be much larger for other processes.  

\begin{table}
\begin{center}
\begin{tabular}{l|c}
\large $g1/g4$                   & 0.85  \vspace{.1cm} \\
\hline\hline                          
\large $f_{g4}^{(H\to ZZ^*)}$          & 0.10  \vspace{.1cm} \\
\large $f_{g4}^{(q\bar{q}\to ZH)}$      & 0.81  \vspace{.1cm} \\ 
\large $f_{g4}^{(q\bar{q}\to Hq\bar{q})}$ & 0.93  \vspace{.1cm} \\
\large $f_{g4}^{(e^+e^-\to ZH)}(\sqrt{s}=250)$ & 0.85   \vspace{.1cm} \\
\large $f_{g4}^{(e^+e^-\to ZH)}(\sqrt{s}=500)$ & 0.99  \vspace{.1cm} \\
\hline
\hline
\end{tabular}
\end{center}
\label{table:fa3Conversion}
\caption{List of $f_{g4}$ values for various processes.}
\end{table}

Similar to the $H\to ZZ^*$ analysis, a kinematic discriminant 
built according to equation~\ref{eq:KD} can be used to measure 
$f_{g4}$ according to equation~\ref{eq:fa3}.  
Toys studies have been done to justify that there are no
biases introduced by the approximations in equation~\ref{eq:fa3}. 

Projections for a future $e^+e^-$
colliders are estimated assuming a collision energy  
of 250~GeV and an integrated luminosity of 250~$fb^{-1}$.
Signal events are simulated with \verb+JHUGen+.  Background events
are modeled using $e^+e^-\to ZZ$ events simulated with
\verb+MADGRAPH+.  
The cross sections and event yields for the signal and
background processes are detailed in 
Table~\ref{table:FutureMeasEventYields}
which are based on previous studies in references~\cite{???}.

All events are required to have two lepton whose 
transverse momentum is greater than 5~GeV, $|\eta|<2.4$, and 
Higgs boson mass between $115<m_H<140~GeV$.  Although
the background process is likely not fully representative
of the expected backgrounds that will exist in $e^+e^-$
collisions, the exact modeling of background events is not
critical for the purposes of this study.
The distribution of signal and background events and the effect
of acceptance cuts is shown for each of the three angles in 
Figure~\ref{fig:ILCanglesWithAccep}. 

\begin{figure}
\begin{center}
\includegraphics[width=.32\linewidth]{FutureMeasurementsPlots/h1_ee_250GeV_acc.eps}
\includegraphics[width=.32\linewidth]{FutureMeasurementsPlots/h2_ee_250GeV_acc.eps}
\includegraphics[width=.32\linewidth]{FutureMeasurementsPlots/phi_ee_250GeV_acc.eps}
\end{center}
\caption{Expected distribution of three helicity angles for
a SM Higgs boson (red) and the SM background (black) before
(solid lines) and after (dashed lines) acceptance cuts.}
\label{fig:ILCanglesWithAccep}
\end{figure}

Similar to before, toys are generated
and fit using Equation~\ref{eq:fa3}.  The distribution of
the best-fit $f_{g4}$ for a signal model corresponding to $f_{g4}=0.1$
is shown in the left plot of Figure~\ref{fig:ILCsensitivity}.  
The expected sensitivity is found to be 
$\sigma_{f_{g4}}=0.04$.  Converting this to the $f_{g4}$ parameter
currently being measured at the LHC, $f_{g4}(H\to ZZ^*)$,
the distribution of the best-fit $f_{g4}$ value is shown in 
the right plot of 
Figure~\ref{fig:ILCsensitivity} and the error on this parameter
is found to be $\sigma_{f_{g4}}=0.0008$.  This results can be
compared to the LHC scenario where the error for the high 
luminosity scenario was $\sigma_{f_{g4}}\sim0.03$.  

%\begin{figure}
%\begin{center}
%\includegraphics[width=.32\linewidth]{}
%\caption{Distribution of the best-fit value of $f_{g4}$ (left) for
%the process $e^+e^-\to Z^*\to ZH$ at $\sqrt{s}=250$~GeV and $f_{g4}$
%(right).  Toys were generated using a value of $f_{g4}=0.1$.}
%\label{fig:ILCsensitivity}
%\end{center}
%\end{figure}

\begin{figure}
\begin{center}
\includegraphics[width=.32\linewidth]{FutureMeasurementsPlots/ILCembedded_fa3p1_2000Evts.eps}
\includegraphics[width=.32\linewidth]{FutureMeasurementsPlots/ILCembedded_fa3p1_converted_2000Evts.eps}
\caption{Distribution of the best-fit value of $f_{g4}$ (left) for
the process $e^+e^-\to Z^*\to ZH$ at $\sqrt{s}=250$~GeV and $f_{g4}$
(right).  Toys were generated using a value of $f_{g4}=0.1$.}
\label{fig:ILCsensitivity}
\end{center}
\end{figure}

\section{Other Channels}

The sensitivity to CP-violating interactions in the 
HZZ amplitude is markedly better using $e^+e^-$ collisions. 
This is due to the fact that the $\sigma_3/\sigma_1$ in 
equation~\ref{eq:HZZmodelParams} is much large when Z bosons
are produced far off shell.
However, it should be noted that this simple exercise does not
completely diminish the potential for similar measurements at 
the LHC.  Other processes at the LHC shown in
Table~\ref{table:fa3Conversion},
e.g. $q\bar{q}\to H + q\bar{q}$ and $q\bar{q}\to Z^* \to ZH$,
also benefit from enhanced $\sigma_3$
due to the isolated phase space that they probe.  As these channels 
continue to gain sensitivity to signal events, they will play an
increasingly important role in constraining CP-violation.
Detailed studies are still to be done, but these channels may
ultimately dominate the sensitivity to $f_{g4}$ at the LHC.  

\section{Summary}

There are several complications involved with applying
multidimensional fit to the $H\to ZZ^*$ or other processes
analysis exit:
modeling an multidimensional transfer function appropriate
to event reconstruction and analysis selections; describing all 
backgrounds accurately; and building likelihoods which can be
efficiently minimized. However, multidimensional fits provide
a flexible approach which
could ultimately measure each of the model parameters which 
describe the HZZ amplitude.  

A number of the challenges related to multidimensional fits
can mitigated using the MELA
technique, discussed in chapter~\ref{sec:HiggsPhen} and applied
in Section~\ref{sec:HZZ4l}.  These techniques
help largely because the problem is reduced from using
many observables to using at most a couple of observables.  As 
with multidimensional fits,
these techniques are applicable to more processes than just 
$H\to ZZ^*$.  

Current measurements being done at CMS to constrain 
CP-violating interactions are only making use of $H\to ZZ^*$
events.  It is
likely that other channels at the LHC will one day probe much
larger regions of the parameters space and eventually, it is 
possible to use an $e^+e^-$ collider to perform high precision
measurements of Higgs properties.  Ultimately, 
these tools may become a staple of Higgs property measurements
for many year.




