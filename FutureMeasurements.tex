

\chapter{Future Measurements}
\label{sec:FutureMeasurements}
\chaptermark{Future Measurements}

The discovery of a Higgs-like resonance provides a new window
for beyond the SM physics searches.  As shown in
section~\ref{sec:HZZ4l}, all property measurements done thus
far suggest that this resonance is consistent with a SM Higgs
boson.   In addition to this, cross section measurements and 
limits set in other diboson channels suggest that this particle
is very likely to play role in electroweak symmetry breaking and 
as a result, will be referred to as a Higgs boson throughout this
chapter. 
What remains is a developing compaign to perform high precision
measurements of Higgs properties.  If this resonance end up 
being exactly the Higgs boson described by the GWS model, 
this campaign will likely extend into the next generation of 
particle accelerators.  

This chapter will discusss the logical progression of the MELA
techniques which have been developed and applied in previous 
chapters.  The use of multidimensional fits for measuring the 
HZZ amplitude parameters presented in 
equation~\ref{eq:HZZmodelParams} will be expounded.  
Projections to high luminosity scenarios of the $H\to ZZ^*\to4\ell$
process at the LHC will be studied using either multidimensional
or the MELA techniques will be presented.  The same tools will
be adapted to a future $e^+e^-$ collider using $Z*\to ZH\to 2\ell2b$
events.   Finally, speculation on adapting the MELA techniques to 
other processes at the LHC will will be made. 

\section{Multidimensional Fits}

The use of multidimensional fits for measuring model parameters
and the MELA technique for measuring model parameters are 
comlementary methods.  While multidimensional fits provide 
the flexibility to measure all model parameters, its does so at
the cost of simplicity. In constrast, it is not possible to use
the MELA 
technique for simultaneously measuring all of the HZZ model 
parameters\footnote{Recent ideas have suggested that measuring
two parameters at a time could be possible}, but allows for 
kinematics to be easily described, including all detector effects,
in terms of one or two observables. 

Consider an experiment in which no background events are expected
and an ideal detector is used.  In this case, the analytic formulas
describing differential cross
sections, which have been used as inputs to the MELA discriminants
can be used to directly build the likelihood for fitting 
model parameters,
\begin{equation}
\mathscr{L}=\Pi^N_i \mathscr{P}(\vec{x}_i ; \vec{\xi}),
\end{equation}
where $\mathscr{P}$ represents the differential cross section, 
$\vec{x}_i$ are the observables for event i, and $\vec{\xi}$ 
are the model parameters for which the likelihood will be 
maximized with respect to.
This method can be computationally efficient, if the analytical
integral of the likelihood can be provided for all points in the
parameter space.  

As an example, consider the 
$H\to ZZ^*\to4\ell$ analysis at the LHC.  There are 8 observables
which distinguish different scalar models.  If one were to 
attempt to measure each of the four model parameters simultaneously,
either the 8D integral should be provided known a piori at each
point in the 4D parameter space, else, numerical integration 
over the 8 observables must be performed at each point in the 
4D parameter space.  This measurements is nearly impossible.  

It is possible, at least for the ideal
distributions, to calculate the integral of the likelihood
analytically as a function of the
4 model parameters.  For the $H\to ZZ^*\to 4\ell$ process, this
has been done.  Using the the likelihood presented above, 
toys studies can be performed to compare the sensitivity for 
measuring $f_{a3}$ using either multidimensional fits or the
MELA technique.  Figure~\ref{fig:} shows results of three 
types of fits: multidimensional fits in which $f_{3}$ is
floated, multidimensional fits in which $f_{3}$ and $\phi_{3}$
are floated, and 1D fits using the MELA technique floating $f_{3}$.
In all three cases, toys were generated which correspond to
a scalar resonance with $f_{3}=0.18$.  The sensitivity of the
1 parameter fits are all compatible.  However, it is found that
the two parameter fit provides a 4\% improvement.  For generated
values of $f_{a3}=0.06$ and 0.02 this improvement is found to be
13\% and 30\%.

\begin{figure}
\begin{center}
%\includegraphics[width=.32\linewidth]{FutureMeasurementsPlots/}
\caption{}
\label{fig:fa3MultiDimComp}
\end{center}
\end{figure}

\section{LHC Projections}

As a point of reference, it is instructive to estimate the
expected sensitivity that CMS can reach using the current method
for measuring $f_3$ in the $H\to ZZ^*4\ell$ analysis.  Detector
simulations are modelled by 
including finite momentum and angular resolution of lepton
four vectors and applying analysis selections both of which are
meant to roughly mimic the CMS public analysis.  Leptons are
required to have $|\eta|<2.4$, $p_T>5$~GeV, and $m_2>12$~GeV.
The resolution effects result in a $m_{4\ell}$ width of 2~GeV, 
similar to that of the $2e2\mu$ channel.  

Two luminosity scenarios are tested, $300~fb^{-1}$ and 
$3000~fb^{-1}$.  Shapes are modeled using ideal MC simulations
with approximate detector effects described above applied.
Background shapes are taken purely from \verb+POWHEG+ simulation
of $q\bar{q}\to ZZ^*\to 2e2\mu$ events.  Signal shapes are
taken purely from \verb+JHUGen+ simulation of 
$gg\to H\to ZZ^*\to 2e2\mu$ events.  
The number of events expected for signal and background are 
used are listed in table~\ref{table:FutureMeasEventYields}.
Events yields for $3000~fb^{-1}$ are simple scaled up by 
a factor of 10. 
It is found that CMS will have sufficient sensitivity for
at least a $3\sigma$ discovery of CP-violating interactions in 
the $H\to ZZ^*$ channel for values of $f_{3}\geq0.06~(0.03)$
 with $300~(3000)~fb^{-1}$, respectively.

\begin{table}
\begin{center}
\begin{tabular}{ccccc}
\hline 
\hline
energy & $\int \mathscr{L}dt~[fb^{-1}]$ & 
$\sigma\times\mathscr{B}~[fb^{-1}]$ & $N_{prod}$ & $N_{reco}$ \\ 
\hline
\hline
\multicolumn{5}{c}{$pp\to H \to ZZ^* \to 4\ell$}  \\
\hline
14~TeV & 300  &  6.23 & 18694 & 5608 \\
\hline
\multicolumn{5}{c}{$pp\to ZZ^* \to 4\ell$}  \\
\hline
14~TeV & 300 &  --- & --- & --- \\
\hline
\multicolumn{5}{c}{$e^+e^-\to Z^* \to ZH \to 2\ell 2b$} \\
\hline
250 GeV & 250 & 9.35 & 2337 & 1870 \\
350 GeV & 350 & 5.03 & 1760 & 1408 \\
500 GeV & 500 & 2.22 & 1110 & 888 \\
1 TeV& 1000& .51  & 505  & 404 \\
\hline
\multicolumn{5}{c}{$e^+e^-\to ZZ \to 2\ell 2b$} \\
\hline
250 GeV & 250 & --- & --- & --- \\
350 GeV & 350 & --- & --- & --- \\
500 GeV & 500 & --- & --- & --- \\
1 TeV   & 1000& --- & --- & --- \\
\hline
\hline
\end{tabular}
\end{center}
\label{table:FutureMeasEventYields}
\caption{List of cross sections and event yields for Higgs
production and decay processes.}
\end{table}

A similar analysis can be performed  to probe anomalous CP-even 
couplings using multidimensional fits of $f_{2}$.  Some 
approximations are used for modeling the detector
effects and background distributions.  Uncorrelated distributions
are used to model both the detector effects and background 
distributions.  Although, these approximations are found to cause 
small biases in toys studies, they are found to provide 
a description which is accurate even to estimate the
sensitivity of such measurements.   More work will be needed
for more accurate descriptions.  The sensitivity to $f_{3}$ 
is found to be similar as those estimated from 1D fits.  
It is estimated that CMS can acheive sufficient sensitivity 
for a $3\sigma$ or better discovery of anomalous CP-even 
couplings for values of $f_{2}\geq0.088~(0.14)$ with 
$300~(3000)~fb^{-1}$, respectively.

\section{Future Colliders}

Similar measurments can be made with other processes such as
$e^+e^-\to Z^*\to ZH \to 2\ell2b$.  The diagrams in 
figure~\ref{fig:HZZprocesses} demonstrate that 
these processes are all equivalent, except they probe different 
regions of phase space.  Thus, the differential cross sections
presented in section~\ref{sec:HiggsPhen} are all still applicable
provided that \textcolor{red}{blah be changed to blah}. 
For the present case process, the Z and H are both on-shell 
and their mass can be approximated as constant. 
Thus, three non-trivial angular distributions describe 
the kinematics of this process. Figure~\ref{fig:ILCprojections}
shows these distributions for several scalar models, SM Higgs, a 
pseudoscalar, and two mixed parity scalar models with 
phases $\phi_3=0,~\pi/2$.  

\begin{figure}
\begin{center}
\includegraphics[width=.32\linewidth]{ConclusionPlots/angles-ZZHbb_snowmass_3.eps}
\includegraphics[width=.32\linewidth]{ConclusionPlots/angles-ZZHbb_snowmass_cms_4.eps}
\includegraphics[width=.32\linewidth]{ConclusionPlots/angles-HZZ4l_snowmass.eps}
\caption{Diagrams showing the different processes produced
via the HZZ ... }
\label{fig:HZZprocesses}
\end{center}
\end{figure}

\begin{figure}
\begin{center}
\includegraphics[width=.32\linewidth]{ConclusionPlots/ILCshapes_costheta1.eps}
\includegraphics[width=.32\linewidth]{ConclusionPlots/ILCshapes_costheta2.eps}
\includegraphics[width=.32\linewidth]{ConclusionPlots/ILCshapes_phi.eps}
\caption{Angular distributions, $\cos\theta_1$ (left), 
$\cos\theta_2$ (middle), and $\Phi$ (right), of four different 
scalar models of the process $e^+e^-\to Z^*\to ZH$.  Markers
show angular distributions from JHUGen simulations while
lines show projections of the angular distributions presented
in section~\ref{sec:HiggsPhen}. Red line/circles represent a 
SM Higgs, blue lines/diamonds represent a pseudoscalar, green
lines/squares and purple lines/solid circles represent a 
mixed parity scalar ($f_3$=0.1) with various phases.}
\label{fig:ILCprojections}
\end{center}
\end{figure}

Note that the equivalent $f_3$ parameter for these processes will 
have slightly different meaning.  For example, 
table~\ref{table:fa3Conversion} summarizes
how the value for $f_3$ of the $H\to ZZ^*$ process can be 
translated.  The numbers in this table reflect the fact that 
the ratio $\sigma_1/\sigma_3$, as define in 
section~\ref{sec:HiggsPhen}, can vary by orders of magnitude
between different processes.  As a result, the
sensitivity to CP-violating interactions, in terms 
of $f_3(H\to ZZ^*)$, is expected to be much larger for other 
processes.  


\begin{table}
  \vspace{1.0cm}
\begin{center}
\begin{tabular}{l|c}

\large $g1/g4$                   & --  \vspace{.1cm} \\
\hline\hline                          
\large $f_3^{(H\to ZZ^*)}$          & --  \vspace{.1cm} \\
\large $f_3^{(q\bar{q}\to ZH)}$      & --  \vspace{.1cm} \\ 
\large $f_3^{(q\bar{q}\to Hq\bar{q})}$ & --  \vspace{.1cm} \\
\large $f_3^{(e^+e^-\to ZH)}$        & --  \vspace{.1cm} \\
\large $f_3^{(e^+e^-\to ZH)}$        & .1  \vspace{.1cm} \\
\hline
\hline

\end{tabular}
\end{center}
\label{table:fa3Conversion}
\caption{List of $f_3$ values for various processes.}
\end{table}

Similar to the $H\to ZZ*$ analysis, a kinematic discriminant 
built according to equation~\ref{eq:KD} can be used to measure 
$f_3$ according to equation~\ref{eq:fa3}.  
Figure~\ref{fig:???} shows this discriminant for SM Higgs
events, pseudoscalar events, and a mixed parity scalar
corresponding to $f_3=0.5$ for a center of mass energy of 250~GeV.
To justify that there are no
biases introduced by approximation in equation~\ref{eq:fa3},
toys studies have been performed for various values of 
$f_3$.  These studies are summarized in figure~\ref{fig:???};
no biases are observed.  

Projections for a future $e^+e^-$
colliders are estimated assuming a collision energy  
of 250~GeV and an integrated luminosity of 250~$fb^{-1}$.
Signal events are simulated with JHUGen.  Background events
are modeled using $e^+e^-\to ZZ$ events simulated using
{\it MADGRAPH}.  
The cross sections and event yields for the signal and
background processes are detailed in 
table~\ref{table:FutureMeasEventYields}
which are based on previous studies in references~\cite{???}.

All events are required to have two lepton whose 
transverse momentum is greater than 5~GeV, $|\eta|<2.4$, and 
Higgs boson mass between $115<m_H<140~GeV$.  The reconstruction
efficiency for signal is found to be above 90\%. Although
the background process is likely not fully representative
of the expected backgrounds that will exist in $e^+e^-$
collisions, for the purpose of this study, the exact modeling 
of background events is not crucial.  

The distribution of $D_{0-}$ for SM Higgs, a pseudoscalar,
and SM background after all selections is shown in 
figure~\ref{fig:???}.  Similar to before, toys are generated
and fit using equation~\ref{eq:fa3}.  The distribution of
the best-fit $f_3$ for various input models is shown in 
figure~\ref{fig:???}.  The expected sensivity is found to be 
$\sigma_{f_3}=0.0XXX$.  Converting this to the $f_3$ parameter
currently being measured at the LHC, $f_3(H\to ZZ^*)\equiv f_{a3}$, 
the distribution of the best-fit $f_{a3}$ value is shown in 
figure~\ref{fig:???} and the error on this parameter is found
to be $\sigma_{f_{a3}}=0.000XX$.

\section{Other Channels}

As argued previously, the sensitivity to CP-violating 
interactions in the 
HZZ amplitude is markedly better using $e^+e^-$ collisions.  
However, it should be noted that this simple exercise does not
completely diminish the potential for similar measurements at 
the LHC.  As table~\ref{table:fa3Conversion} also demonstrate,
other process at the LHC, $q\bar{q}\to H + q\bar{q}$ and
$q\bar{q}\to Z^* \to ZH$, also benefit from enhanced $\sigma_3$
due to the isolated phase space that they probe.  As these channels 
continue to gain sensitivity to signal events, they will play an
increasingly important role in constraining CP-violation.
Detailed studies are still to be done, but these channels may
ultimately dominate the sensitivity to $f_3$ at the LHC.  

\section{Summary}

There are several complications involved with applying
multidimensional fit to the $H\to ZZ^*$ or other processes
analysis exit:
modeling an multidimensional transfer function appropriate
to event reconstruction and analysis selections; describing all 
backgrounds accurately; and building likelihoods which can be
efficiently minimized. However, multidimensional fits using the
differential cross section calculation described in
chapter~\ref{sec:HiggsPhen} provides a flexible approach which
could ultimately measure each of the model parameters which 
describe the HZZ amplitude.  

A number of the challenges related to multidimensional fits
can migitated using the MELA
technique discussed in chapter~\ref{sec:HiggsPhen} and applied
in section~\ref{sec:HZZ4l} as well as here.  These techniques
help largely because the problem is reduced from using
many observables to using at most a couple of observables.  It
was also shown that, just like with the multidimensional fits,
these techniques are applicable to more processes than just 
$H\to ZZ^*$.

Current measurements being done at CMS to constraint 
CP-violating interactions are only making use of $H\to ZZ^*$
events.  However, as argued in the previous section, it is
likely that other channels at the LHC will one day probe much
larger regions of the parameters space.  Eventually, it is 
possible to use an $e^+e^-$ collider to perform high precision
measurements of Higgs properties. 

Finally, it was demonstrated in chapter~\ref{sec:HiggsPhen}
that using the MELA technique to measure $f_{2}$ is currently 
not possible.  In the absense of other novel techniques, it
will be critical to using multidimensional fits to probe 
anomalous CP-even couplings in the HZZ amplitude.  

The studies presented in this chapter provide a glimse of the 
natural pogression of the tools developed in
chapter~\ref{sec:HiggsPhen}.  The use of multidimensional 
tools as well as the application of the MELA techniques to
other processes at the LHC will start to drive precision
measurements being done via event kinematics.  Ultimately, 
these tools may become a staple of Higgs property measurements
for many year, even after the LHC has run its course.  




