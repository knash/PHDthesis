\chapter{Higgs searches with ZZ decays}
\label{sec:HZZsearches}
\chaptermark{Higgs searches with ZZ decays}

The ZZ channel is particularly well suited for high mass ($m_H>200 GeV$)
where the branching ratios to WW and ZZ are dominant.  The ZZ channel 
has the advantage that there are several fully reconstructable final
states: the $4\ell$ final state, also known as the golden channel, and
the $2\ell 2q$ channel.  While the $4\ell$ channels has very good 
mass resolution and low background, it suffers from low branching ratios,
$\mathscr{Br}(Z\to \ell\bar{\ell})\sim3$\%.  In complement, the $2\ell2q$
channel has considerably larger background and mass resolution, but the
hadronic branching ratio for the Z is large, mostly due to the large 
multiplicity of this final state.  The $4\ell$ channel is expected to 
provide high sensitivity to a broad range of Higgs mass hypotheses, while
the dominant sensitivity for $2\ell 2q$ wil occur at high mass and only 
moderate sensitivity can be acheived below the ZZ kinematic threshold.

In this chapter, two analyses will be presented in which Higgs searches
are performed over the entire range of valid Higgs masses.  The first 
section will concentrate on the semileptonic final state.  Novel analysis
techniques to reduce and control for the background are presented.  
The sensitivity is found to be competative with that expected from the 
$4\ell$ channel.  The second section will discuss Higgs searches in the 
context of the $4\ell$ final state in which a significant excess of 
events has been observed consistent with a narrow width neural bosonic
resonance.  The corresponding cross section of the excess
is interpretted in the context of SM Higgs expectation and property
measurements are performed using event kinematics to constrain both the 
spin and parity of the observed resonance. 

