\def\cm{\hbox{$\;\hbox{\rm cm}$}}
\def\eV{\hbox{$\;\hbox{\rm eV}$}}
\def\GeV{\hbox{$\;\hbox{\rm GeV}$}}
\def\MeV{\hbox{$\;\hbox{\rm MeV}$}}
\def\TeV{\hbox{$\;\hbox{\rm TeV}$}}
\def\ifb{\hbox{$fb^{-1}\rm$}}

\def\POWHEG{{\it POWHEG}}
\def\JHUGen{{\it JHUGen}}
\def\MCFM{{\it MCFM}}
\def\PYTHIA{{\it PYTHIA}}
\def\MADGRAPH{{\it MADGRAPH}}
\def\GEANTfour{{\it GEANT4}}

\newcommand{\KD}{$\mathscr{D}^{\rm kin}_{\rm bkg}$}
\newcommand{\superKD}{$\mathscr{D}_{\rm bkg}$}
\newcommand{\spinKD}{$\mathscr{D}_{J^P}$}
\newcommand{\pt}{$p_{T}$}
\newcommand{\sip}{$SIP_{3D}$}
\newcommand{\isocomb}{$ISO_{comb}$}

\newcommand{\2011Lumi}{5.1~\ifb}
\newcommand{\2012Lumi}{19.6~\ifb}

\chapter{Higgs searches with ZZ decays}
\label{sec:HZZsearches}
\chaptermark{Higgs searches with ZZ decays}



\section{Semi-leptonic decay channel}

\section{Golden decay channel}


\subsection{Electron Reconstruction and Identification}
\label{sec:HZZ4lelectrons}

\subsection{Muon Reconstruction and Identification}
\label{sec:HZZ4lmuons}

\subsection{Photon Reconstruction and Identification}
\label{sec:HZZ4lphotons}

\subsection{Jet Reconstruction and Identification}
\label{sec:HZZ4ljets}

\subsection{Datasets and Triggers}
\label{sec:HZZ4ldatasets}

Events used in the $ZZ\to 4\ell$ analysis are selected via either the
double electron, double muon, or triple electron triggers.  The
double electron and muon triggers require that the total momentum
of the leading and sub-leading leptons be greater than 17 and 
8~\GeV respectively; the triple electron triggers requires that
the total momentum for the leading, sub-leading, and 
sub-sub-leading lepton be greater than 15, 8, and 5~\GeV 
respectively.  The efficiencies for these triggers have been 
measured on data and are found to be 98\% for a SM Higgs boson 
with a mass of 120~\GeV.  

Monte Carlo (MC) simulations have been used to modeled signal 
and background events.  These simulations have been used as a
reference for optimizing analysis selections, evaluate acceptance
efficiencies, estimate systematic uncertainties.  Signal samples
are generated using either \POWHEG~\cite{???} at next-to-leading 
order (NLO) in $\alpha_s$ for SM Higgs couplings or with 
JHUGen~\cite{???} at leading order (LO) in $\alpha_s$ for either
SM Higgs couplings or non-SM scalar couplings and higher spin
models.  For simulation of Higgs bosons produced in association
with either weak vector bosons, VH, or $t\bar{t}$ pairs, ttH, 
the event generator \PYTHIA~\cite{???} is used.  

Since the \POWHEG~ samples do not model the interference
of final leptons for the $4\mu$ and $4e$ channels.  These samples
are reweighted using the JHUGen matrix element calculation where
appropriate.  However, the branching fractions 
$\mathscr{B}(H\to 4\ell)$ are taken from {\it PROPHECY4F} which 
includes both interference effects and NLO QCD/EW corrections. 
The narrow-width approximation for the $m_{4\ell}$
line shape is employed at low mass resulting in a Breit-Wigner
distribution.  At larger masses were the Higgs width become large,
the $m_{4\ell}$ line shape is reweighted to match the complex-pole
scheme described in~\cite{???}.  Effects from the intereference
bewteen signal and the continuum $gg\to ZZ$ production is also 
accounted for following the prescription of~\cite{???}.

The SM continuum production of ZZ events via $q\bar{q}$
annihilation is simulated at NLO using \POWHEG~.  The gluon-gluon
fusion production of continuum ZZ events is simulated using
{\it GG2ZZ}~\cite{???}. Other di-boson background as well 
as Drell-Yan events  are simulated at LO using \MADGRAPH~.  
Di-boson samples produced at leading order are rescaled to match
cross sections predicted by NLO calculations while Drell-Yan 
samples are rescaled to match cross sections predicted by NNLO
calculations.  Finally $t\bar{t}$ events are simulated at NLO
with \POWHEG~.

All initial-state and final-state radiation is modelled using 
\PYTHIA~ and parton density function are taken from 
{\it CTEQ6L}~\cite{???} ({\it CT10}~\cite{???}) for LO (NLO) 
generators.  Detector effects and event reconstruction is 
simulated using {\it GEANT4}.  The number of reconstructed 
vertices per collision is reweighted to match the distribution 
seen in data.  

The generators and cross sections for each of these event types 
is shown in table~\ref{table:MCsamples}

\begin{table}
\begin{tabular}{c|c|c}
\hline 
\hline
$H\to ZZ^{(*)}\to 4\ell$ & \POWHEG & --- \\
$H\to ZZ^{(*)}\to 4\ell$ & JHUGen & --- \\
$X(0^{+}_{h})\to ZZ^{(*)}\to 4\ell$ & JHUGen & --- \\
$X(0^{-})\to ZZ^{(*)}\to 4\ell$ & JHUGen & --- \\
$X(1^{+})\to ZZ^{(*)}\to 4\ell$ & JHUGen & --- \\
$X(1^{-})\to ZZ^{(*)}\to 4\ell$ & JHUGen & --- \\
$X(2^{+}_{m,g\bar{q}})\to ZZ^{(*)}\to 4\ell$ & JHUGen & --- \\
$X(2^{+}_{m,gg})\to ZZ^{(*)}\to 4\ell$ & JHUGen & --- \\
$X(2^{+}_{h})\to ZZ^{(*)}\to 4\ell$ & JHUGen & --- \\
$X(2^{-}_{h})\to ZZ^{(*)}\to 4\ell$ & JHUGen & --- \\
\hline
\hline
\end{tabular}
\label{table:HZZ4lMCsamples}
\caption{List of MC samples used for the $ZZ^{(*)}\to 4\ell$ analysis.
along with the event generator used to simulate them and their
respective cross sections.  Note, the meaning of cross sections
for non-SM samples is highly model-dependent.  As a results, 
cross sections are ommitted for these samples.  An examplation
of how the number of events is interpretted will be presented in
section~\ref{sec:???}.}
\end{table}

\subsection{Event Selection and Categorisation}
\label{sec:HZZ4lselection}

Selections are applied to increase the purity of signal events 
as much as possible.  Selections based isolation and identification
requirements are used to reduce background in which the physical
process does not produce four leptons, e.g. $Z+jet$ events, 
generally referred to as instrumental backgrounds.  All 
reconstructed leptons are also required to have in impact
parameter which is sufficiently compatible with the primary 
vertex, which is quantified by the significance of the impact 
parameter, \sip. Each event is required to have 4 leptons whose
combine isolation, \isocomb, and \sip~ satisfy,

\begin{equation}
ISO_{comb}<x.xx ; SIP_{3D}<4.0.
\end{equation}

Events are then classified into a number of categories as 
outlined in table~\ref{???}.  Categories which make up the 
signal region always consist of events with two oppositely 
charged lepton pairs. The signal regions are then further 
broken up into categories based on the number of jets and
lepton flavors.  The former improves sensitivity to various
production mechanisms, especially VBF where at least two 
additional jets are always produced.  Events are either 
in the dijet tag category if there are at least two jets
or in the inclusive non-dijet category if there are less
than two jets.  The latter optimizes
use of the $m_{4\ell}$ line shape by group events with similar
resolution.  Control regions in which either looser ID 
requirements or same-sign leptons pairs 
are used to estimate the amount of instrumental background 
from data. 

\begin{table}
\begin{center}
\begin{tabular}{c|c|c|c}
\hline 
\hline
category & notes & resolution & $N_{exp}$ per \ifb \\
\hline 
\hline
$H\to 4\mu$ & -- && \\ \hline
$H\to 2e2\mu$& -- && \\ \hline
$H\to 4e$   & -- &&\\ \hline \hline
$Hjj\to 4\mu$& -- && \\ \hline 
$Hjj\to 2e2\mu$& -- && \\ \hline 
$Hjj\to 4e$& -- && \\ \hline
\hline
\hline
\end{tabular}
\label{table:HZZ4lCategories}
\caption{...}
\end{center}
\end{table}

A minimal amount of the kinematic selections are applied to 
further reduce the irreducible backgrounds, continuum ZZ 
production.  In order to reduce the contamination of low-mass
resonances, such as $J/\psi$'s, all dilepton pairings are 
required to have a 
minimum invariant mass, $m_{\ell\ell}>4$\GeV.  Dilepton 
pairings whose invariant mass is closest to the Z pole-mass
is referred to as $Z_{1}$, while the other pairing is referred
to as $Z_{2}$.  The invariant mass of these dilepton pairs is 
denoted by $m_1$ and $m_2$, respectively, and are required
to satisfy $12<m_2<120$~\GeV~and $40<m_1<120$~\GeV.  The 
leading and subleading leptons are required to have
\pt$>20$ and \pt$>10$~\GeV, respectively.

\section{Yields and Kinematics Distributions}
\label{sec:HZZ4lyields}

The expected shape and normalization of irreducible 
background events is taken from MC simulation.  Cross sections
for $q\bar{q}$ annihilation and gg initiated events are 
calculated at NLO using \MCFM.  Systematic variation due to 
QCD renormalization scale, factorization scale, and parton 
distribution functions, are calculated as a function of $m_{4\ell}$
following the PDF4LHC prescription~\ref{???}. The total 
uncertainties from QCD and PDFs are typically 8\%.  

{\it NEED TO REWRITE}

To estimate the reducible ($Zb\bar{b}$, $t\bar{t}$) and 
instrumental ($Z+light jets$, $WZ+jets$) backgrounds, a 
$Z_1+X$ background control region, well separated from the 
signal region, is defineed.  In addition, a sample $Z_1+\ell_{reco}$,
with at least one reconstructed lepton object, is defined for the
measurement of the lepton misidentification probability - the
probability for a reconstructed objet to pass the isolation and 
identification requirements.  The contamination from WZ in these
events is suppressed by requiring the imbalance of the measured
energy deposition in the transverse plane to be below 25~\GeV. 
Then lepton misidentification probability is compared, and found
compatible, with the one derived from MC simulation.

The event rates measured in the background control region are
extrapolated to the signal region.  Two different approaches are
used.  They differ in the way the contribution from electrons
coming from photon conversions is handled.  Both start by 
relaxing the isolation and identification criteria for two 
additional reconstructed lepton objects.  The first approach 
follows from the previous CMS analysis [14].  It aims at
estimating all contributions of reducible background in one 
single step.  The additional pair of leptons is required to have
the same charge (to avoid signal contamination) and same flavour,
a reconstructed invariant mass .. 

{\it END OF REWRITE}

Systematic uncertainties are evaluated from data for trigger 
and combined lepton reconstruction, identification, and
isolation efficiencies.  Sample of $Z\to\ell\ell$, 
$\Upsilon\to\ell\ell$, and $J/\psi\to\ell\ell$ events are used
to set and validate the abolute momentum scale and resolution.  
More details on the measurement and validation of momentum 
scale and resolution can be found in Appendix A of~\ref{???}.
Additional systematics arrise from limited statistics in 
background control regions as well as systematic differences 
between the control regions.  All systematics for background
are tabulated in table~\ref{???}.

Higgs production cross sections and branching rations as well 
as theoretical uncertainties on the crossare taken from the LHC 
Higgs Cross Section Working Group~\ref{???}.  Efficiency of 
signal events are derived from MC and corrected for differences
seen between data and MC in $Z\to\ell\ell$ events.  Shapes of 
signal distributions are taken from MC.  All systematic 
uncertainties for signal events are tabulated in table~\ref{???}.

\section{Cross Section Measurement}
\label{sec:HZZ4lxsec}

The expected and observed yields for the different event classes
is shown in tables~\ref{table:HZZ4lYieldsLowMass},~\ref{table:HZZ4lYields}.  
The distributions of events in $m_{4\ell}$ is show in figure~\ref{??},
where the reducible and irreducible backrounds are shown in green
and blue respectively.  The expected shape of a 126~\GeV SM Higgs
signal is shown in red.  The data shows a clear excess of events 
which are consistent with a SM Higgs.  Elsewhere, there are no
significant deviations from SM expectation found.  

To quantify the statistical significance of the observed data
with respect to either background or signal and background, event
are described in terms of two variables, $m_{4\ell}$ and $K_D$.
The $K_D$ variable is built from the matrix element calculations
for signal and background and provides close to optimal separation
bewteen a SM Higgs and ZZ events produced in $q\bar{q}$
annihilation.  The input matrix element calculations are taken
from \JHUGen, $ME_{SMH}$, and \MCFM, $ME_{qqZZ}$, for signal and 
background processes, respectively.  The kinematic discriminant, 
$K_D$ is then built as,
\begin{equation}
K_D = \frac{ME_{SMH}}{ME_{SMH}+ME_{qqZZ}}
\end{equation}
The expected and observed distribution of events in the 
$m_{4\ell}-K_D$ plane are shown in figure~\ref{???}.  The signal
tends to not only be concentrated in narrow regions of $m_{4\ell}$
but also tends to be distributed closer to 1 in $K_D$.  In 
contrast, background tends to be broad in $m_{4\ell}$ and
distributed closer to 0 in $K_D$

To quantify the compatibility between data and the expected
background, upper limits are quoted over the entire mass range.  
The expected and observed 95\% confidence level upper limits on
$\sigma/\sigma_{SM}$ are
shown in figure~\ref{???}.  Aside from the significant deviation
from expectation near 126~\GeV, the observed upper limits are 
always consistent with expectation to within $2\sigma$.  The
observed upper limit imply that the current data is sufficient
to rule out SM Higgs mass hypotheses bewteen $130<m_{H}<827$~\GeV.

The excess around 126 GeV can be interpretted in terms of the 
p-value with respected to the background only hypothesis.  
The p-value scan as a function of the hypothetical Higgs mass
is shown in figure~\ref{???}.  The maximum local p-value 
occurs at 125.7~\GeV~and has a value of $7.3\sigma$.  This 
significant deviation from the background-only hypothesis, has 
a cross section which is compatible 
with that expected from the SM Higgs.  The ratio of the best-fit
cross section with respected to the expected SM Higgs cross
section, $\mu=\sigma_{obs}/\sigma_{SM}$, is found to be 
$0.9XXX\pm0.XXX$.  The width of the particle is found to be 
constistent with be much smaller than the experimental resolution
of $m_{4\ell}$, consistent with expectation of a 126~\GeV~SM Higgs
boson, $\Gamma_{SMH}(126~GeV)=X.XX~MeV$.  

\begin{table}
\begin{center}
\begin{tabular}{|c|c|c|c|c|}
\hline 
Channel & $4e$ & $4\mu$ & $2e2\mu$ & $4\ell$ \\
\hline 
\hline

ZZ background & $6.6\pm0.8$ & $13.8\pm1.0$ & $18.1\pm1.3$ & $38.5\pm1.8$ \\ \hline 
Z+X & $2.5\pm1.0$ & $1.6\pm0.6$ & $4.0\pm1.6$ & $8.1\pm2.0$ \\ \hline \hline
All background expected & $9.1\pm1.3$ & $15.4\pm1.2$ & $22.0\pm2.0$ & $46.5\pm2.7$ \\ \hline \hline
$m_{H}=125$~\GeV & $3.5\pm0.5$ & $6.8\pm0.8$ & $8.9\pm1.0$ & $19.2\pm1.4$ \\ \hline
$m_{H}=126$~\GeV & $3.9\pm0.6$ & $7.4\pm0.9$ & $9.8\pm1.1$ & $21.1\pm1.5$ \\ \hline\hline
Observed & $16$ & $23$ & $32$ & $71$ \\ \hline
\end{tabular}
\label{table:HZZ4lYieldsLowMass}
\caption{Expected and observed yields in the mass range $110<m_{4\ell}<160$ for difference class of events.}
\end{center}
\end{table}

\begin{table}
\begin{center}
\begin{tabular}{|c|c|c|c|}
\hline 
Channel & $4e$ & $4\mu$ & $2e2\mu$ \\
\hline 
\hline
ZZ background & $78.9\pm10.9$ & $118.9\pm15.5$ & $192.8\pm24.8$ \\ \hline
Z+X & $6.5\pm2.6$ & $3.8\pm1.5$ & $9.9\pm4.0$ \\ \hline \hline
All background expected & $85.5\pm11.2$ & $122.6\pm15.5$ & $202.7\pm25.2$ \\ \hline \hline
$m_{H}=125$~\GeV & $3.5\pm0.5$ & $6.8\pm0.8$ & $8.9\pm1.0$ \\ \hline
$m_{H}=126$~\GeV & $3.9\pm0.6$ & $7.4\pm0.9$ & $9.8\pm1.1$ \\ \hline
$m_{H}=500$~\GeV & $5.1\pm0.6$ & $6.8\pm0.8$ & $12.0\pm1.3$\\ \hline
$m_{H}=800$~\GeV & $0.7\pm0.1$ & $0.9\pm0.1$ & $1.6\pm0.1$ \\ \hline
\hline
Observed & $86$ & $125$ & $240$ \\ \hline
\end{tabular}
\label{table:HZZ4lYields}
\caption{Expected and observed yields in the mass range $100<m_{4\ell}<1000$ for difference class of events.}
\end{center}
\end{table}

\section{Spin and Parity Measurements}
\label{sec:HZZ4lspinParity}

Understanding whether or not this new boson is the SM Higgs, 
one of several Higgses, or even something more exotic, like 
a graviton, is currently the most promising route to searching
for physics beyond the SM.  One of the most promising methods,
which will ultimately illuminate the full structure of the $XZZ$
amplitude, is to use event kinematics in the di-boson decay
channels to constrain propeties of the Higgs candidate resonance. 

This section will following the general outline presented in 
section~\ref{sec:???}.  Experimental considerations ... 

Kinematic discriminants are built to separate either SM Higgs
signal from the dominant SM background, $q\bar{q}\to ZZ$ or to
separate SM Higgs-like events from some alternative $J^{PC}$ 
signal.  The former is built from $K_D$ incorporating also the 
$m_{4\ell}$ information according to:
\begin{equation}
D_{bkg}=...
\end{equation}
The latter discriminant is built from the ratio of the SM Higgs
matrix element and the alternative $J^{PC}$ matrix element 
according to:
\begin{equation}
D_{J^{PC}}=\frac{ME^2_{SMH}}{ME^2_{SMH}+ME^2_{J^{PC}}}
\end{equation}
There are 8 variations of the $D_{J^{PC}}$ discriminants that are
studied correspond to 8 alternative signal models, listed in table~\ref{table:HZZ4lAltSig}.

\begin{table}
\begin{center}
\begin{tabular}{|c|c|c|}
\hline 
 $J^{P}$ & production & comment \\
\hline 
\hline
$0^-$            & $gg\to X$ & pseudoscalar \\ \hline
$0_h^+$          & $gg\to X$ & high dim operator \\ \hline
$2_{m,gg}^+$      & $gg\to X$ & minimal couplings \\ \hline
$2_{m,q\bar{q}}^+$ & $q\bar{q}\to X$ & minimal couplings \\ \hline
$1_{-}$          & $q\bar{q}\to X$ & exotic vector \\ \hline
$1_{+}$          & $q\bar{q}\to X$ & exotic axialvector \\ \hline
$2_{h}^+$        & $gg\to X$ & high dim operator \\ \hline
$2_{h}^+$        & $gg\to X$ & high dim operator \\ \hline
\end{tabular}
\label{table:HZZ4lAltSig}
\caption{List of alternative signal models considered for spin and 
parity measurements.}
\end{center}
\end{table}

Similar to the cross section measurements, events are described
in the two dimensional plane, $D_{bkg}$ vs $D_{J^{PC}}$.  In this
manner the maximal amount of information is contained to have 
sensitivity to both signal-like features, as well as distinguishing
feature of a SM Higgs. 

As with the cross section measurements, the reducible and
instrumental background are found to be consistent with the
$q\bar{q}\to ZZ$ background.  Thus, the shape is models with
the irreducible MC and a systematic is assigned corresponding 
to the difference between the irreducible MC and the background
control regions.  

The shape of each of the eight $D_{J^P}$ variables is shown in 
figure~\ref{fig:???}.  Each plot shows that the SM Higgs tends
to be distributed more towards $D_{J^P}=1$ while the corresponding
alternative signal is distributed more towards $D_{J^P}=0$. The
distribution of $D_{bkg}$ is shown for each signal type and for
the different background in figure~\ref{fig:??}.  Although 
some alternative signals are more or less background like than
the SM Higgs, these variations are typically small compared to 
the difference between each signal and background.  Thus this
variable serves as a sufficient, model independent way of
isolating signal events.  

The expected production cross section for each of the alternative 
signals is taken to be the same as that of the SM Higgs.  The 
effect of different couplings on the different ZZ branching 
ratios as well as different relative efficiencies is accounted
for by calculating scale factors for each of the six different 
channels comparing SM Higgs against alternative $J^P$ samples 
with \JHUGen.  Table~\ref{table:???} shows each of these scale
factors for all alternative signals in all channels.  The large
difference in the $q\bar{q}$ initiated samples are due to the
more forward rapidity distributions of these samples relative to 
the $gg$ initiated samples.  

Hypothesis testing is done between the SM Higgs and each of the 
8 different alternative signal models.  The test statistic used
to distinguish the null hypothesis, SM Higgs, and the alternative 
hypothesis is $q=ln(\mathscr{L}_{SM}/\mathscr{L}_{J^P})$. 
Expected results are
obtained in two different ways, generating pseudoexperiments using
the SM Higgs cross section for each hypothesis, 
or using the best-fit signal strength modifying, $\mu$, for each
hypothesis individually.  These results are shown in 
table~\ref{table:??} where observed $0^+$ ($J^P$) refers to the 
p-value 
of the observed test statistic, represented by the red arrow,
calculated according to the SM (alternative signal) toy 
distribution, shown in yellow (blue), converted to normal 
quantiles.  A $CL_s$ criterion is built from the p-values 
according to:
\begin{equation}
CL_s = P(q>q_0|SM)/P(q>q_0|J^P)
\end{equation}  
All results show that data is more consistent with 
the SM expectation and disfavor the alternative hypothesis
at a level of 8.1\% or better.  

Several results show large observed significance with respect to 
the expected, namely the $1^+$, $1^-$, and $2_{m,q\bar{q}}^+$ tests.
Each of these cases have m_{Z} and m_{Z*} distributions which are
distinct from SM expectation.  As a result of a statistical
fluctuation in the tails of these distributions observed in data,
these models all have large q-values.  However, it is important
to note that not only are these results correlated as a result
of this statistical fluctuation, this effect is also one of the 
reason that the observed improvement between the 1D and 2D 
discovery significance is much larger than expected on average.

Of the 8 results presented, one of the most interesting is the 
$0^-$ hypothesis separation.  Since one of the simplest extensions
to the SM, the class of two-Higgs-doublet models (2HDMs), 
predicts the presence of a pure pseudo-scalar, ruling out the 
pure $0^-$ model suggests that the observed resonance is very 
unlikely to be a pure pseudoscalar. 

\section{Constraining CP-violation}
\label{sec:HZZ4lcpViolation}

The CP-even and CP-odd
scalars that are predicted by 2HDM models can also mix to produce
a parity-violating resonance.  It has been recently argued that
the additional amount of CP violation which is still allowed by
experimental data could explaining the observed baryon assymetry
in the universe.  Thus, constraining CP-violation in the HZZ 
amplitude is currently one of the most promising ways of probing
new physics of the SM which could help to explain not only 
theoretical problems the SM is thought to suffer from, e.g. 
fine tuning, but emperical facts the SM is currently thought to
be insufficient to explain.  

The parameter $f_{a3}$ 
described in section~\ref{sec:?????} is a measure of the amount
of CP-violation in the HZZ amplitude.  Given that $f_{a3}=1$
has been ruled out through hypothesis testing in favor of the
SM Higgs hypothesis at the level of $3.3\sigma$.  Measuring any
non-zero value of $f_{a3}$ would be direct evidence of CP-violation
given that $f_{a2}$ is zero.  

The $D_{0^-}$ variable used for hypothesis testing in
section~\ref{sec:HZZ4lspinParity} is suitable for measuring the
value of $f_{a3}$.  A simplified model for a mixed-CP state can 
be constructed 
using a weighted sum of the $D_{0^-}$ distribution for a SM Higgs
and a psuedoscalar, 
\begin{equation}
\mathscr{P}_{mixed-CP} = (1-f_{a3})\mathscr{P}_{SM} 
+ f_{a3}\mathscr{P}_{0^-}.
\end{equation}
This equation assumes implicitly that the two term do not 
interfer.  
