
\section{Semi-leptonic decay channel}

The semileptonic final state of the ZZ channel is studied in two 
different kinematic regions, the low mass region 
($125 < m_{2\ell2q} < 170 GeV$) and the high mass region 
($183 < m_{2\ell2q} < 800 GeV$).  Because of the small ZZ branching ratio
expected from the SM Higgs, the intermediate range 
($170 < m_{2\ell2q} < 183 GeV$) is not considered in this analysis.

\subsection{Event Simulation}
\label{sec:HZZ2l2qSimulation}

The analysis strategy include selections and data-driven background
estimations were optimized and validated on MC simulations.  Signal
samples are generated with {\it POWHEG}~\cite{??} and JHUGen~\cite{??}.  
Inclusive Z 
production is generated with either {\it MADGRAPH} 4.4.12~\cite{??} or
{\it ALPGEN} 2.13~\cite{??}.  Continuum diboson production, ZZ, WW, 
and ZW, samples are generated with {\it PYTHIA} 6.4.22~\cite{???}.  
top backgrounds are generated with either {\it MADGRAPH} 4.4.12 or 
{\it POWHEG}.  Parton distribution functions are modeled using 
{\it CTEQ6}~\cite{???} at leading order and {\it CT10}~\cite{??} at 
next-to-leading order (NLO).  Parton showering is modeled with {\it PYTHIA}
while detector response is simulated with a CMS specific implementation
of {\it GEANT4}~\cite{??}.  A full list of the MC samples used is shown 
in table~\ref{table:??} along with the cross section for each process.  

\subsection{Event Selection and Categorisation}
\label{sec:HZZ2l2qselection}

Only events which contain two oppositely charge leptons, either electrons or
muons, and two jets are considered in this analysis.  Both leptons
flavors are required to have tranverse momentum, $p_T$, greater than 20 GeV
and 10 GeV for the leading and subleading $p_T$, respectively.  For events 
which are used in the high mass analysis, this constraint is tightened to 
$p_T>40, 20 GeV$.  Only muons (electrons) in the pseudorapidity range 
$|\eta|<2.4 (2.5)$, excluding the gap between the barrel and endcap region, 
are considered.  These selections not only serve as a rudimentary method for
rejecting background but are consistent with the double electron and 
double muon triggers that are used.  The efficiency of these triggers are 
measured on data and found to be X.XX.  MC is corrected for any residual 
inefficiency found between MC and data. Muons are required to be well 
isolated from hadronic activity in the detector by restricting the sum of
transverse momentum from the tracker or transverse energy in the ECAL and
HCAL within a cone of $\Delta R  = \sqrt{(\Delta\eta)^2+(\Delta\phi)^2}<0.3$
to be less than 15\% of the measured $p_T$.  Similar requirements are placed
on electrons although the details depend also on the electron shower shape.

Jets are reconstructed with the particle-flow (PF) algorithm~\cite{???}.
Reconstructed particle  candidates are clusted with the anti-$k_T$ 
algorithm~\cite{???} with a distance parameter $R=0.5$.  All jets which 
overlap with well-isolated leptons are then removed from consideration.  
Jets are required to be in the tracker acceptance, $|\eta<2.4|$, to maximize
the effectivness of the PF algorithm.  Energy corrections  are applied
to jets to account for systematic instrumental effects including the 
non-linear energy response of the calorimeters.  These corrections are 
derived from in-situ measurements using dijet and $\gamma+$jet control
samples~\cite{???}. To mitigate the additional energy in the calorimeters 
from pileup, the average energy density times the jet area is subtracted 
from each jet~\cite{???}.  Some requirement is also applied to the energy
balance between the charged and neutral hadronic content in each jet.  
Finally, all jets are required to have $p_T>30 GeV$.

With the basic objects in hand, the $2\ell2q$ system is constructed under
the assumption that all pairs of leptons and quarks are the duaghters of
Z bosons.  Each di-lepton pair must have a combined invariant mass of 
$70 < m_{\ell\ell} < 110$.  This is primary meant to reduce backgrounds which
don't have an intermediate Z, like $t\bar{t}$ and QCD backgrounds.  In order 
to reduce the overwhelming Z+jets background, the dijet invariant mass of 
the event is required to satisfy $75 < m_{jj} < 105$.  Figure~\ref{fig:HZZ2l2qPreselectionOnly} shows the $m_{jj}$ for signal and background.  

Catagorizing events based on jet flavor provides a significance 
increase in sensitivity to signal events since two b-jets are 
more likely to result from a Z decay than from QCD radiation.  
To isolate jets which are likely to be b-jets, the CMS track 
counting high-efficiency (TCHE) b-tagging algorithm~\cite{???} 
is used.  This algorithm relies on tracks with large impact 
parameters. {\it explain a little better.} The discriminant used
to determine how b-like jets are in shown in figure~\ref{fig:??}.
Using this discriminant, the events are divided into three
categories: those which have at least one jet passing the median 
working point ($\sim65\%$ efficient) and another jet passing the
loose working point ($\sim80\%$ efficient); those which have at 
least one jet passing the loose working point but; those which 
have zero jets passing the loose working point.  Although there 
is a non-negligible mistag rate for each of these working points, 
the categories are referred to as 2 b-tag, 1 b-tag, and 
0 b-tag, respectively. The categories are defined such that 
they are mutually exclusive by putting events in the category 
with the most stringent requirements.  The division of events
in each of the three categories after preselection requirements
is shown if figure~\ref{fig:???}.

Since there is a significant amount of $t\bar{t}$ and $tW$ 
events in the 2 b-tag category, events are required to have
missing transverse energy, $E_T^{miss}$ which is compatible with 
zero.  The $E_T^{miss}$ is defined to be the modulus of the 
negative vector sum of all the PF candidates in the event.  
To quantify this, a likelihood ratio, $\lambda(E_T^{miss})$ 
is built comparing two hypothesis, $E_T^{miss}=0$ and 
$E_T^{miss}\neq 0$.  Events in the 2 b-tag category are then 
required to satisfy $2\ln\lambda(E_T^{miss})<10$.  In the 
low mass analysis, we instead require $E_T^{miss}<50~GeV$ in
the 2 b-tag category.    

In order to reduce further the overwhelming background, Z+jets,
a kinematic discriminant is built to distinguish between
Higgs-like and Z+jets-like events.  The basis of variables 
that is used as well as the distributions that describe the 
signal events is described in the section~\ref{sec:HiggsPhen}.
The discrimant that is used makes use of the 5D probability 
distributions, 
$\mathscr{P}(cos\theta^*,cos\theta_1,cos\theta_2,\Phi,\Phi_1|m_{ZZ})$,
to build a discriminant,
\begin{equation}
D = \frac{\mathscr{P}_{Higgs}}{\mathscr{P}_{Higgs}+\mathscr{P}_{Zjets}}.
\label{eq:KD}
\end{equation}

Since the dominant background can't easily be described in 
terms of angular distributions, the distributions are found 
emperically from MC using approximate functions.  Fits to 
Z+jets MC are performed in slices of $m_{ZZ}$.  A function
is then extrapolated between slices so that $\mathscr{P}_{Zjets}$ 
is continuous in $m_{ZZ}$.  Some examples of these fits are shown
in figure~\ref{fig:HZZ2l2qBackgroundPDF}.  Since the MC includes 
detector effects, 
the signal parameterization should also include detector effects.
The ideal distributions from section~\ref{sec:HiggsPhen} are 
modified with 5D uncorrelated function which is then fit to 
MC to account for any detector effects.  As with background, 
these fits are performed in slices of $m_{ZZ}$ and extrapolated
to arbitrary values.  Examples of the signal parameterizations
are shown in figure~\ref{fig:HZZ2l2qSignalPDF}.  Combining these 
two density
functions together, the discriminant, D, is shown in
figure~\ref{fig:HZZ2l2qAngularLD}.  The signal events tend to peak more 
towards 1 while the background events tend to peak more towards
zero.  Thus, selecting events greater than some value will
increase the signal purity.  

Because the shape of D changes with $m_{ZZ}$, the optimal cut will 
be $m_{ZZ}$ dependent.
An optimization was run using $\kappa=N_{sig}/\sqrt{N_{bkg}}$ as a figure of 
merrit.  This variable represents an approximation of the expected 
upper limit, UL.  For a simple counting experiment in which the 
expected number of background events is large, $\kappa$ is a good
approximation of the true UL.  The optimization was performed 
separately for each of the three b-tagging categories and the
proposed cuts are shown in table~\ref{table:HZZ2l2qCuts}.

{\it discuss the effect a cut on D has on $m_{ZZ}$ compared with other vars.}

\begin{figure}
\begin{center}
%\includegraphics[width=.49\linewidth]{}
%\includegraphics[width=.49\linewidth]{}\\
%\includegraphics[width=.49\linewidth]{}
%\includegraphics[width=.49\linewidth]{}
\label{fig:HZZ2l2qBackgroundPDF}
\caption{ Emperical derivation of 5D PDF for Z+jets events.  Points
represent histograms of MC, lines represent the final model at the 
median $m_{ZZ}$ value.}
\end{center}
\end{figure}

\begin{figure}
\begin{center}
%\includegraphics[width=.49\linewidth]{}
%\includegraphics[width=.49\linewidth]{}\\
%\includegraphics[width=.49\linewidth]{}
%\includegraphics[width=.49\linewidth]{}
\label{fig:HZZ2l2qSignalPDF}
\caption{ Emperical derivation of 5D PDF for signal events.  Points
represent histograms of MC, lines represent the final model at the 
median $m_{ZZ}$ value. }
\end{center}
\end{figure}

\begin{table}
\begin{center}
\begin{tabular}{c|c|c|c}
\hline 
\hline

\multicolumn{4}{c}{preselection} \\ \hline
$p_T(\ell^\pm)$    & \multicolumn{3}{c}{leading $p_T>40(20)~GeV$, subleading $p_T>20(10)~GeV$} \\ 
$p_T(jets)$       & \multicolumn{3}{c}{$>30~GeV$} \\ 
$|\eta|(\ell^\pm)$ & \multicolumn{3}{c}{$<2.5(e^\pm), <2.4(\mu^\pm)$} \\ 
$|\eta|(jets)$    &  \multicolumn{3}{c}{$<2.4$} \\ \hline \hline
\multicolumn{4}{c}{final selection} \\ \hline \hline
           & 0 b-tag & 1 b-tag & 2 b-tag \\ \hline
b-tag      & none    & 1 loose & 1 loose \& 1 medium \\ 
D          & $>0.55+0.00025m_{ZZ}$ & $>0.302+0.000656m_{ZZ}$ & $>0.5$ \\
$E_T^{miss}$ & none   & none    & $2\ln\lambda(E_T^{miss})<10$ \\
&&& ($E_T^{miss}<50~GeV$) \\ \hline
$m_{jj}$    & \multicolumn{3}{c}{$\in [75,105]~GeV$} \\
$m_{\ell\ell}$& \multicolumn{3}{c}{$\in [70,110](<80)~GeV$}\\
$m_{ZZ}$    & \multicolumn{3}{c}{$\in [183,800] (\in [125,170])~GeV$} \\ \hline \hline
\end{tabular}
\label{table:HZZ2l2qCuts}
\caption{Table listing analysis selections.  The top portion details 
preselection cuts applied to all objects to be consistent with trigger 
requirements and detector acceptance.  The bottom portion details all 
cuts applied in each of the different b-tag categories to optimize the 
sensitivity to signal events.}
\end{center}
\end{table}

\begin{figure}
\begin{center}
\includegraphics[width=.49\linewidth]{HZZ2l2qPlots/data_loose_mjj.pdf}
\includegraphics[width=.49\linewidth]{HZZ2l2qPlots/data_loose_tche.pdf}\\
\includegraphics[width=.49\linewidth]{HZZ2l2qPlots/data_loose_tag.pdf}
\includegraphics[width=.49\linewidth]{HZZ2l2qPlots/data_loose_met.pdf}
\label{fig:HZZ2l2qPreselectiOnly}
\caption{ ... }
\end{center}
\end{figure}

\begin{figure}
\begin{center}
\includegraphics[width=.49\linewidth]{HZZ2l2qPlots/angle_cosThetaStar.pdf}
\includegraphics[width=.49\linewidth]{HZZ2l2qPlots/angle_cosTheta1.pdf}\\
\includegraphics[width=.49\linewidth]{HZZ2l2qPlots/angle_cosTheta2.pdf}
\includegraphics[width=.49\linewidth]{HZZ2l2qPlots/angle_phi.pdf}\\
\includegraphics[width=.49\linewidth]{HZZ2l2qPlots/angle_phi1.pdf}
\includegraphics[width=.49\linewidth]{HZZ2l2qPlots/data_loose_ld.pdf}
\label{fig:HZZ2l2qAngularLD}
\caption{ ... }
\end{center}
\end{figure}

\subsection{Yields and Kinematics Distributions}
\label{sec:HZZ2l2qyields}

From figures~\ref{fig:HZZ2l2qPreselectionOnly} and~\ref{fig:HZZ2l2qAngularLD}
it is clear that the agreement between data and MC is fairly good,  
although there are some disagreements in some of the distributions.
These disagreements reflect the complexity that exist in modeling of
inclusive Z production.  To ensure that background estimations are
reliable in the more restricted phase space of the final selections,
it is important to have a methodology for measuring background 
shapes and normalizations directly on data.  

To do this data control regions are defined using event passing all
of the final selctions in table~\ref{table:HZZ2l2qCuts} but instead
lie in the regions $60 < m_{jj} < 75~GeV$ or $105 < m_{jj} < 130~GeV$. 
This regions is mutually exclusive from the signal region and includes
only a small contribution from signal events, as evident from 
figure~\ref{fig:HZZ2l2qPreselectiOnly}.  Since the kinematics of
this control region are not expected to be the same as the signal 
region, events in this region are reweighted to account for the 
differences between the signal region and the control region.  The
weights for doing so are derived from MC.  Thus the expected number
of background events in a given $m_{ZZ}$ range is given by
\begin{equation}
N_{bkg}(m_{ZZ}) = N_{con}(m_{ZZ})\times\frac{N_{bkg}^{sim}(m_{ZZ})}{N_{con}^{sim}(m_{ZZ})}=N_{con}(m_{ZZ})\times\alpha(m_{ZZ}).
\label{eq:HZZ2l2qAlpha}
\end{equation}
Where $N_{bkg}$ is the number of events expected in data in the 
signal region, $N_{con}$ is the number of events observed in the
data control region, and $N_{con}^{sim}$, $N_{bkg}^{sim}$ are the 
events measured in the MC control region and signal region, 
respectively.  Thus, $\alpha$ represents the weight to be applied
to the data control region inorder to estimate the signal region. 


These weights range between 0.75 and 1.2 and have been calculated
with two different MC generators, {\it ALPGEN} and {\it ??? }, 
both gave statistically consistent results.  This method allows
for both the expected shape and the normalization of the SM
background to be calculated for each of the three b-tag categories.
Once the expected distributions are calculated, the shape of the
background is to the rescaled control region using an empirical
function that was found to provide a good description of the 
background in the three different b-tag categories in MC.  
The uncertainties of the fit parameters are taken as systematic
uncertainties.  Statistical uncertainties of $\alpha$ are also 
propagated to the systematic uncertainties.  
Figure~\ref{fig:HZZ2l2qMassDistributions} shows the expected 
shape and normalization taken directly from MC (filled histograms), 
the data-driven estimation of the background shape and normalization
(blue line), and the observed distribution from data. Although
the MC generally does a reasonably good job of describing the 
observed distribution, there are some minor systematic effects which 
are corrected for by the data-driven estimation.  The SM Higgs
expectation enhanced by a factor 2 or a Higgs mass of $400~GeV$ 
is also shown in yellow.  

\begin{figure}
\begin{center}
\includegraphics[width=.49\linewidth]{HZZ2l2qPlots/lowmass_mzz0btag.pdf}
\includegraphics[width=.49\linewidth]{HZZ2l2qPlots/mzz_0btag.pdf}\\
\includegraphics[width=.49\linewidth]{HZZ2l2qPlots/lowmass_mzz1btag.pdf}
\includegraphics[width=.49\linewidth]{HZZ2l2qPlots/mzz_1btag.pdf}\\
\includegraphics[width=.49\linewidth]{HZZ2l2qPlots/lowmass_mzz2btag.pdf}
\includegraphics[width=.49\linewidth]{HZZ2l2qPlots/mzz_2btag.pdf}
\label{fig:HZZ2l2qMassDistributions}
\caption{The $m_{ZZ}$ invariant mass distribution after final selection in
three categories: 0 b-tag (top), 1 b-tag (middle), and 2 b-tag (bottom). 
The low-mass range, $120<m_{ZZ}<170~GeV$ is shown on the left and the
high-mass range, $183<m_{ZZ}<800~GeV$ is shown on the right.  Points with
error bars show distributions of data and solid curved lines show the 
prediction of background from the control region extrapolation procedure.
In the low-mass range, the background is estimated from the $m_{ZZ}$ for
each Higgs mass hypothesis and the average expectation is shown.  Solid
histograms depicting the background expectation from simulated events
for the different components are shown.  Also shown is the SM Higgs boson
signal with the mass of 150 (400) GeV and cross section 5 (2) times that 
of the SM Higgs boson, which roughly corrsponds to the expected exclusion 
limits in each category.}
\end{center}
\end{figure}

While the background distributions are derived from data, the signal
distributions are taken from MC.  Signal production cross sections
and branching ratios are taken from the Higgs cross sections working
group~\cite{??} and calculated at NLO.  Shapes are modeled using both
{\it POWHEG} to model production at NLO and {\it PYTHIA} to model the 
decay kinematics or {\it JHUGen} to model model both production and
decay at LO.  These shapes are fit with analytical parameterization
which are them selves parameterized as a function $m_H$.  Efficiencies 
are taken from CMS simulations and are
corrected for known differences between data and MC using tag and
probe measurements.  

A number of systematic uncertainties are associated with the calculation
of the number of expected signal events.  Many of these result from
limited understanding of reconstruction efficiencies. For example the 
muon and electron reconstruction effiencies have been assigned 
uncertainties of 2.7\%, 4.5\%, respectively.  Jet reconstruction
effiency uncertainties range from 1-8\% depending on the Higgs mass
hypothesis.  The efficiency uncertainty of $E_T^{miss}$ cuts range from
3-4\%.  The b-tagging efficiency uncertainties depend both on the 
category as well as the Higgs mass hypothesis and range between 2-11\%. 
The additional jet identification requirements applied in the 0 b-tag
category, gluon-tagging, is assigned an uncertainty of 4.6\%.
Uncertainties from Higgs production, either through parton distribution
functions, NLO corrections, or VBF modeling are assigned to both the
overall cross section calculation or the effect on acceptance due to
shape differences.  Mismodeling of Higgs $m_{ZZ}$ shape introduces some
additional systematic to the effective amount of event near the signal
peak.  Since the width depends strong on the mass hypthosis, $m_H$, the
uncertainties also depends on $m_H$ according to 
$1.5\times10^{-7}\%\times m_H^3$ [GeV].  Finally, uncetainties from 
luminosity measurements are accounted for in the signal systematics. 
All systematic uncertainties on the signal yields are summarized in 
table~\ref{table:HZZ2l2qSystematics}.

\begin{table}
\begin{center}
\begin{tabular}{l|c|c|c}
\hline 
\hline

source & 0 b-tag & 1 b-tag & 2 b-tag \\ \hline \hline

muon reconstruction & \multicolumn{3}{c}{2.7\%} \\ \hline
electron reconstruction & \multicolumn{3}{c}{4.5\%} \\ \hline
jet reconstruction & \multicolumn{3}{c}{1-8\%} \\ \hline
pile-up & \multicolumn{3}{c}{3-4\%} \\ \hline
$E_T^{miss}$ & -- & -- & 3-4\% \\ \hline
b-tagging & 2-7\% & 3-5\% & 10-11\% \\ \hline
gluon-tagging & 4.6\% & -- & -- \\ \hline
acceptance(HqT) & 2\% & 5\% & 3\% \\ \hline
acceptance(PDF) & \multicolumn{3}{c}{3\%} \\ \hline
acceptance(VBF) & \multicolumn{3}{c}{1\%} \\ \hline
signal cross section (PDF) & \multicolumn{3}{c}{8-10\%} \\ \hline
signal cross section (scale) & \multicolumn{3}{c}{8-11\%} \\ \hline
signal shape    & \multicolumn{3}{c}{$1.5\times10^{-7}\%\times m_{H}^{3}$ [GeV]} \\ \hline
luminosity      & \multicolumn{3}{c}{4.5\%} \\ \hline \hline

\end{tabular}
\label{table:HZZ2l2qSystematics}
\caption{Summary of systematic uncertainties on signal normalization.
Most sources give multiplicative uncertainties on the cross section
measurement, except for the expected Higgs boson production cross 
section, which is relevant for the measurement of the ratio to the 
SM expectation.  The ranges indicate dependence on $m_{H}$.}
\end{center}
\end{table}

\subsection{Results and Summary}
\label{sec:HZZ2l2qxsec}

Since there are no significant excesses found in any of the observed 
invariant mass spectra, limits on the Higgs cross section are calucated.
A simultanious fit for the signal cross section in the six different
channels is perform and 95\% CL limits are derived from the distribution 
of a $CL_s$ test statistic,
\begin{equation}
CL_s = \frac{x}{y}
\end{equation}
Expected limits are derived from pseudoexperimental which are generated
based on expected distributions.  Nuisance parameters associated with
the different systematic uncertainties are profiled and toys are randomly
samples for each toy.  For the low mass analysis, the median expected upper 
limit on the $H\to ZZ$ cross section, solid black line, $1\sigma$ ($2\sigma$) 
band of the expected limit,  
green (yellow), and the observed upper limit as a function of $m_H$ is 
shown in the left plot of figure~\ref{fig:HZZ2l2qLimitsSM4}.  Similarly, 
the expected and 
observed distribution of upper limits on the $H\to ZZ$ cross section are 
shown for the high mass region in right plot of 
figure~\ref{fig:HZZ2l2qLimitsSM4}.  In both plots of 
figure~\ref{fig:HZZ2l2qLimitsSM4}, the Higgs cross sections within the 
SM and SM4\cite{???} models are shown in red.  

Upper limits on the ratio of the 95\% CL cross section limits with respect
to the SM Higgs cross section, including all theoretical uncertainties, 
are shown in figure~\ref{fig:HZZ2l2qLimits}.  While the low mass region
limits are at best around four times SM Higgs cross sections, the high mass
region has an expected exclusion for Higgs masses between X.XX and Y.YY. 
The observed data excludes Higgs boson masses between X.XX and Y.YY.
In the context of the SM4 model the ranges 154-161 GeV and 200-470 GeV 
are excluded at 95\% CL. 

\begin{figure}
\begin{center}
\includegraphics[width=.49\linewidth]{HZZ2l2qPlots/plot_low_ul.pdf}
\includegraphics[width=.49\linewidth]{HZZ2l2qPlots/plot_ul.pdf}
\label{fig:HZZ2l2qLimits}
\caption{Observed (solid) and expected (dashed) 95\% CL upper limit on
the ration f the production cross section o the SM expectation for the 
HIggs boson obtained using the $CL_s$ technique.  The 68\% ($1\sigma$) 
and 95\% ($1\sigma$) ranges of expectation for the background-only model
are shown with green and yellow bands, respectively.  The solid line at 
1 indicates the SM expectation.  Left: low-mass range, right: high-mass
range. }
\end{center}
\end{figure}

\begin{figure}
\begin{center}
\includegraphics[width=.49\linewidth]{HZZ2l2qPlots/plot_low_sigma.pdf}
\includegraphics[width=.49\linewidth]{HZZ2l2qPlots/plot_sigma.pdf}
\label{fig:HZZ2l2qLimitsSM4}
\caption{Observed (dashed) and expected (solid) 95\% CL upper limit on
the product of the production cross section and branching fraction for 
$H\to ZZ$ obtained with the $CL_s$ technique.  The 68\% ($1\sigma$) 
and 95\% ($2\sigma$) ranges of expectation for the background-only model
are also shown with green and yellow bands, respectively.  The expected
product of the SM Higgs production cross section and the branching
fraction is shown as a solid red curve with a band indicating theoretical
uncertainties at 68\% CL.  The same expectation in the fourth-generation
model is shown with a red dashed curve with a band indicating theoretical
uncertainties.  Left: low-mass range, right: high-mass range.  }
\end{center}
\end{figure}

A search for the SM Higgs boson decaying into two Z boson which
subsequently decay into to quark jets and two leptons has been presented.
The data used in this analysis constitute $4.6fb^{-1}$ of integrate
luminosity.  No significant excess of events was found and upper limits
have been measure in the context of the SM and SM4 models.  Although 
only a small portion of Higgs masses have been excluded at 95\% CL, 
more data should allow sensitivity which is sufficient for excluding almost
the entire range between 200 and 600 GeV with this channel alone.  
This decay topology has proven
to be one of the most improtant for exclusion of high mass Higgs models and
has played a major role in the CMS grand combination~\cite{???}. 
