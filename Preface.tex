%% FRONTMATTER
\begin{frontmatter}

% generate title
\maketitle

\begin{abstract}
We present two searches for massive resonances decaying with a single top-quark signature.  
First, we present a search for the W' boson decaying to a top and bottom quark, then a search for the singly produced $\bs$ 
quark decaying to a top quark and W boson.
The data analysed for these searches corresponds to an integrated luminosity of 19.7~fb$^{-1}$ collected by the CMS detector in proton-proton collisions at $\sqrt{s}=8$~TeV.
The use of cutting edge jet substructure algorithms allows the top quark jet to be distinguished from standard model hadronic jet backgrounds and is the central feature of these analyses.

We set 95\% C.L. limits on the production cross section of a right-handed W' boson, 
together with constraints on the left and right-handed couplings of the W' boson to quarks. The production of a right-handed W' boson with a mass below 
2.02 TeV decaying to an all-hadronic final state is excluded. This mass limit increases to 2.15 TeV when both all-hadronic 
and semileptonic decays are considered.

Additionally, limits on the production cross section of the right-handed, left-handed, and vectorlike $\bs$ quark boson are obtained.  
The masses of the left-handed, right-handed and vectorlike $\bs$-quark states are excluded in the range of 880-1390 GeV, 820-1430 GeV, and 800-1530 GeV respectively when considering the all-hadronic channel only.  
The masses of the left-handed, right-handed and vectorlike $\bs$-quark states are excluded below 1390, 1420 and 1520 GeV when considering the combined all-hadronic, semileptonic, and dileptonic channels.  

\vspace{.3cm}
\noindent Primary Reader: Petar Maksimovic\\
Secondary Reader:

\end{abstract}

\begin{acknowledgment}
I would first like to thank my parents for their constant encouragement and love.  
Thanks to my Mom for getting me interested in physics by discussing the mysteries of the universe, and to my Dad who taught me not to hate math at a young age.  
When something went wrong they were always there to put it in perspective, and when something went well they were always there to congratulate me.  

I would like to thank my advisor, Petar Maksimovic, for his guidance and optimism.  
He taught me both how to perform a physics analysis, and (possibly more important) how to deal with people in such a large collaboration.   
I thank him for teaching me that the better is the enemy of the good, for telling me to calm down when things look bleak, and for taking analysis ending problems and turning them into 30 minute cross-checks.  

There are many to thank for my education in the specifics of experimental high energy physics analyses, but probably none more than Salvatore Rappoccio.  
Sal was there from my first attempts at getting things to work and had the patience to answer all of my basic questions.  
Many thanks to Guofan Hu, Gavril Giurgiu, Justin Pilot, and James Dolen for helping me with specific analysis level questions. 

I would like to thank my friends and colleagues in the physics department.  
Thanks to Marc Osherson for putting up with me in the same office for four years.  
Thanks for discussing silly physics ideas with me, and reminding me to not take everything so seriously.  
Thanks to Dave Fehling, Ian Anderson, Chris Martin, Yongjie Xin, Matt Morris, Keith Redwine, Justin Bankert, Kevin Grizzard, 
and everyone else in the department for the lunches, parties, happy hours, and just being there to talk to.  

I would like to thank the JHU physics staff for making sure my exams were scheduled on time and my forms were in order.  
Thanks for making sure I did not encounter any problems during my time here.  

Without the hard work of everyone contributing to the construction and maintenance of the LHC and the CMS detector none of this would be possible.  
Also, thanks to everyone at Fermilab for making sure the LPC Cluster was working when approval was looming over my head.  

Last but not least  I would like to thank Grace for preserving my happiness regardless of what else was going on.  
Thank you for putting up with the QFT all-nighters, the late night Higgs discovery announcement, the early morning meetings, and all of my stress and worry.  

\end{acknowledgment}

%\begin{dedication} 

%\end{dedication}

% generate table of contents
\tableofcontents

% generate list of tables
\listoftables

% generate list of figures
\listoffigures

\end{frontmatter}

