\chapter{Higgs Phenomenology at the LHC}
\label{sec:HiggsPhen}
\chaptermark{Higgs Phenomenology at the LHC}

In the simplest incarnation of the Higgs mechanism, the Higgs boson
mass is the only free parameter.  Given the mass of the Higgs 
boson, the production cross section, branching fractions, and 
decay width can be calculated.
Generally, the Higgs boson 
couples most strongly to the most massive particles in the 
SM.  However, the mechanism for which the weak gauge 
bosons acquire mass and the fermions acquire mass in the SM is 
different.  
Thus, the coupling of the Higgs boson to fermions is proportional 
to the mass of the fermion while the coupling of the 
Higgs boson to the weak gauge bosons is proportional to the square 
of the gauge boson's mass. These features and the structure 
functions of the proton combine to produce the predictions shown in 
Figure~\ref{fig:HiggsProdXS} for the production cross-section and 
branching fraction of the Higgs.  

\begin{figure}
\begin{center}
\includegraphics[width=.56\linewidth]{HiggsPhenPlots/Higgs_XS_8TeV.eps}
\includegraphics[width=.42\linewidth]{HiggsPhenPlots/Higgs_BR.eps}
\caption{left: Higgs production cross section vs $m_H$ for 
different processes at $\sqrt{s}=8~TeV$. right: Higgs branching
ratios vs $m_H$.  Both calculations are taken from the LHC Higgs
cross section working group. }
\label{fig:HiggsProdXS}
\end{center}
\end{figure}

In this chapter, the terminology of the different production and 
decay channels are introduced as well as the experimental 
signatures for each.  Kinematics of spin-0, spin-1, 
and spin-2 resonances decaying to two vector bosons are introduced.  
Techniques for using decay kinematics for increasing signal 
sensitivity and performing property measurements are presented.  

\section{Higgs Signatures}

\subsection{Gluon-gluon Fusion}
\label{sec:ggHiggs}

The gluon-gluon fusion production mechanism is responsible for 
$\sim87\%$ of Higgs events produced at 
the LHC, assuming $m_H=125$~GeV and $\sqrt{s}=8$~TeV.  This is due 
to the gluon-gluon cross 
section dominating over other initial states for the relevant range 
of invariant masses, as shown in Figure~\ref{fig:LHCpdfs}.
However, because the Higgs cannot couple to gluons directly, the
interaction must be mediated through a loop, show in 
Figure~\ref{fig:ggFusion}.  The dominant contributions 
come from the heavy quarks, top and bottom quarks, which couple
strongly to both gluons and the Higgs.  The production cross 
section for this process varies from $3\times10^{-2}~pb$ to $40~pb$
for Higgs masses between 80 and 1000 GeV and $\sqrt{s}=8~TeV$.

\begin{figure}
\begin{center}
\includegraphics[trim=0 0 0 420,clip,width=.7\linewidth]{HiggsPhenPlots/pdf_factor.eps}
\caption{Distribution of parton factor, F(s,Y=0), showing the 
relative probability for producing resonances from gluon-gluon, or 
$q\bar{q}$ interaction.}
\label{fig:LHCpdfs}
\end{center}
\end{figure}

\begin{figure}
\begin{center}
\unitlength=1mm
\begin{fmffile}{ggFusion}

\begin{fmfgraph*}(40,40) \fmfpen{thick}
  \fmfleft{i1,i2} \fmfright{o1}
  \fmf{gluon,label=$g$}{i1,v1} \fmf{gluon,label=$g$}{i2,v2}
  \fmf{fermion,label=$t$}{v1,v2}
  \fmf{fermion,label=$~\bar{t}$}{v3,v1}
  \fmf{fermion,label=$~~~t$}{v2,v3}
  \fmf{dashes,label=$H$}{v3,o1}
\end{fmfgraph*}

\end{fmffile}
\end{center}
\caption{Feynman diagram depicting the leading contribution to 
gluon-gluon fusion production of a Higgs boson.}
\label{fig:ggFusion}
\end{figure}

\subsection{Weak Vector Boson Fusion}
\label{sec:VBFHiggs}

The Weak Vector Boson Fusion (VBF) production mechanism has the next to
largest cross section at the LHC, depicted in Figure~\ref{fig:VBF}.  The 
signature of this production mechanism is two energetic jets at high values 
of pseudorapidity.  Because of gluon radiation from next-to-leading
order (NLO) and next-to-NLO (NNLO) QCD effects, gluon-gluon fusion events can
also have this same signature.  As such, event classes which attempt to 
distinguish the VBF production mechanism tend to have a large contamination 
from gluon-gluon fusion.  Usually the kinematics of the spectator jets
can be used to further isolate VBF-like events. 

\begin{figure}
\begin{center}
\unitlength=1mm
\begin{fmffile}{VBF}

\begin{fmfgraph*}(40,30) \fmfpen{thick}
  \fmfleft{i1,i2} \fmfright{sp1,H,sp2}
  \fmf{fermion,label=$q$}{i1,v1,sp1} 
  \fmf{fermion,label=$q$}{i2,v2,sp2}
  \fmf{photon,label=$V$}{v1,v3}
  \fmf{photon,label=$V$}{v2,v3}
  \fmf{dashes,label=$H$}{v3,H}
\end{fmfgraph*}

\end{fmffile}
\end{center}
\caption{Feynman diagram depicting weak vector boson fusion production
of a Higgs boson.}
\label{fig:VBF}
\end{figure}

\subsection{Other Production Mechanisms}
\label{sec:VHiggs}

Other production mechanisms produce Higgs bosons in association with either
a weak gauge boson or top pair, both of which are depicted in 
Figure~\ref{fig:VHttH}.  In these cases either W or Z can be tagged or the 
presence of 
b-jets can be included.  However, for $m_H=125$~GeV, these processes only make up ~5\% of the 
total Higgs boson production cross section at the LHC.  As such, having 
significant 
sensitivity to these production mechanisms requires very high amount of 
integrated 
luminosity, $\mathscr{O}(100~fb^{-1})$. 

\begin{figure}
\begin{center}
\unitlength=1mm
\begin{fmffile}{VH_and_ttH}

\begin{fmfgraph*}(40,30) \fmfpen{thick}
  \fmfleft{i1,i2} \fmfright{V,H}
  \fmf{fermion,label=$\bar{q}$}{v1,i1}
  \fmf{fermion,label=$q$}{i2,v1}
  \fmf{photon,label=$V^{*}$}{v1,v2}
  \fmf{photon,label=$V$}{v2,V}
  \fmf{dashes,label=$H$}{v2,H}
\end{fmfgraph*}
~
\begin{fmfgraph*}(50,30) \fmfpen{thick}
  \fmfleft{i1,i2} \fmfright{sp1,H,sp2}
  \fmf{photon,label=$g$}{i1,v1}
  \fmf{photon,label=$g$}{i2,v2}
  \fmf{fermion,label=$t$}{v1,sp1}
  \fmf{fermion,label=$\bar{t}$}{v3,v1}
  \fmf{fermion,label=$\bar{t}$}{sp2,v2}
  \fmf{fermion,label=$t$}{v2,v3}
  \fmf{dashes,label=$H$}{v3,H}
\end{fmfgraph*}

\end{fmffile}
\end{center}
\caption{Feynman diagram depicting associated production (left) and $t\bar{t}$ 
fusion production of a Higgs boson.}
\label{fig:VHttH}
\end{figure}

\subsection{Decay Channels}
\label{sec:HiggsDecays}

The partial decay widths of the Higgs boson, just as with productions, are typically related to the mass of the decay products.  As such, at low 
mass, where the production of weak gauge bosons is suppressed from 
phase-space effects, b-quarks are the dominant decay, making up $\sim80\%$ of 
the events.  At high mass, the leading decays are to W and Z pairs.  
The SM has the particular feature that the $H\to\gamma\gamma$ and 
$H\to Z\gamma$ branching ratios are much smaller than the $H\to ZZ$ or 
$H\to WW$ branching ratios
because the Higgs does not couple directly to massless particles.  Thus, these
processes are required to proceed through loops which would contain massive
particles, usually either top quarks or W bosons.  This is one of the 
most distinguishing features which results in a large suppression of 
the $\gamma\gamma$ and $Z\gamma$ channels with respect to the ZZ and WW
channels.  The branching ratios versus $m_H$ are shown in 
Figure~\ref{fig:HiggsProdXS} for different decay channels. 

Because of the distinct signature of ZZ,
WW, and $\gamma\gamma$ decays, these channels are the most sensitive for 
discovering a Higgs-like resonance.  The $4\ell$ final state of the ZZ 
channel is especially promising because it is a high resolution, fully 
reconstructable channel with very small SM backgrounds.


\section{Kinematics of Scalar Resonances}
\label{sec:Kinematics of scalar resonances}

The simplest incarnation of the Higgs mechanism predicts one scalar 
boson with the simplest coupling to the SM fields.  However, there 
are models which go beyond the minimal Higgs mechanism and predict 
other scalars which would couple differently to the SM fields.  
The most generic amplitude for a scalar which couples to two 
bosons is
\begin{equation}
\begin{split}
  \mathscr{A}(X\to VV) = v^{-1}(g_1m_v^2\epsilon_1^*\epsilon_2^*+g_2f_{\mu\nu}^{*(1)}f^{*(2),\mu\nu}+ \\
g_3f^{*(1),\mu\nu}f_{\mu\alpha}^{*(2)}\frac{q_\nu q^\alpha}{\Lambda^2}+g_4f_{\mu\nu}^{*(1)}\tilde{f^{*(2),\mu\nu}}),
\label{eq:scalarAmp}
\end{split}
\end{equation}
where $f$ and $\tilde{f}$ are the field strength tensor and the 
conjugate field strength tensor, $g_i$ are dimensionless couplings,
$\epsilon_i$ are the polarization vectors of the vector bosons, 
$\Lambda$ denotes the scale where new physics could appear, $m_V$ 
is the mass of the vector boson, and q is the momentum of the 
VV-system.  This amplitude corresponds to three independent
Lorentz structures and can be rewritten as,
\begin{equation}
\mathscr{A}(X\to VV) = v^{-1}\epsilon_1^{*\mu}\epsilon_2^{*\nu}(a_1g_{\mu\nu}m_X^2+a_2q_\mu q_\nu+a_3\epsilon_{\mu\nu\alpha\beta}q_1^{\alpha}q_2^{\beta}). 
\label{eq:masterEq}
\end{equation}
The translation between the couplings used in Equation~\ref{eq:scalarAmp} and those used in Equation~\ref{eq:masterEq} can be found in Equation 12 of Reference~\cite{Bolognesi:2012mm}.  The SM Higgs boson couples to the weak vector boson only through the $a_1$ term and
couples to photons through an effective coupling which is a 
combination of the $a_1$ and $a_2$ terms.  A CP-odd scalar, 
commonly referred to as a pseudoscalar, couples to the gauge 
bosons through the $a_3$ term.

The amplitude can be broken into several more specific amplitudes, 
known as helicity amplitudes, corresponding to the helicity states 
of the vector bosons, where the quantization axis is taken to be 
the direction of the VV decay in the resonance's rest frame.  For 
a scalar resonance, there are only three non-zero helicity 
amplitudes out of the nine permutations,
\begin{center}
\begin{subequations}
  \begin{equation}
    A_{00} = -\frac{m_X^2}{v}\left(a_1\sqrt{1+x}+a_2\frac{m_1m_2}{m_X^2}x\right),    \end{equation}
  \begin{equation}
    A_{++} = \frac{m_X^2}{v}\left(a_1+ia_3\frac{m_1m_2}{m_X^2}\sqrt{x}\right),
    \end{equation}
  \begin{equation}
    A_{--} = \frac{m_X^2}{v}\left(a_1-ia_3\frac{m_1m_2}{m_X^2}\sqrt{x}\right),
    \end{equation}
\end{subequations}
\end{center}
where $x$ is defined as
\begin{equation}
x=(\frac{m_X^2-m_1^2-m_2^2}{2m_1m_2})^2-1.
\end{equation}

While the above formulas apply to all bosonic decays of 
scalar resonances, $ZZ\to4\ell$ decays are particularly well 
suited for performing property measurements.  This final 
state has very good momentum and angular resolution, low
SM backgrounds, and sufficient complexity for all features
of the most generic amplitude to be manifested.

A convenient basis of variables which can be
used to fully describe $ZZ\to4\ell$ decays in the ZZ rest frame 
consists of the three invariant masses ($m_{ZZ}$, $m_Z$, and 
$m_Z^*$) and 5 angles, depicted in Figure~\ref{fig:HZZdiagram}.
Each helicity amplitude has a distinct angular distribution while
the magnitude of each helicity amplitude depends on the invariant
masses of the two Z bosons and the resonance.  Together these 
combine into the differential cross section according to
\begin{equation}
\begin{split}
\mathscr{P}(m_1,m_2,\vec{\Omega})\propto|P_V(m_1,m_2)|  \\
\times\frac{m_1^3}{(m_1^2-m_V^2)^2+m_V^2\Gamma_V^2}\times\frac{m_2^3}{(m_2^2-m_V^2)+m_V^2\Gamma_V^2} \\
\times\frac{d\Gamma_J(m_1,m_2,\vec{\Omega})}{d\vec{\Omega}},
\end{split}
\end{equation}
where q is the magnitude of the vector boson momentum in the resonance's rest-frame.
For a spin-0 resonance, the angular distributions are given 
by
\begin{equation}
\begin{split}
\frac{d\Gamma_{J=0}}{\Gamma d\vec{\Omega}} = 4|A_{00}|^2\sin^2\theta_1\sin^2\theta_2 \\
+|A_{++}|^2(1-2A_{f1}\cos\theta_1+\cos^2\theta_1)(1+2A_{f2}\cos\theta_2+\cos^2\theta_2)\\
+|A_{--}|^2(1+2A_{f1}\cos\theta_1+\cos^2\theta_1)(1-2A_{f2}\cos\theta_2+\cos^2\theta_2)  \\
+4|A_{00}||A_{++}|(A_{f1}+\cos\theta_1)\sin\theta_1(A_{f2}+\cos\theta_2)\sin\theta_2\cos(\Phi+\phi_{++}) \\
+4|A_{00}||A_{--}|(A_{f1}-\cos\theta_1)\sin\theta_1(A_{f2}-\cos\theta_2)\sin\theta_2\cos(\Phi-\phi_{--}) \\
+2|A_{++}||A_{--}|\sin^2\theta_1\sin^2\theta_2\cos(2\Phi-\phi_{--}+\phi_{++})
\end{split}
\label{eq:angularDist}
\end{equation}
where $A_{fi}$ are the $Z\to f\bar{f}$ amplitudes which can be
found in Reference~\cite{Bolognesi:2012mm}.
The resulting differential cross section is parameterized in terms of 
the underlying couplings. 
The angular and mass distributions for several types of scalar 
models are shown in Figures~\ref{fig:ScalarMasses} 
and~\ref{fig:ScalarHelicityAngles}.  The red and blue distributions
correspond to a SM Higgs and pseudoscalar resonances.  The green 
distributions correspond to a scalar model in which the resonance couples
to the vector boson only through the $g_2$ term of 
Equation~\ref{eq:scalarAmp}, referred to here as
the $0_{h}^+$ model.  Thus, these three models 
represent the three independent Lorentz structures of the most generic
scalar-vector-vector amplitude.  

\begin{figure}
\begin{center}
\includegraphics[width=.49\linewidth]{FutureMeasurementsPlots/angles-HZZ4l_snowmass.eps}
\caption{Diagram depicting $H\to ZZ\to4\ell$ decays and definition
of angles which describe the kinematics of these decays.}
\label{fig:HZZdiagram}
\end{center}
\end{figure}

\begin{figure}
\begin{center}
\includegraphics[width=.32\linewidth]{HiggsPhenPlots/spinParityPaper/plots/z1mass_125GeV_spin0_3in1.eps}
\includegraphics[width=.32\linewidth]{HiggsPhenPlots/spinParityPaper/plots/z2mass_125GeV_spin0_3in1.eps}
\caption{Distributions of the Z boson masses.  The smaller of the two masses is
plotted on the right, while the larger of the two masses is plotted on the
left. Markers show simulation of events using JHUGen; lines are projections
of the analytical distribution described above.  Red lines/circles correspond
to a SM Higgs, blue lines/diamonds, a pseudoscalar, and green lines/square, 
a CP-even scalar produced from higher dimension operators.}
\label{fig:ScalarMasses}
\end{center}
\end{figure}

\begin{figure}
\begin{center}
\includegraphics[width=.32\linewidth]{HiggsPhenPlots/spinParityPaper/plots/costheta1_125GeV_spin0_3in1.eps}
\includegraphics[width=.32\linewidth]{HiggsPhenPlots/spinParityPaper/plots/costheta2_125GeV_spin0_3in1.eps}
\includegraphics[width=.32\linewidth]{HiggsPhenPlots/spinParityPaper/plots/phi_125GeV_spin0_3in1.eps}
\caption{Distributions of helicity angles, $\cos\theta_1$ (left), 
$\cos\theta_2$ (middle), and $\Phi$ (right). Markers show simulation of 
events using JHUGen; lines are projections
of the analytical distribution described above.  Red lines/circles correspond
to a SM Higgs, blue lines/diamonds, a pseudoscalar, and green lines/square, 
a CP-even scalar produced from higher dimension operators.}
\label{fig:ScalarHelicityAngles}
\end{center}
\end{figure}

In principle,
a mixture of these terms can occur.  In fact, there is a small but 
negligible contribution from the $g_2$ term in the SM from 
higher order electroweak corrections.  In various extensions to the SM, e.g. 
2 Higgs doublet models, multiple scalars exist with different CP 
properties.  It is even possible that CP-violating interactions 
could exist.  
%The distributions in Figures~\ref{fig:ScalarMasses} 
%and~\ref{fig:ScalarHelicityAngles} clearly demonstrate that with 
%enough events, the angular and mass distributions are sufficient to 
%determine the couplings of an observed scalar resonance.  
Constraining the contribution
from either the $g_2$ or $g_4$ term of the amplitude can be more 
aptly 
formulated through a reparametrization of the HZZ amplitude.  
Starting from the three complex couplings, $g_1$, $g_2$, and $g_4$,
four real parameters can be defined
\begin{center}
\begin{subequations}
  \begin{equation}
    f_i = \frac{|g_i|^2\sigma_i }{|g_1|^2\sigma_1+|g_2|^2\sigma_2+|g_4|\sigma_4}
    \end{equation}
  \begin{equation}
    \phi_{gi} = arg(\frac{g_i}{g_1}),
    \end{equation}
\label{eq:HZZmodelParams}
\end{subequations}
\end{center}
for $i=2,~4$. 
In the above formula, $\sigma_i$ is the cross section of the 
process corresponding to $g_i=1$ and $g_{\neq i}=0$.  The $f_{gi}$
parameters represent an effective fraction of events
resulting from the corresponding term of the amplitude.  In the 
case where there is no interference, this interpretation is exact.
This parametrization factorizes out the total cross section,
assuming that it will be measured separately. These variables are 
also straight forward measurables for experiments where rates are 
directly measured, as will be discussed in later sections.  In 
Chapters~\ref{sec:HZZsearches}, a slightly different notation will
be used for the fractions and the translation, $f_{a3}=f_{g4}$ and 
$f_{a2}=f_{g2}$ should be applied.

Similar differential cross sections can be calculated for a generic spin-1 or spin-2 
resonance decaying to two Z bosons~\cite{Bolognesi:2012mm}.  
Figures~\ref{fig:VectorMasses},~\ref{fig:VectorProdAngles}, 
and~\ref{fig:VectorHelicityAngles} show two choice vector 
resonance models.  
Figures~\ref{fig:TensorMasses},~\ref{fig:TensorProdAngles}, 
and~\ref{fig:TensorHelicityAngles}
show three choice tensor 
resonance models.  The couplings used to define each of these 
models are shown in Table~\ref{table:alternativeModels}.  
%Similar 
%to the case of a scalar resonance, sufficient information is 
%contained in the angular and mass distributions to 
%constrain all the parameters of the vector-vector-vector amplitude. 

\begin{figure}
\begin{center}
\includegraphics[width=.32\linewidth]{HiggsPhenPlots/spinParityPaper/plots/z1mass_125GeV_spin1_2in1.eps}
\includegraphics[width=.32\linewidth]{HiggsPhenPlots/spinParityPaper/plots/z2mass_125GeV_spin1_2in1.eps}
\caption{Distributions of the Z boson masses.  The smaller of the two masses is
plotted on the right, while the larger of the two masses is plotted on the
left. Markers show simulation of events using JHUGen; lines are projections
of the analytical distribution described above.  Red lines/circles correspond
to a CP-even vector, blue lines/diamonds to a CP-odd vector.}
\label{fig:VectorMasses}
\end{center}
\end{figure}

\begin{figure}
\begin{center}
\includegraphics[width=.32\linewidth]{HiggsPhenPlots/spinParityPaper/plots/costhetastar_125GeV_spin1_2in1.eps}
\includegraphics[width=.32\linewidth]{HiggsPhenPlots/spinParityPaper/plots/phistar1_125GeV_spin1_2in1.eps}
\caption{Distributions of the production angles, $\cos\theta^*$ (left) and 
$\Phi_1$ (right).  Markers show simulation of events using JHUGen; lines 
are projections
of the analytical distribution described above.   Red lines/circles correspond
to CP-even vector, blue lines/diamonds to a CP-odd vector.}
\label{fig:VectorProdAngles}
\end{center}
\end{figure}

\begin{figure}
\begin{center}
\includegraphics[width=.32\linewidth]{HiggsPhenPlots/spinParityPaper/plots/costheta1_125GeV_spin1_2in1.eps}
\includegraphics[width=.32\linewidth]{HiggsPhenPlots/spinParityPaper/plots/costheta2_125GeV_spin1_2in1.eps}
\includegraphics[width=.32\linewidth]{HiggsPhenPlots/spinParityPaper/plots/phi_125GeV_spin1_2in1.eps}
\caption{Distributions of the helicity angles, $\cos\theta_1$ (left), 
$\cos\theta_2$ (middle), and $\Phi$ (right). Markers show simulation of 
events using JHUGen; lines are projections
of the analytical distribution described above.  Red lines/circles correspond
to CP-even vector, blue lines/diamonds to a CP-odd vector.}
\label{fig:VectorHelicityAngles}
\end{center}
\end{figure}

\begin{figure}
\begin{center}
\includegraphics[width=.32\linewidth]{HiggsPhenPlots/spinParityPaper/plots/z1mass_125GeV_spin2_3in1.eps}
\includegraphics[width=.32\linewidth]{HiggsPhenPlots/spinParityPaper/plots/z2mass_125GeV_spin2_3in1.eps}
\caption{Distributions of the Z boson masses.  The smaller of the two masses is
plotted on the right, while the larger of the two masses is plotted on the
left. Markers show simulation of events using JHUGen; lines are projections
of the analytical distribution described above.  Red lines/circles correspond
to a minimal coupling graviton, blue lines/diamonds to a CP-odd tensor, 
and green lines/square to
a CP-even tensor produced from higher dimension operators.}
\label{fig:TensorMasses}
\end{center}
\end{figure}

\begin{figure}
\begin{center}
\includegraphics[width=.32\linewidth]{HiggsPhenPlots/spinParityPaper/plots/costhetastar_125GeV_spin2_3in1.eps}
\includegraphics[width=.32\linewidth]{HiggsPhenPlots/spinParityPaper/plots/phistar1_125GeV_spin2_3in1.eps}
\caption{Distributions of the production angles, $\cos\theta^*$ (left) and
$\Phi_1$ (right). Markers show simulation of events using JHUGen; lines
are projections
of the analytical distribution described above.  Red lines/circles correspond
to a minimal coupling graviton, blue lines/diamonds to a CP-odd tensor, 
and green lines/square to a CP-even tensor produced from higher dimension operators.}
\label{fig:TensorProdAngles}
\end{center}
\end{figure}

\begin{figure}
\begin{center}
\includegraphics[width=.32\linewidth]{HiggsPhenPlots/spinParityPaper/plots/costheta1_125GeV_spin2_3in1.eps}
\includegraphics[width=.32\linewidth]{HiggsPhenPlots/spinParityPaper/plots/costheta2_125GeV_spin2_3in1.eps}
\includegraphics[width=.32\linewidth]{HiggsPhenPlots/spinParityPaper/plots/phi_125GeV_spin2_3in1.eps}
\caption{Distributions of the helicity angles, $\cos\theta_1$ (left), 
$\cos\theta_2$ (middle), and $\Phi$ (right). Markers show simulation of 
events using JHUGen; lines are projections
of the analytical distribution described above.  Red lines/circles correspond
to a minimal coupling graviton, blue lines/diamonds to a CP-odd tensor, 
and green lines/square to a CP-even tensor produced from higher dimension operators.}
\label{fig:TensorHelicityAngles}
\end{center}
\end{figure}

\begin{table}
\begin{center}
\begin{tabular}{cccc}
\hline 
\hline
scenario & X prod & $X\to VV$ decay & comments \\
\hline
$0_m^+$ & $gg\to X$ & $g_1\neq 0$ & SM Higgs boson \\ 
$0_h^+$ & $gg\to X$ & $g_2\neq 0$ & scalar with higher-dim operators \\ 
$0^-$ & $gg\to X$ & $g_4\neq 0$ & pseudoscalar \\ 
$1^+$ & $q\bar{q}\to X$ & $b_2\neq 0$ & exotic pseudovector\\
$1^-$ & $q\bar{q}\to X$ & $b_1\neq 0$ & exotic vector \\ 
$2_m^+$ & $g_1^{(2)}=g_5^{(2)}\neq 0$ & $g_1^{(2)}=g_5^{(2)}\neq 0$ & tensor with min couplings\\
$2_b^+$ & $g_1^{(2)}=\neq 0$ & $g_5^{(2)}\neq 0$ & bulk tensor with min couplings \\
$2_h^+$ & $g_4^{(2)}\neq 0$ & $g_4^{(2)}\neq 0$ & tensor with higher-dim operators\\
$2_h^-$ & $g_8^{(2)}\neq 0$ & $g_8^{(2)}\neq 0$ & ``pseudotensor''\\ 
\hline
\hline
\end{tabular}
\end{center}
\caption{List of alternative signal models to be tested against the SM Higgs 
hypothesis along with a description of the their couplings to ZZ.  Amplitude
parametrization for spin-0 resonances is given in Equation~\ref{eq:scalarAmp};
parametrizations for spin-1 and spin-2 resonances are given in Equations 16 and 18 elsewhere~\cite{Bolognesi:2012mm}.}
\label{table:alternativeModels}
\end{table}

\subsection{Variables for Property Measurements}
\label{sec:Spin-parity}

Several extensions
to the SM discussed previously in Chapter~\ref{sec:intro}, can result 
in ZZ resonances.  
Consequently, understanding the spin and CP of any new resonance discovered 
at the LHC will be critical to understanding its role in nature.  
%Since measuring model parameters requires a description of all 
%backgrounds, acceptance effects, and resolution in 8 dimensions, 
An efficient way of constraining resonance properties is to use 
compact variables to isolate specific properties.  Such a variable 
can be built from either the square of the matrix element for two 
processes, or equivalently, the differential cross section defined 
above, according to 
\begin{equation}
\mathscr{D}_{J^P} = \left(1+\frac{\mathscr{P}_{J^{P}}(m_1,m_2,\vec{\Omega}|m_{4\ell})}{\mathscr{P}_{0^+}(m_1,m_2,\vec{\Omega}|m_{4\ell})}\right)^{-1}
\label{eq:KD}
\end{equation}
where $\mathscr{P}_{J^P}$ and $\mathscr{P}0^+$ are evaluated using
the corresponding matrix elements.  These types of variables use
ideal distributions to isolate the relevant kinematic differences
between two choice models.  For $ZZ\to4\ell$ events these variables
will be close to optimal since acceptance effects will cancel when
calculating ratios and resolution effects are relatively small (see Section~\ref{sec:alignment}).  
In other channels, steps can be taken to mitigate the effects of 
resolution (see Section~\ref{sec:HZZ2l2q}).

An accurate description of the detector level distribution of 
$\mathscr{D}_{J^P}$ must be modeled.  Simulated Monte Carlo (MC)
events can be used, including all detector simulations, 
reconstruction algorithms, and analysis selections, to model the 
shape of these discriminants. Thus, MC simulations can effectively 
be used to model the appropriate 
transfer function for a given analysis.  The discriminant 
$\mathscr{D}_{J^P}$ can be used either as an additional selection 
variable, or for constructing likelihoods.
This process of building discriminants from kinematic distributions
using a matrix element calculation paired with MC simulations is 
known as the Matrix Element Likelihood Analysis (MELA).

Even with a relatively small number of signal events, the MELA 
technique can be used to perform hypothesis separation to rule 
out definite non-SM signals.  For example, the variable 
$\mathscr{D}_{0-}$ can be used to isolate the relevant properties 
that distinguish a SM Higgs from a purely CP-odd scalar.  The SM 
Higgs and pseudoscalar distribution of $\mathscr{D}_{0-}$ for ideal 
MC is shown in Figure~\ref{fig:fa3Comparison}.  The separation 
between these two models can be quantified using Neyman-Pearson 
hypothesis testing. 
In this way, the compatibility of data with respect to either the 
null hypothesis (always the SM Higgs hypothesis) or the alternative
hypothesis can be quantified.  Other models, such as spin-1 or 
spin-2 models, can be tested using variables analogous to 
$\mathscr{D}_{0-}$.
A list of models which will be used in Section~\ref{sec:HZZ4l} 
to perform such tests are listed in 
Table~\ref{table:alternativeModels} along with a description.  

\begin{figure}
\begin{center}
\includegraphics[width=.49\linewidth]{HiggsPhenPlots/D0minusComparison.eps}
\includegraphics[width=.49\linewidth]{HiggsPhenPlots/D0hplusProj.eps}
\caption{Distributions of $D_{0-}$ (left) and $D_{0_h^+}$ (right) for 
various scalar models.  Expected shape for a $0^+$ scalar is
shown in solid black.  Dashed black line represent two alternative
scalar models, $0^-$ and $0^+_h$ for the left and right plots,
respectively.  
Red lines represent a mixture of the SM Higgs and the alternative
models.  Blue lines represent the weighted average
of $0^+$ and either of the two alternative model shapes.}
\label{fig:fa3Comparison}
\end{center}
\end{figure}

Certain discriminants have properties which allow them to be efficiently
used to measure model parameters.  Assuming $f_{g2}=0$,
$f_{g4}$ can be measured directly using $\mathscr{D}_{0-}$.  
Figure~\ref{fig:fa3Comparison} shows this discriminant for
both the SM Higgs (solid black line), a pseudoscalar 
(dashed black line), and a mixed parity 
model corresponding to $f_{g4}=0.5$ (red line).  All of the mixed
parity samples can be described by a weighted sum of the SM Higgs
distribution and the 
pseudoscalar distribution (blue line), 
\begin{equation}
\begin{split}
\mathscr{P}(D_{0-}|f_{g4}) = |\mathscr{A}_{0^+}|^2 + |\mathscr{A}_{0-}|^2 + 2Re(\mathscr{A}_{0^+}^*\mathscr{A}_{0-}) \\ 
\simeq (1-f_{g4})\mathscr{P}_{0^+}(\mathscr{D}_{0-})+f_{g4}\mathscr{P}_{0-}(\mathscr{D}_{0-}).
\label{eq:fa3}
\end{split}
\end{equation}
$\mathscr{P}_{0^+}$ and $\mathscr{P}_{0-}$ represent the differential cross section of
the SM Higgs model and the pseudoscalar, respectively.
Thus, Equation~\ref{eq:fa3} explicitly neglects interference, but
Figure~\ref{fig:fa3Comparison} demonstrates that $\mathscr{D}_{0-}$ is insensitive to
the interference and the relative phase between $\mathscr{A}_{0^+}$ and
$\mathscr{A}_{0-}$.  
%A more explicit justification of this procedure can be
%realized by drawing toys from the simulation of the full matrix
%element and fitting for $f_{g4}$ with Equation~\ref{eq:fa3}.  
%Figure~\ref{fig:???} shows the result of fitting $f_{g4}$ for 
%several mixed parity models, corresponding 
%$f_{g4}=.05,.1,.5$; no bias is found in any of these toys studies.

In contrast, the $\mathscr{D}_{0h+}$ discriminant cannot be used 
measure $f_{g2}$.  Figure~\ref{fig:fa3Comparison} shows that the 
interference between the $g_1$ and $g_2$ terms cannot be neglected 
and depends strongly on the $\phi_{g2}$.  This implies that more 
advanced techniques which can fit for both the fraction and the 
phase simultaneously will be needed to constrain this parameter.

Similar variables can be constructed to help discriminate signal 
effects from SM background events,
\begin{equation}
\mathscr{D}^{kin}_{bkg} = \left(1+\frac{\mathscr{P}_{bkg}(m_1,m_2,\vec{\Omega}|m_{4\ell})}{\mathscr{P}_{sig}(m_1,m_2,\vec{\Omega}|m_{4\ell})}\right)^{-1}.
\label{eq:Dbkg}
\end{equation}
Analytical calculations for the continuum ZZ process are taken
from Reference~\cite{Gainer:2011xz,Chen:2012jy}.
Typically, invariant mass distributions are used in resonances 
searches.  As will be shown in Chapter~\ref{sec:HZZsearches}, 
variables similar to $D_{bkg}$ have proven to provide a significant 
increase in sensitivity to  Higgs-like events if used in 
conjunction with the relevant invariant mass distributions.  It 
should be noted that these variables are important
for properties as well; understanding properties of signal events 
first requires good sensitivity to signal events.

\section{Summary}

Understanding the role in electroweak symmetry breaking of any 
Higgs-like resonance can be divided into two classes of
measurements: measuring relative cross sections in various 
production and decay channels, and measuring kinematic distributions
within a given channel.  These sets of measurements provide
complementary information.  Kinematic distributions can be used
to build kinematic distributions to either perform hypothesis 
testing to constrain properties or to measure certain model
parameters.  Kinematic distributions will eventually
allow for measurements of the effective couplings between a 
resonance and the Z bosons. In addition, the tools presented above
can be used to maximize sensitivity to signal-like events.
Two implementations of these ideas will be presented in the 
following chapter. However, these tools are quite general and 
apply to other production and decay processes as well as other 
colliders, e.g. $e^+e^-\to Z*\to ZH$.  Chapter 5 will address 
the prospects of applying these tools to other processes.  
