\chapter{Introduction}
\label{sec:intro}
\chaptermark{Theoretical Motivation}
Modern particle physics is described by the standard model (SM).  This section will describe a basic outline of the theory as well as the known issues and need for a more basic theory.
The SM describes a universe in which matter is made up of particles of half-integer spin\footnote{Intrinsic angular momentum} called fermions.  
These fermions interact with each other through forces that manifest through the exchange of integer spin particles called bosons.  

\section{Fundamental Particles}
The SM matter in the universe is around 98\% Hydrogen and Helium with the final 2\% being heavier elements.  
To a very good approximation, the known matter in the universe is protons, neutrons, and electrons.  
Electrons are categorized in the standard model as leptons and are fundamental as far as we know.  
Protons and neutrons are not fundamental, but are made up of combinations of three up and down quarks.  
The up quark has +2/3 charge, and the down quark has -1/3 charge, so the proton is an up-up-down combination and the neutron is down-down-up.  
These combinations are called hadrons and are categorizes into combinations of three quarks called baryons, and two quarks called mesons.

Although this is a good approximation of the known universe, through experimental and theoretical advances we know that there are 
more exotic phenomena that can be described by extending the known quarks and leptons to three generations.  
The three lepton generations are defined by the electron, muon, tau and their corresponding neutrinos.
The +2/3 charge quarks are the up, charm, and top; whereas the -1/3 charge quarks are the down, strange, and bottom.  
These quarks and leptons are summarized in Table~\ref{REF} along with their intrinsic properties.
    
Quarks and leptons define all known fermionic matter, with bosonic particles being resposible for particle interactions.
    
\section{Fundamental Interactions}
Interactions in the SM can be described by the four fundamental forces electromagnetic, weak nuclear, strong nuclear, and gravity.  
These forces manifest by the exchange of a corresponding elementary boson.  
The intrinsic properties of these force carrying particles are responsible for the range and relative strength of the interaction.


The electromagnetic force is responsible well known phenomena such as molecular bonds.  
This force is mediated be the photon, a massless, chargeless, spin 1 particle.      
The fact that the photon is massless leads to the infinite range of the electromagnetic force.  
Particles with charge can interact through the electromagnetic force.


The weak nuclear force manifests in nuclear decay, and is described by three force carrying bosons; the W^+,W^-, and Z.  
These bosons are massive, which leads to the short range nature of the weak force.  
The W^{\pm} bosons have integer charge whereas the Z boson is chargeless, and all three have spin 1.  
The weak force is responsible for transitions between flavors\footnote{The six quark types} of quarks (see Section~\ref{REF}).
Quarks and leptons alike interact by the weak force.


The strong nuclear force is responsible for binding quarks together to form hadrons.  
The strong force describes the interactions of particles that carry color.  
Color is an intrinsic property of fundamental particles, and has three varieties; red, green and blue.
This force is mediated by gluons, which are massless and interact with quarks.  
The strong force is the strongest and shortest range of the known forces.  
The theory behind the strong force is described in more detail in Section~\ref{REF}.


The final force, gravity, is both the most recognizeable and least understood of the forces.  
All attempts at including gravity into the SM have failed, but hypotheically gravity should bed mediated by the spin 2 graviton.
Gravity is by far the weakest of the fundamental forces, and interacts with massive particles.

\section{Feynaman Diagrams}
Calculations in theoretical patricle physics are fully defined by the use of feynman diagrams.  
Feynman diagrams are pictoral representations of elementary physical processes.  
These diagrams include the particles that interact (external lines), as well as the particles that mediate the interaction (internal lines), 
and where these external and internal lines meet (vertices).
An example diagram is shown in Figure~\ref{FIG}.  
Here the horizontal axis is time, such that the diagram describes an electron (e) and a muon (\mu) entering and an electron an a muon leaving.  
The electron and muon exchange a photon (\gamma), and a force is observed on the external particles.
This diagram represents electromagnetic repulsion (coulomb force).  

The rules governing the calculation of physical observables from diagrams such as Figure~\ref{FIG} are defined by the theory of Quantum Electrodynamics (QED).
Using these diagrams, the rules of QED let us calculate relevant quantities such as the cross section of the process.  
The cross section can be thought of as the likelihood of a process to occur such that if all other variables are held constant 
a higher cross section means that the interaction will occur more frequently.  
Aditionally, in the case that a particle decays we can evaluate the decay width (\Gamma).  
When a particle of mass M decays the measured mass is not always M due to the uncertainty principle, and there is a range of values followdddding a Breit-Wigner function centered at the true mass.
The decay width represents the width of this distribution and is inversely related to the average lifetime of the particle.

For a given process there can be multiple contributing diagrams, for instance for a calculation involving the culomb repulsion shown in Figure~\ref{FIG}, one must also consider the 
digrams shown in Figure~\ref{FIG}.  In these diagrams we have the same incoming and outgoing particles.  
Diagrams such as this interfere with each other constructively or destructively, which increases or decreases the cross section of the full process.

Vertices in a Feynman diagram are points where energy and momentum are conserved in the calculation.  The number of vertices in a diagram is referred to as the order.  
The diagram in Figure\ref{FIG} is called leading order because it is impossible to construct a diagram with less vertices.  
Higher order diagrams such as those seen in Figure~\ref{FIG} are referred to as next-to leading order (NLO) and next-to-next-to leading order (NNLO).  
For a QED process, these vertices decrease the cross section by a the coupling constant \alpha (factor of 1/137) and thus including higher order 
diagrams does not greatly impact the result, and it is common to use just the leading order computation. 

\section{Quantum Chromodynamics}
Quantum chromodynamics (QCD) is the theory describing the strong force, and is of particular importance for this thesis.  
As mentioned in Section~\ref{SEC}, the strong force is mediated by gluons which interact with particles that carry color.  
In QED, the photon is not charged, and thus can not interact with itself, however in QCD the gluon carries color and thus can interact with itself.  
Additionally, wheras the addition of a vertex greatly reduces the cross section in QED, the strong coupliing constant \alpha_s is of order 1, so 
higher order diagrams can contribute substantially to the measurement.  This means that QCD processes are much more difficult to calculate than QED 
processes.

However, \alpha_s is not constant, and in fact increases as the distance scale of an interaction increases.  
This property of the the strong interaction is called asymptotic freedom.  
Therefore as the distance between two quarks increases, so do the forces holding them together.  
This large force at a characteristic distance ($\sim10^{-15}$ m) is why quarks are only observed as part of a hadron.    
This property is called quark confinement, and is the reason why it is difficult to observe a free quark.  
Although quarks can not be observed alone, there are ways of precisely determining the physical poddrperties of free quarks through reconstruction of their decay prooducts.  

When a hadron is produced at high energy, the quarks constituents will have a high momentum.  
This momentum will increase the separation of the quarks, and thus increase the energy between them.  
If the momentum is high enough, the energy beween the quarks reaches the threshold that quark anti-quark production is energetically favorable.  
The constituent quarks are then joined by these pair produced quarks forming two hadrons.  
This process can take place many times, and the initial hadron is detected as many hadrons that are highly collimated into a roughly conical shape called a jet.

The \alpha_s parameter is low at short distances (or equivalently high energy).  
Which means that just like QED, the impact of higher order diagrams is low.  
This allows the calculation of QCD diagrams using a finite perturbative expansion, and essentially lets us only consider free quarks in high energy QCD calculations. 
The characteristic interaction energy where free quarks can be considered is around 400\MeVcc, which is much lower than energy scales 
considered in this thesis, so we will only be referring to free quark interactions without specifying what hadron grouping is being analyzed.  

\section{The Weak Force}
Weak force interactions are dependent on the chirality of the interacting particle.
Chirality for massless particles is dependent on the relative orientation of the momentum and spin axes.  
Particles with momentum and spin aligned are referred to as right-handed, and particles with the momentum 
axis opposite to spin are left-handed\footnote{This convention is reversed for anti-particles}.
For massive particles this concept is generalized such that right- and left-handed components of a wavefunction 
can be extracted by using the right-handed operator (1+\gamma^5)/2 and the left-handed operator (1-\gamma^5)/2.  
The W boson only interacts with left-handed fermions whereas the Z boson interacts with right- and left-handed fermions with differing strengths. 

The weak force is responsible for changing of quark flavor in an interaction.  
The relative probabilities that a quark can change flavor are given by $|V_{ij}|^2$ where V_{ij} is an element of the Cabibbo Kobayashi Maskawa (CKM) matrix (shown in Figure\ref{FIG}). 

\section{Electroweak Symmetry Breaking}
Higgsy stuff

\section{Beyond the Standard Model}
The SM is possibly the most successful theory in physics, but also one that is ultimately incomplete.  
We know that there are physical phenomena that the SM does not predict.  
The presence of dark matter and dark energy in the universe~\cite{CITE} is not currently explained by the SM.  
Given that dark matter and energy account for approximately 95\% of the universe, this is not a small issue.  
The SM does not explain the observation of neutrino oscillation~\cite{CITE}, which implies that neutrinos are massive objects.  
The SM also does not naturally explain the relative values of seemingly random constants such as why the weak force is 10^33 times as strong as gravity.  
This issue is known as the hierarchy problem, and it is assumed that a complete theory would have a natural explanation for these constants.  

Therefore, it is essential to a complete understanding of the universe that we probe beyond the standard model (BSM) theories that provide solutions to these issues.  
One such theory is universal extra dimensions~\cite{CITE}, which provides a natural explanation to the hierarchy problem.  
In this theory, all forces are constrained within the known 4 dimensional universe (brane), whereas gravity propegates throughout the bulk of the higher dimensions as well.  
This makes the residual observed gravitational force on the brane low compared to the other forces.  
The propegation of SM massive fields in higher dimensions leads to discrete modes, which are detectible as new massive particles.   .
The propegation of the SM W or Z leads to exccited modes that are referred to as the \wp and \zp.
A novel way to look for BSM physics then is to attempt the creation and dectection of massive states such as these bosons. 
In this thesis we discuss one such search for a W' boson.  

The W' boson is a particle predicted by many BSM theories such as Little Higgs~\cite{CITE}, Compositeness models~\cite{CITE}, and Noncommuting Extended Technicolor~\cite{CITE}.  

Another signature of new physics is the discovery of a new heavy quark.    


 



