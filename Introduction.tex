\chapter{Introduction}
\label{sec:intro}
\chaptermark{Theoretical Motivation}
Modern particle physics is described by the standard model (SM).  This section will describe a basic outline of the theory as well as the known issues and need for a more basic theory.
The SM describes a universe in which matter is made up of particles of half-integer spin\footnote{Intrinsic angular momentum} called fermions.  
These fermions interact with each other through forces that manifest through the exchange of integer spin particles called bosons.  

\section{Fundamental Particles}
The SM matter in the universe is around 98\% Hydrogen and Helium with the final 2\% being heavier elements.  
To a very good approximation, the known matter in the universe is protons, neutrons, and electrons.  
Electrons are categorized in the standard model as leptons and are fundamental as far as we know.  
Protons and neutrons are not fundamental, but are made up of combinations of three up and down quarks.  
The up quark has +2/3 charge, and the down quark has -1/3 charge, so the proton is an up-up-down combination and the neutron is down-down-up.  
These combinations are called hadrons and are categorizes into combinations of three quarks called baryons, and two quarks called mesons.

Although this is a good approximation of the known universe, through experimental and theoretical advances we know that there are 
more exotic phenomena that can be described by extending the known quarks and leptons to three generations.  
The three lepton generations are defined by the electron, muon, tau and their corresponding neutrinos.
The +2/3 charge quarks are made up of the up, charm, and top; whereas the -1/3 charge quarks are made up of the down, strange, and bottom.  
These quarks and leptons are summarized in Table~\ref{REF} along with their intrinsic properties.
    d
Quarks and leptons define all known fermionic matter, with bosonic particles being resposible for particle interactions.
    
\section{Fundamental Interactions}
Interactions in the SM can be described by the four fundamental forces electromagnetic, weak nuclear, strong nuclear, and gravity.  
These forces manifest by the exchange of elementary particles.  
The intrinsic properties of these force carrying particles are responsible for the range and relative strength of the interaction.


The electromagnetic force is responsible well known phenomena such as molecular bonds.  
This force is mediated be the photon, a massless, chargeless, spin 1 particle.      
The fact that the photon is massless leads to the infinite range of the electromagnetic force.  
Particles with charge can interact through the electromagnetic force.


The weak force manifests in nuclear decay, and is described by three force carrying bosons; the W^+,W^-, and Z.  
These bosons are massive, which leads to the short range nature of the weak force.  
The W^{\pm} bosons have integer charge whereas the Z boson is chargeless, and all three have spin 1.  
The weak force is responsible for transitions between flavors\footnote{The six quark types} of quarks (see Section~\ref{REF}).
Quarks and leptons alike feel the weak force.


The strong force is responsible for binding quarks together to form hadrons.  
The strong force describes the interactions of particles that carry color.  
Color is an intrinsic property of fundamental particles, and has three varieties; red, green and blue.
This force is mediated by gluons, which are massless and interact with quarks.  
The strong force is the strongest and shortest range of the known forces.  
The theory behind the strong force is described in more detail in Section~\ref{REF}.


The final force, gravity, is both the most recognizeable and least understood of the forces.  
All attempts at including gravity into the SM have failed, but hypotheically gravity should be mediated by the spin 2 graviton.
Gravity is by far the weakest of the fundamental forces, and interacts with massive particles.

\section{Feynaman Diagrams}
Calculations in theoretical patricle physics are fully defined by the use of feynman diagrams.  
Feynman diagrams are pictoral representations of elementary physical processes.  
These diagrams include the particles that interact (external lines), as well as the particles that mediate the interaction (internal lines).
An example diagram is shown in Figure~\ref{FIG}.  
Here the horizontal axis is time, such that the diagram describes an electron (e) and a muon (\mu) entering and an electron an a muon leaving.  
The electron and muon exchange a photon (\gamma), and a force is observed on the external particles.
This diagram represents electromagnetic repulsion (coulomb force).  

The rules governing the calculation of physical observables from diagrams such as Figure~\ref{FIG} are defined by the theory of Quantum Electrodynamics (QED).
Using these diagrams, the rules of QED let us calculate relevant quantities such as the cross section of the process.  
The cross section can be thought of as the likelihood of a process to occur such that if all other variables are held constant 
a higher cross section means that the interaction will occur more frequently.  
Aditionally, in the case that a particle decays we can evaluate the decay width (\Gamma).  
When a particle of mass M decays the measured mass is not always M due to the uncertainty principle, and there is a range of values folloing a Breit-Wigner function centered at the true mass.
The decay width represents the width of this distribution and is inversely related to the average lifetime of the particle.

For a given process there can be multiple contributing diagrams, for instance for a calculation involving the culomb repulsion shown in Figure~\ref{FIG}, one must also consider the 
digrams shown in Figure~\ref{FIG}.  In these diagrams we have the same incoming and outgoing particles.  
Diagrams such as this interfere with each other constructively or destructively, which increases or decreases the cross section of the full process.

%%%%%%%%%%%%%%%
   
  

\section{Quantum Chromodynamics}
Quantum chromodynamics (QCD) is the theory describing the strong force, and is of particular importance for this thesis.



 . 

  .  
 



The Standard Model (SM) of particle physics  
is a mathematical description of the fundamental particles and 
their interactions.  Within the SM, particles are described by
quantized excitations of spin-0, spin-1/2, and spin-1 fields which
are solutions to the Klein-Gordon, Dirac, and Proca equations,
respectively.  These equations govern the time evolution of each
field.  
Other spin states can arise from bound states.  
%Although bound states and possible extensions to the SM can have
%higher spin, the SM doesn't explicitly describe the 
%interactions of these particles.  

The interactions of fields are encoded in the SM
Lagrangian.  For example, the electromagnetic interactions of
electrons are described by the Lagrangian
%\begin{center}
\begin{equation}
\mathscr{L}_{EM} = \bar{\psi}\left(i\gamma^{\mu}(\partial_{\mu}+ieA_{\mu})-m\right)\psi - \frac{1}{4}F_{\mu\nu}F^{\mu\nu}.
\label{eq:EMlagrangian}
\end{equation}
%\end{center}
The probability for some initial state evolving into some
final state can be be expanded in powers of the coupling
constant, $e$, according to the modulus squared matrix element
%\begin{center}
\begin{equation}
\mid<ee\mid e^{\bar{\psi}\left(i\gamma^{\mu}(\partial_{\mu}+ieA_{\mu})-m\right)\psi+\frac{1}{4}F_{\mu\nu}F^{\mu\nu}}\mid ee>\mid^2.
\label{eq:matrixElementSquared}
\end{equation}
%\end{center}
Often, the amplitude of a process at some order in 
the couplings is represented by a Feynman diagram such as the
one in Figure~\ref{fig:eeScattering} which represents
electron-electron scattering to lowest order in a purely 
electromagnetic theory.

\begin{figure}
\begin{center}
\unitlength=1mm
\begin{fmffile}{eeScattering}

\begin{fmfgraph*}(40,30) \fmfpen{thick}
  \fmfleft{i1,i2} \fmfright{sp1,sp2}
  \fmf{fermion,label=$e^-$}{i1,v1,sp1} 
  \fmf{fermion,label=$e^-$}{i2,v2,sp2}
  \fmf{photon,label=$\gamma$}{v1,v2}
\end{fmfgraph*}

\end{fmffile}
\end{center}
\caption{Feynman diagram depicting electron-electron scattering via
the electromagnetic interaction.}
\label{fig:eeScattering}
\end{figure}

The interactions of the SM are derived by enforcing local gauge 
symmetries and
thus can be described through a symmetry group.  For example, the 
electromagnetic interactions are known to be generated from a U(1) 
gauge symmetry.  Each symmetry has a corresponding charge which is
conserved and which the gauge mediators couple to. For example,
the photon couples to the electric charge, $e$.
Thus, specifying the gauge symmetries and the charges of particles
provides a clear description of particle interactions.
%Thus, understanding the gauge symmetries
%which generate the interactions of the SM and the 
%charges of
%each particle tells us how the particles interact.

Currently, the SM describes three of the four known forces: the
electromagnetic, the weak, and the strong force, which are
generated from U(1), SU(2), and SU(3) gauge symmetries,
respectively.  The charges of the fundamental
fields known to exist in the SM are shown in 
Table~\ref{table:SMcharges}.  
The photon only couples to electrically charged
particles (Q), the W boson couples to particles charged under weak 
isospin ($T_3$), 
and the gluons couple to colored particles. As the names 
suggest, at low energies, the strong force is the strongest
and the weak force is the weakest.  It is commonly believed 
that these interactions
should all be unified at some energy scale where the strength
will become comparable. 
%One unique consequence of this charge 
%structure is that the photon cannot couple to itself while the 
%weak vector boson and the gluons can.  The is also indicitive of
%the fact that the SU(2) and SU(3) gauge groups are non-abelian. 

\begin{table}
\begin{center}
\begin{tabular}{l|c|c|c|c}
\hline 
\hline
particle & Q  & $T_3$ & $Y_W$ & colored \\ \hline \hline
$e_L$, $\mu_L$, $\tau_L$  & -1 & -1/2   &  -1 &  no \\ 
$e_R$, $\mu_R$, $\tau_R$  & -1 & 0      &  -2 &  no \\ 
$\nu_L$  & 0   & 1/2 & -1 & no \\ 
$u_L$, $c_L$, $b_L$    & 2/3 & 1/2  & 1/3& yes \\ 
$u_R$, $c_R$, $b_R$    & 2/3 & 0  & 4/3& yes \\ 
$d_L$, $s_L$, $t_L$    & -1/3& -1/2 & 1/3& yes \\ 
$d_R$, $s_R$, $t_R$    & -1/3& 0 & -2/3& yes \\
%H        & 0   & 1/2  & -- & no \\
%W        & 1   & 1    & -- & no \\
%Z        & 0   & --   & -- & no \\
%$\gamma$ & 0   & --   & -- & no \\
%gluon    & 0   & 0    & 0  & yes \\
\hline
\end{tabular}
\end{center}
\caption{List of SM particles and their charges. 
Q represents the charge of the $SU(1)_{em}$ gauge symmetry,
$T_3$ the broken SU(2) gauge symmetry, and $Y_W$ the broken 
U(1) gauge symmetry.}
\label{table:SMcharges}
\end{table}

Naively, the idea of interactions arising from enforcing gauge
symmetries produces inconsistencies between theory and
experiments.  Even at the time when the SU(2) structure of the
weak interactions was first proposed by
Glashow~\cite{Glashow:1961}, the W boson was known to be massive.  
%One of the predictions of this gauge structure, the 
%existance of a neutral weak current, was later found to also 
%be massive.  
However, mass terms in a Lagrangian break gauge
invariance.  This internal inconsistency suggested that
the SU(2) gauge symmetry must be broken in a specific way in order
to allow the weak vector bosons to be massive, a process known
as electroweak symmetry breaking.  

\section{Electroweak Symmetry Breaking}
\label{sec:Electroweak Symmetry Breaking}

In 1963, Phil Anderson proposed that spontaneously
broken symmetries could provide a theoretical framework for
explaining massive gauge bosons in non-relativistic
systems\cite{Anderson:1963pc}. In 1964,
these ideas were studied in the context of relativistic quantum 
field theories.  It was shown that a complex 
scalar field whose potential was particularly chosen could 
spontaneously break a gauge symmetry and generate gauge boson 
masses through the interaction of this field with the gauge 
bosons\cite{Higgs:1964,Englert:1964,Higgs:1964-2,Guralnik:1964}.  
Most notably, 
Peter Higgs suggested that this would also predict the presence
of a new massive scalar particle~\cite{Higgs:1964-2}.

Glashow, Weinberg, and
Salam~\cite{Glashow:1961,Weinberg:1967,Salam:1968} 
showed that the Higgs mechanism could be used to break a
$SU(2)\times U(1)_{Y}$ symmetry to a $U(1)_{em}$ symmetry producing all
of the known electroweak interactions and massive 
weak gauge bosons. The Glashow-Weinberg-Salam (GWS) model predicted
a massive, neutral gauge boson, the Z boson,
whose mass would be around 90~GeV; this was confirmed
indirectly through electron-neutrino 
scattering~\cite{Hasert:1973,Hasert:1973-2,Hasert:1974}.  
The Z boson was later directly detected~\cite{Brianti:1983qi,Arnison:1983mk}.
Another experimental signature of the GWS model was that there
should exist a chargeless,
colorless, spinless, massive boson, similar to that suggested by Higgs; this particle is now commonly referred to as the Higgs boson.
Except for its mass all properties of this particle could be
calculated whithin the framework of the SM (see
Section~\ref{sec:HiggsPhen}).  

Electroweak symmetry breaking is the cornerstone of the SM model
and illuminating the exact mechanism by which it occurs is
paramount to our understanding of the universe.  Thus, the
experimental verification of the Higgs boson and its properties
has been the top priority of the field of particle physics for
nearly fifty years.

\section{Higgs Boson Constraints}
\label{sec:The Higgs boson}

Several accelerators have been built to discover the Higgs
boson,   
the first of which was the Large Electron-Positron (LEP) collider
which accelerated electrons and positrons to energies up to 209 GeV.
Although a broad range of Higgs boson masses 
were accessible to LEP experiments, no evidence was found and 
95\% confidence level exclusion limits were set for all masses up to 114.4~GeV~\cite{Barate:2003sz}.
However, high precision
measurements made on a number of SM 
quantities could be used to constrain the Higgs boson mass under the
assumption it were to exist according to the SM.  These constraints
suggested that a SM Higgs boson would be more likely in the range
$m_H\lesssim185$~GeV~\cite{Group:2008aa}.  

The Tevatron
%, built at Fermi National Accelerator Laboratory, 
and its experiments also contributed major efforts towards Higgs
searches.  As a 2~TeV $p\bar{p}$ collider, considerably larger 
masses were accessible compared to LEP.  However, no evidence of 
the 
Higgs boson  was found and 95\% confidence level exclusion limits
were set for Higgs boson mass between
$162<m_H<166$~GeV~\cite{Aaltonen:2010yv}.  
Despite the lack of a Higgs boson observation, the discovery of
the top quark and measurement of its mass helped
to refine calculations of the Higgs boson production cross section
and branching ratios which include contributions from virtual
top quarks.

By the time the LHC was delivering beams, theory
calculations had been refined and both direct limits and indirect
limits had been set by LEP and Tevatron experiments.
Figure~\ref{fig:HiggsConstraints} summarizes the 
status of Higgs searches at this time.  Since the Higgs mechanism
must unitarize VV scattering, there is a limited mass range for 
which the Higgs mechanism makes sense, $m_H\lesssim1000$~GeV.
This theoretical upper bound and the experimental lower bound from
the LEP direct search limits suggest that the LHC would suffice
to make the final statement about the existence of the Higgs
boson, nearly 50 years after it was first proposed.  

\begin{figure}
\begin{center}
\includegraphics[width=.7\linewidth]{IntroductionPlots/HiggsGraph_031309_MedRes.eps}
\caption{Constraints on the SM Higgs boson mass from Tevatron and 
LEP experiments either through direct searches or indirect 
evidence based on precision measurements. 
%http://www.quantumdiaries.org/2009/03/12/some-higgs-are-bigger-than-others/
}
\label{fig:HiggsConstraints}
\end{center}
\end{figure}

\section{Beyond the SM Higgs}

The Higgs mechanism, as described in the SM, conveniently solved 
several problems:
the existence of massive gauge bosons, the apparent disparity
between the electromagentic and weak forces, and the non-unitarity
of longitudinal weak boson scattering. Yet, despite its success
at describing terrestrial 
experiments, the SM fails to explain a number of phenomena 
observed in the universe.
  
It is thought that more than 95\% of the known universe consists 
of dark matter ($\sim27$\%) and dark energy
($\sim68$\%)\cite{Sullivan:2011kv}.
Since there is currently no way to explain either dark matter or 
dark energy within the SM, the SM can only attempt to explain
about 5\% of the energy of the universe.

The overabundance of matter, as opposed to anti-matter, in the 
universe, is a phenomenon known as the baryon asymmetry.
It was shown by Sakharov\cite{Sakharov:1967dj} that there are 
three necessary 
conditions a model of baryogenesis must satisfy: baryon-number 
violation, charge-symmetry and charge-parity-symmetry violation (CP-violation), and interactions 
which are out of thermal equilibrium at early stages of the 
universe.  Although it has been shown that the SM does
contain the three necessary conditions for baryogenesis, it is 
believed to be insufficient for explaining the degree of baryonic 
asymmetry in the visible 
universe~\cite{Jarlskog:1985ht,Shaposhnikov:1986jp}.  As such,
additional sources of CP-violation in the SM would provide a
promising solution to the baryon-asymmetry problem.  

The expected naturalness of electroweak symmetry breaking is
also often cited as evidence for physics beyond the SM.  
Quantum corrections to the Higgs boson mass have been found to be much 
larger than the physical Higgs boson mass~\cite{Aitchison:2007}. 
If it is to provide the 
necessary cancellations to preserve unitarity in longitudinal weak 
boson scattering, these corrections should be offset by the 
bare Higgs boson
mass in order to keep the physical mass small.  This introduces
what is known as fine tuning.  The unnaturalness of the Higgs boson mass
relative to the plank scale ($10^{19}$~GeV) is also known as the
hierarchy problem.

There are a 
number of proposed solutions to the fine tuning problem, some 
of which 
could also provide solutions to some of the problems noted above,
for example, Supersymmetry (SUSY).  
Since SUSY predicts that all fermions have a symmetry with a 
corresponding boson, 
all Feynman diagrams which provide quantum corrections to the Higgs 
boson mass have a canceling partner which removes the large quantum 
corrections\footnote{Although this was not the original motivation
for SUSY, it was later suggested to provide a solution to 
fine-tuning in the SM by Witten~\cite{Witten:1981nf}, 
Veltman~\cite{Veltman:1980mj}, and Kaul~\cite{Kaul:1981wp}.  
This is discussed in more detail elsewhere~\cite{Aitchison:2007}. }.
SUSY is also thought to provide a natural dark matter candidate and
is a prerequisite for string theory, which naturally incorporates 
gravity.  Finally, it is possible 
for SUSY to allow for additional CP-violation in the Higgs sector.
Recent work has studied this idea in the more generic framework of
type-II 2 Higgs doublet models (2HDM) and found that the amount of
additional CP-violation possible in the Higgs sector could provide
a reasonable model for baryogenesis~\cite{Shu:2013uua}.  

Other explanations of fine tuning include composite Higgs models
or Randall-Sundrum models of gravity.  Composite Higgs models 
interpret the Higgs mechanism as only an effective theory and
introduce a new strongly interacting QCD-like force above the 
electroweak scale.  It was shown by Randall and 
Sundrum~\cite{Randall:1999ee} that
higher-dimensional models with warped space-time metrics can 
provide a natural explaination of the hierarchy problem and
thus fine tuning.

\section{Summary}

Although many of the above arguments for naturalness in the SM 
are heuristic, they suggest that the Higgs sector could be a 
window to physics beyond
the SM through: the discovery of multiple scalars,
the discovery of CP-violation in Higgs interactions,
or the discovery of Higgs compositeness.  Today, the muon 
magnetic moment has been calculated and measured 
to an extremely high precision and has been used as a 
test of the SM as well as a probe for new physics.
Analogously, the Higgs boson may become the next source of high 
precision tests of the SM which may
ultimately illuminate the existence of new physics.

%overview of material in thesis
This thesis will discuss several analyses designed to search for 
a SM Higgs boson
%new resonances, especially those related to electroweak symmetry 
%breaking, 
using tools which have been developed to not only provide
increased sensitivity to signal events but also to measure 
properties of observed resonances. 
Chapter~\ref{sec:ExpSetup} will 
discuss the experimental details of the Large Hadron Collider 
(LHC) and the Compact Muon Solenoid (CMS). 
Chapter~\ref{sec:HiggsPhen} will discuss Higgs phenomenology 
at the LHC.  Chapter~\ref{sec:HZZsearches} will
present two analyses designed to search for the SM Higgs boson using the 
$ZZ\to2\ell 2q$ signature and using the $ZZ\to 4\ell$ signature. 
The latter will include the discovery and characterization of 
a new bosonic resonance using the tools developed in
Chapter~\ref{sec:HiggsPhen}.  Chapter~\ref{sec:FutureMeasurements} 
will discuss the prospects of precision measurements of Higgs boson
properties at both the LHC and a future $e^+e^-$ collider. Finally, 
Chapter~\ref{sec:Conclusions} will discuss the interpretation 
of these results in the context of the beyond the SM physics 
mentioned above.  
