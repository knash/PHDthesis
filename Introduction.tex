\chapter{Introduction}
\label{sec:intro}
\chaptermark{Introduction}

The Standard Model (SM) of particle physics \textcolor{red}{(reference)} 
is a mathematical description of the fundamental particles and their 
interactions.  Particles are described by quantized excitations of 
various types of fields.  These fields are the solutions to either the
Klein-Gordon, Dirac, Proca, or Maxwell equation which govern the time
evolution of the solutions for spin-0, spin-1/2, and spin-1 particles.
Although, bound states and possible extensions to the SM can have
larger spin, the standard model doesn't explicitly describe the interaction
of these particles.

The interactions themselves can be derived from the Standard Model lagrangian.
The electroweak interactions of the leptons, for example, are described by
the lagrangian in equation~\ref{eq:GSWlagrangian}.  The probability of the
some initial state, i, evolving into some final state, j, can be 
be expanded in powers of the coupling constants, g and g', according to 
equation equation~\ref{eq:matrixElementSquared}.  This matrix element
squared can also be represented by Feynman diagrams, figure~\ref{fyn:eeScattering} 
represents electron-electron scattering at leading order which is 
represented in the SM lagrangian by the second half of equation~\ref{eq:QED}.

\begin{center}
\begin{equation}
\mid<ee\mid e^{\frac{g}{2c_w}\bar{\Psi}_e\gamma^{\mu}(I_f^3-2s_w^2Q_f-I_f^3\gamma_5)\Psi_eZ_{\mu}}\mid ee>\mid^2
\label{eq:matrixElementSquared}
\end{equation}
\end{center}

\begin{figure}
\begin{center}
\includegraphics[width=4.0cm]{IntroductionPlots/Electron-scattering.png}
\label{fyn:eeScattering}
\caption{Feynman diagram depicting one of the processes contributing to the leading order
electron-electron scattering amplitude. A similar contribution exist where a Z boson is
exchanged instead of a photon.}
\end{center}
\end{figure}

\begin{center}
\begin{subequations}
%\begin{multiline}
\begin{equation}
\mathscr{L} = \sum_f(\bar{\Psi}_f(i\gamma^{\mu}\partial_{\mu}-m_f)\Psi_f
-eQ_f\bar{\Psi}_f\gamma^{\mu}\Psi A_{\mu})
\label{eq:QED}
\end{equation}
\begin{equation}
+\frac{g}{\sqrt{2}}\sum_i(\bar{a}_L^i\gamma^{\mu}b_L^iW_{\mu}^++\bar{b}_L^i\gamma^{\mu}a_L^iW_{\mu}^-)
\end{equation}
\begin{equation}
+\frac{g}{2c_w}\sum_f(\bar{\Psi}_f\gamma^{\mu}(I_f^3-2s_w^2Q_f-I_f^3\gamma_5)\Psi_fZ_{\mu})
\end{equation}
\begin{equation}
-\frac{1}{4}|\partial_{\mu}A_{\nu}-\partial_{\nu}A_{\mu}-ie(W_{\mu}^-W_{\nu}^+ - W_{\mu}^+W_{\nu}^-)|^2
\end{equation}
\begin{equation}
- \frac{1}{2}|\partial_{\mu}W_{\nu}^+-\partial_{\nu}W_{\mu}^+ - ie(W_{\mu}^+A_{\nu}-W_{\nu}^+A_{\mu})+ig'c_w(W_{\mu}^+Z_{\nu}-W_{\nu}^+Z_{\mu})|^2
\end{equation}
\begin{equation}
-\frac{1}{4}|\partial_{\mu}Z_{\nu}-\partial_{\nu}Z_{\mu}+ig'c_w(W_{\mu}^-W_{\nu}^+-W_{\mu}^+W_{\nu}^-)|^2
\end{equation}
\begin{equation}
-\frac{1}{2}M_{\Phi}^2\Phi^2-\frac{gM_{\Phi}^2}{8M_{W}}\Phi^3-\frac{g'^2M_{\Phi}^2}{32M_W}\Phi^4
\end{equation}
\begin{equation}
+|M_WW_{\mu}^++\frac{g}{2}\Phi W_{\mu}^+|^2+\frac{1}{2}|\partial_{\mu}\Phi+iM_ZZ_{\mu}+\frac{ig}{2c_w}\Phi Z_{\mu}|^2
\end{equation}
\begin{equation}
-\sum_f\frac{g}{2}\frac{m_f}{M_W}\bar{\Psi_f}\Psi_f\Phi
\end{equation}
%\end{multiline}
\label{eq:GSWlagrangian}
\end{subequations}
\end{center}

Starting from this framework, many experimental measurements have been accurately 
predicted.  One famous example is the anomalous magnetic moment of the muon 
\textcolor{red}{(electron?)}, first calculated by Julian Schwinger in \textcolor{red}{19??}
in the context of Quantum Electrodynmics (QED). 

Before the advent of quantum field theory the magnetic moment of the 
muon and electron were thought to be exactly two due the quantized spin 
they carry.  However, the quantum fluctuations predicted by quantum field 
theory and QED, the theory of the electromagnetic interactions, add small 
corrections the muon's magnetic moment. Today, this quantity has been calculated 
and measured to \textcolor{red}{11} significant digits.  

\section{Electroweak Symmetry Breaking}
\label{sec:Electroweak Symmetry Breaking}

It was known in non-quantum systems that spontaneously broken symmetry
could provide a theoretical framework for explaining massive gauge 
bosons\cite{}. In early 1960's these ideas were studied in the context
of quantum field theories.  It was shown that a complex scalar field 
whos potential was particularly chosen, could not only spontaneously 
break a gauge symmetry but generate gauge boson masses through the 
interaction of this field with the gauge boson\cite{}.  Most notibly, 
Higgs suggested that this would also predict the presence of a new 
massive scalar particle whos interactions with the gauge bosons 
scaled with their mass\cite{}.  

\subsection{Electroweak Interactions}
\label{sec:Electroweak interactions}

The gauge structure of the weak interactions demonstrated by Glashow 
\textcolor{red}{and Salam(?)}\cite{} and Weinberg showed that the 
electromagentic and weak interactions could be unified and implemented
the Higgs mechanism into was is today know to be the Standard Model (SM)
of electroweak interactions.  Together, this is referred to as the 
Glashow-Salam-Weinberg (GSW) model.

The GSW model had concrete experimentally observable predictions.  Since
the gauge symmetry reprossible for generating the interactions was
$U(1)\otimes SU(2)$ which has a total of four generators, a third guage 
boson should exist.  The GSW model specifically that this boson would 
be neutral and mediate neutrino-electron interactions through the 
diagram in Figure~\ref{}.  These so-called neutral currents were later
discovered in \textcolor{red}{197?} by the \textcolor{red}{--- collaboration}.
Both the charged and neutral weak gauge bosons were later discovered
through means of direct detection\cite{}.  

The experimental verification of the Higgs boson and the ...

Another experimental signature of the GSW model was that a chargeless,
colorless, spinless, massive boson, the Higgs boson, should exist.  
Except for its mass all properties about this particle could be calculated
from first principles, see Section~\ref{sec:HiggsPhen}.  Several accelerators
have been built to confirm its existance, Large Electron-Positron collider (LEP), 
the Tevatron, and most recently the Large Hadron Collider.  

\section{The Higgs boson}
\label{sec:The Higgs boson}

\begin{itemize}
\item Why it is critical to observe and understand the properties of the Higgs
and EWSB.

Both performed direct searches for the SM Higgs boson to either 
discover or rule out the existance of a Higgs boson as specific
masses.  Since the lifetime of a Higgs is \textcolor{red}{universally}
small searches are done looking for its decay products.  

\item Describe LEP searches for the Higgs.

\item Describe Tevatron searches for the Higgs.

\item Precision measurements of the Z boson ...
Global fits suggest that ... 

\item What could be beyond the SM ...

\end{itemize}

Despite its success at describing terrestrial experiments, the SM fails 
to explain a number of phenomena observed in the universe.  
%In was first
%suggested in 1932 that dark matter, matter which interacts very weakly
%with the electromagnetic force mediator, the photon, might make up a 
%large fraction of the mass of the universe.  Many phenomena have been 
%able to show indirect evidence of dark matter, the distribution of 
%orbital velocities of star in the Milky Way galaxy, the distribution of 
%orbital velocities of galaxies in galaxy clusters, gravitational lensing, 
%tempurature distributions of hot gases in various astronomical objects, 
%and the anisotropies of the cosmic microwave background\footnote{what is
%the CMB?}\footnote{All of these sources of evidence are currently taken
%from the dark matter wikipedia article and should be cited by papers...
%maybe ask someone who knows a lot about dark matter.}.  
% NEW PARAGRAPH
%More recently, it was shown that the rate of expansion of the universe, 
%as percieved by red-shift distributions of distant type-Ia supernovae, is 
%accelerating. The most common explaination of this phenomenon is an energy 
%source that fills the universe.  \textcolor{red}{Little is known about dark 
%energy although the vacuum energy of the ... could explain ... blah.}
It is thought that more than 95\% of the known universe consists of dark 
matter ($\sim25$\%) and dark energy ($\sim75$\%)\cite{??}.  Since there is 
currently no way to explain either dark matter or dark energy with the SM, 
only about 4\% of the constituents of the universe can be explained by the 
SM.  

The overabundance of matter, as opposed to anti-matter, in the 
universe, a phenomenon known as baryon assymetry, is thought to not be 
attributable to any known process in the SM.
It was shown by Sakharov\cite{??} that there are three necessary conditions 
a model of baryogenesis must satisfy, baryon-number violation, 
charge-symmetry (C-symmetry) violation, charge-parity-symmetry (CP-symmetry)
violation, and interaction which are out of thermal equilibrium at early
stages of the universe.  Although it has been shown that the SM does
contain the three necessary conditions for baryogenesis, it is believed
to be insufficient for explaining the degree of baryonic assymetry in the 
visible universe.  As such, additional sources of CP-violation in the SM would 
thus be a promising solution to the baryon-assymetry problem.  

The standard model also does not include a quantum description of the 
gravitational force.  It was shown by Randall and Sundrum that extra dimensional
models with warped space-time metrics can provide a natural explaination of the
heirarchy of not only the gravitational force and the weak force, but also
heirarchy between the expected bare higgs mass and its physical mass.  

Part of the success of QED was its relation between the dynamics of the 
electromagnetic interaction and a U(1) gauge symmetry.  This enspired 
others to find similar gauge symmetry which could describe the weak 
interactions.  However, unlike the electromagnetic interactions, the 
weak interactions are mediated by massive gauge boson, 
\textcolor{red}{known from the fermi theory of interactions}.  The 
complication of having massive gauge mediators was 
related to the fact that mass terms were known to not respect gauge 
symmetries, a caveat that could be overlooked for the electromagnetic 
interactions. However, understanding how to resolve this problem and the
vastly different scales which the electromagnetic and weak forces operate
at was a major triumph of what is currently known as the SM.

This thesis will discuss several analyses designed to search for the
new resonances, especially those related to electroweak symmetry 
breaking. The sections are organized as follows: chapter~\ref{sec:LHC}, will 
discuss the experiment details of the Large Hadron Collider (LHC) and 
the Compact Muon Solenoid (CMS); chapter~\ref{sec:HiggsPhen} will discuss 
the Higgs phenomenology at the LHC; chapter~\ref{sec:HZZsearches} will
present two analyses designed to search for the SM Higgs at high mass
using the $ZZ\to2\ell 2q$ signature and over a broad range of masses
using the $ZZ\to 4\ell$ signature; chapter~\ref{sec:HiggsProp} will discuss
properties measurments of the newly discovered boson using the MELA 
technique as well as prospect for high precision property measurements
at either the LHC or a future $e^+e^-$ collider; finally, chapter will
discuss the interpretation of these results in the context of the 
beyond the SM physics mentioned above.  
